% ==========================================================
% 7. Data Parallelism (DP)
% ==========================================================
\section{Data Parallelism (DP)}
\label{sec:dp}

In data parallelism, each replica holds a full copy of the model, but
processes a different subset of the batch. Conceptually, each device runs
the same forward and backward computation as in the single-node setting
(Section~\ref{sec:sn}), but on a different mini-batch. After the backward pass,
gradients are synchronized across replicas via All-Reduce so that the
optimizer update is identical to what would have been obtained on a
single node with the full batch.

From the point of view of a single replica, the forward graph inside each
Transformer block is therefore identical to the one in
Section~\ref{sec:sn} (single-node), or to Section~\ref{sec:tp} (tensor-parallel) if TP is also
enabled. The only change is the input: instead of the full batch
$\mathbf{X} \in \mathbb{R}^{B \times S \times D}$, replica $d$ sees a
local shard $\mathbf{X}_d \in \mathbb{R}^{B_{\text{local}} \times S \times D}$,
with $B_{\text{local}} = B / N_D$.

We denote the number of data-parallel replicas by $N_D$. Devices are
indexed by $d \in \{0,\dots,N_D-1\}$.

\textbf{Key ideas:}
\begin{itemize}
  \item Each device has a full copy of the model parameters (weights and
        optimizer state).
  \item The global batch of size $B$ is split into $N_D$ local batches of
        size $B_{\text{local}} = B / N_D$.
  \item The forward pass on each device is identical to the single-node
        computation in Section~\ref{sec:sn}, but uses only its local batch.
  \item The backward pass computes local gradients
        $\nabla W^{(d)}$ on each device.
  \item All-Reduce over all data-parallel replicas averages or sums the
        gradients, producing the same effective update as a single-node
        run with batch size $B$.
\end{itemize}

In contrast to tensor parallelism (Section~\ref{sec:tp}), DP does not shard weight
matrices or activations; instead, it replicates the entire model and
splits only the data.

% ------------------------ 7.1 Overall DP Flow -------------------------
\subsection{Overall DP Training Flow}

Figure~\ref{fig:dp_overall_flow} shows a high-level view of data
parallelism applied to a Transformer layer. Compared to the single-node
overview (Figure~\ref{fig:single_node_overall}), the main difference is
that we now have $N_D$ identical copies of the block, each processing a
different input shard $\mathbf{X}_d$ and producing local predictions and
losses.

For a single training step, the sequence of operations is:

\begin{enumerate}
  \item \textbf{Batch sharding.} The input batch of size $B$ is split
        into $N_D$ local batches, each of size $B_{\text{local}}$:
        \[
          \{\mathbf{X}_0,\dots,\mathbf{X}_{N_D-1}\},
          \qquad
          \mathbf{X}_d \in \mathbb{R}^{B_{\text{local}} \times S \times D_{\text{in}}}.
        \]
        Each device $d$ also receives the corresponding target tokens
        $\mathbf{Y}_d$.
  \item \textbf{Local forward pass.} On each device, the full Transformer
        stack (input embedding, MHA, MLP, output projection) is applied to
        its local shard $\mathbf{X}_d$ exactly as in Section~\ref{sec:sn}. If tensor
        parallelism from Section~\ref{sec:tp} is also used, then each replica runs
        the same TP-augmented forward graph on $\mathbf{X}_d$. In either
        case, the structure of the forward pass is unchanged; only the
        batch dimension and the presence of gradient synchronization
        differ. Each device produces local logits, probabilities, and
        loss $\mathcal{L}_d$.
  \item \textbf{Local backward pass.} The loss $\mathcal{L}_d$ is
        backpropagated locally, yielding gradients
        $\nabla W^{(d)}$ for all parameters on device $d$.
  \item \textbf{Gradient synchronization (All-Reduce).} For each
        parameter tensor $W$, an All-Reduce over the data-parallel group
        is performed:
        \[
          \nabla W
            = \frac{1}{N_D}
              \sum_{d=0}^{N_D-1} \nabla W^{(d)}.
        \]
        After this step, all replicas hold identical averaged gradients
        $\nabla W$.
  \item \textbf{Optimizer update.} Each device applies the same optimizer
        (SGD, Adam, etc.) to its local copy of the parameters using the
        synchronized gradients. Because the gradients are identical across
        devices, the updated weights are also identical.
\end{enumerate}

Apart from the All-Reduce in step 4, the computations on each device are
identical copies of the single-node graph.

\begin{figure}[htbp]
  \centering
  \resizebox{\linewidth}{!}{%
\begin{tikzpicture}[
    node distance=2.5cm,
    >=stealth,
    block/.style={rectangle, draw=black, fill=white, text width=5em, text centered, rounded corners, minimum height=8em, font=\bfseries},
    forward/.style={-{Stealth[length=2mm]}, thick, black},
    backward/.style={-{Stealth[length=2mm]}, thick, black, densely dashed},
    dpcomm/.style={<->, thick, red!70, dashed},
    io/.style={text centered, font=\bfseries}
]
    % Title
    \node[font=\Large\bfseries] at (7, 12) {Data Parallel Transformer Flow};

    % ========== Node i (Upper) ==========
    \node (input_i) [io] at (0, 10) {$\mathbf{X}_i$};
    \node (encoding_i) [block, right of=input_i, yshift=-3em] {Input\\Encoding$_i$};
    \node (mha_i) [block, right of=encoding_i] {MHA$_i$};
    \node (mlp_i) [block, right of=mha_i] {FFN$_i$};
    \node (output_i) [block, right of=mlp_i] {Output\\Projection$_i$};
    \node (pred_i) [io, right of=output_i, yshift=3em] {$\mathbf{Y}_i$};
    \node (loss_i) [align=center, io, right of=output_i] {\small LOSS:\\$\mathcal{L}_i$};
    \node (gradient_i) [io, right of=output_i, yshift=-3em] {$\mathbf{dY}_i$};

    % Forward arrows - Node i
    \draw [forward] (input_i) -- ([yshift=3em]encoding_i.west);
    \draw [forward] ([yshift=3em]encoding_i.east) -- ([yshift=3em]mha_i.west);
    \draw [forward] ([yshift=3em]mha_i.east) -- ([yshift=3em]mlp_i.west);
    \draw [forward] ([yshift=3em]mlp_i.east) -- ([yshift=3em]output_i.west);
    \draw [forward] ([yshift=3em]output_i.east) -- (pred_i);
    \draw [forward] (pred_i) -- (loss_i);
    \draw [backward] (loss_i) -- (gradient_i);

    % Backward arrows - Node i
    \draw [backward] (gradient_i) -- ([yshift=-3em]output_i.east);
    \draw [backward] ([yshift=-3em]output_i.west) -- ([yshift=-3em]mlp_i.east);
    \draw [backward] ([yshift=-3em]mlp_i.west) -- ([yshift=-3em]mha_i.east);
    \draw [backward] ([yshift=-3em]mha_i.west) -- ([yshift=-3em]encoding_i.east);

    % Brace for layer repetition - Node i
    \draw[decorate, decoration={brace, amplitude=10pt}]
        ([yshift=1.0em]mha_i.north west) -- ([yshift=1.0em]mlp_i.north east)
        node[midway, above=12pt, font=\normalsize] {$N$ layers};

    % ========== Node j (Lower) ==========
    \node (input_j) [io] at (0, 6) {$\mathbf{X}_j$};
    \node (encoding_j) [block, right of=input_j, yshift=-3em] {Input\\Encoding$_j$};
    \node (mha_j) [block, right of=encoding_j] {MHA$_j$};
    \node (mlp_j) [block, right of=mha_j] {FFN$_j$};
    \node (output_j) [block, right of=mlp_j] {Output\\Projection$_j$};
    \node (pred_j) [io, right of=output_j, yshift=3em] {$\mathbf{Y}_j$};
    \node (loss_j) [align=center, io, right of=output_j] {\small LOSS:\\$\mathcal{L}_j$};
    \node (gradient_j) [io, right of=output_j, yshift=-3em] {$\mathbf{dY}_j$};

    % Forward arrows - Node j
    \draw [forward] (input_j) -- ([yshift=3em]encoding_j.west);
    \draw [forward] ([yshift=3em]encoding_j.east) -- ([yshift=3em]mha_j.west);
    \draw [forward] ([yshift=3em]mha_j.east) -- ([yshift=3em]mlp_j.west);
    \draw [forward] ([yshift=3em]mlp_j.east) -- ([yshift=3em]output_j.west);
    \draw [forward] ([yshift=3em]output_j.east) -- (pred_j);
    \draw [forward] (pred_j) -- (loss_j);
    \draw [backward] (loss_j) -- (gradient_j);

    % Backward arrows - Node j
    \draw [backward] (gradient_j) -- ([yshift=-3em]output_j.east);
    \draw [backward] ([yshift=-3em]output_j.west) -- ([yshift=-3em]mlp_j.east);
    \draw [backward] ([yshift=-3em]mlp_j.west) -- ([yshift=-3em]mha_j.east);
    \draw [backward] ([yshift=-3em]mha_j.west) -- ([yshift=-3em]encoding_j.east);

    % ========== DP Communications ==========
    \draw [dpcomm] (mha_i.south) -- (mha_j.north) node[midway, right, font=\tiny, align=left] {All-Reduce\\$(\nabla W_{Q,K,V,O})$};
    \draw [dpcomm] (mlp_i.south) -- (mlp_j.north) node[midway, right, font=\tiny, align=left] {All-Reduce\\$(\nabla W_{\text{up},\text{down}})$};

    % Labels (Legend)
    \coordinate (legend) at ([xshift=11.5cm, yshift=8.5em]input_i);

    % Forward (작은 화살표 + 작은 글자)
    \draw[
        forward,
        -{Stealth[length=1.2mm,width=1.4mm]}, % 화살표 더 작게
        line width=0.3pt                       % 선 더 얇게
    ] (legend) -- ++(0.8,0)
      node[right, font=\scriptsize] {Forward};

    % Backward
    \draw[
        backward,
        -{Stealth[length=1.2mm,width=1.4mm]},
        line width=0.3pt
    ] ([yshift=-0.4cm]legend) -- ++(0.8,0)
      node[right, font=\scriptsize] {Backward};

    % DP Comm
    \draw[
        dpcomm,
        line width=0.3pt
    ] ([yshift=-0.8cm]legend) -- ++(0.8,0)
      node[right, font=\scriptsize] {DP Comm};
\end{tikzpicture}%
}
  \caption{Overall Transformer layer under data parallelism. Each device
  holds a full copy of the model and processes a different shard of the
  input batch ($\mathbf{X}_i$, $\mathbf{X}_j$, \dots). Forward and
  backward passes are performed locally as in the single-node case, and
  gradients for each block (MHA, MLP, output projection) are synchronized
  across replicas via All-Reduce.}
  \label{fig:dp_overall_flow}
\end{figure}


% ------------------------ 7.2 Relation to Single-Node -----------------
\subsection{Relationship to the Single-Node Computation}

It is often useful to think of DP as a pure “wrapper” around the
single-node computation from Section~\ref{sec:sn}:

\begin{itemize}
  \item \textbf{Same computation graph per replica.}
        For any given layer (MHA, MLP, output projection), the forward and
        backward graphs on a single device are identical to those in the
        single-node diagrams (Figures~\ref{fig:single_node_mha_forward},
        \ref{fig:single_node_mha_backward},
        \ref{fig:single_node_mlp_forward},
        \ref{fig:single_node_mlp_backward}, etc.), or to their
        tensor-parallel counterparts in Section~\ref{sec:tp} when TP is enabled.
        Only the input batch is different:
        $\mathbf{X}_d$ is a shard of the global batch.
  \item \textbf{Different mini-batches.}
        The only difference in the forward pass is that each replica sees
        a different chunk of the global batch. There is no communication
        needed for forward activations in DP.
  \item \textbf{Gradient aggregation only.}
        In the backward pass, each replica computes local gradients as if
        it were running independently. Only after all local gradients have
        been computed do we introduce communication: an All-Reduce over
        all replicas for each parameter tensor.
  \item \textbf{Equivalence to larger batch.}
        When gradients are averaged across replicas, the resulting update
        is mathematically equivalent to a single-node run with batch size
        $B = N_D \cdot B_{\text{local}}$ (ignoring subtle differences from
        e.g.\ dropout noise).
\end{itemize}

In contrast, tensor parallelism (Section~\ref{sec:tp}) changes the graph inside each
layer by sharding weight matrices and adding collectives in the middle of
the forward and backward passes. DP keeps the layer graphs intact and
adds communication only at the level of gradients.

% ------------------------ 7.3 MHA Backward under DP -------------------
\subsection{MHA Backward under DP}

For the multi-head attention block, the local backward pass on each device
is exactly the same as in
Figure~\ref{fig:single_node_mha_backward}: gradients flow from the loss
through the output projection, attention heads, Q/K/V projections, and
layer normalization back to the input $\mathbf{X}$.

The only DP-specific difference is how gradients with respect to the MHA
parameters are combined across replicas. Let
$W_Q, W_K, W_V, W_O$ be the query, key, value, and output projection
matrices. On device $d$, the local backward pass computes
$\nabla W_Q^{(d)}, \nabla W_K^{(d)}, \nabla W_V^{(d)}, \nabla W_O^{(d)}$.
An All-Reduce across all data-parallel replicas then aggregates these
gradients:
\[
  \nabla W_Q
    = \frac{1}{N_D} \sum_{d=0}^{N_D-1} \nabla W_Q^{(d)},\quad
  \nabla W_K
    = \frac{1}{N_D} \sum_{d=0}^{N_D-1} \nabla W_K^{(d)},
\]
\[
  \nabla W_V
    = \frac{1}{N_D} \sum_{d=0}^{N_D-1} \nabla W_V^{(d)},\quad
  \nabla W_O
    = \frac{1}{N_D} \sum_{d=0}^{N_D-1} \nabla W_O^{(d)}.
\]
After this synchronization, each device holds the same averaged gradients
$\nabla W_Q, \nabla W_K, \nabla W_V, \nabla W_O$.

Figure~\ref{fig:mha_backward_dp} illustrates this process: the internal
nodes and shapes inside the MHA block are identical to
Figure~\ref{fig:single_node_mha_backward}, but additional DP-specific
boxes and red dashed arrows mark the All-Reduce operations over replica
gradients.

\begin{landscape}
\begin{figure}[p]
  % no \centering here to avoid compilation issues
  \resizebox{\linewidth}{!}{%
\begin{tikzpicture}[
  every node/.style={transform shape},
  >=stealth,
  auxnode/.style={draw, rectangle, fill=white, minimum height=6mm, inner sep=2pt, font=\footnotesize, align=center},
  mulnode/.style={draw, circle, fill=white, minimum size=6mm, font=\footnotesize, align=center},
  addnode/.style={draw, circle, fill=white, minimum size=6mm, font=\footnotesize, align=center},
  sumnode/.style={draw, circle, fill=white, minimum size=6mm, font=\tiny, align=center},
  arnode/.style={draw, rectangle, fill=red!30, minimum height=6mm, inner sep=2pt, font=\footnotesize, align=center, thick, draw=red!70},
  flow/.style={->, thick, black!85},
  flow2/.style={->, double, thick, black!85},
  dimlabel/.style={font=\scriptsize, inner sep=1pt, align=center},
  gradflow/.style={->, thick, black!85},
  gradweight/.style={->, thick, black!85},
  dpgradweight/.style={->, thick, black!85},
  dpcomm/.style={<->, thick, red!70, dashed}
]

\begin{scope}[xscale=1.35, yscale=1.2]

\def\yoffset{-1.0}
\def\dVXyoffset{-6.5}

\coordinate (Grad_Aout_B) at (17.5, \yoffset);
\coordinate (dDrop_center) at (15.9, \yoffset);
\coordinate (ProjGradSplit) at (14.3, \yoffset);
\coordinate (dOProj_center) at (12.7, \yoffset);
\coordinate (C_center) at (11.1, \yoffset);
\coordinate (R_center) at (9.0, \yoffset);
\coordinate (dPV_AS_calc_center) at (7.5, \yoffset);
\coordinate (dSoft_center) at (5.3, \yoffset);
\coordinate (dSM_calc_center) at (3.3, \yoffset);
\coordinate (dV_calc_center) at (8.2, \dVXyoffset+\yoffset);
\coordinate (R_V_bwd_center) at (5.9, \dVXyoffset+\yoffset);
\coordinate (dVX_calc_center) at (3.9, \dVXyoffset+\yoffset);
\coordinate (dQK_calc_Q_center) at (1.7, \yoffset);
\coordinate (dQK_calc_K_center) at (1.7, -3.3+\yoffset);
\coordinate (K_BWD_input_center) at (1.7, 1.0+\yoffset);
\coordinate (T_Q_bwd_center) at (1.7, -4.3+\yoffset);

\node[font=\Large\bfseries] at (8, 4.6+\yoffset) {Multi-Head Attention Backward Pass (Data Parallel)};

\node (Grad_Aout_B) at (18.5, \yoffset) {$\mathbf{dA}_{\text{out}}$\\$[B,S,D]$};
\node[auxnode] (DO) at (dDrop_center) {DO};
\node[mulnode] (dOProj) at (dOProj_center) {$\bullet$};

\node[auxnode] (T_WO) [below=0.9cm of dOProj] {T};
\node[dimlabel] (WO_BWD) [below=0.45cm of T_WO] {$\widetilde{\mathbf{W}}_{O}$\\$[D,D]$};

\node[mulnode] (dWO_calc) at ($(ProjGradSplit)+(0, 1.6)$) {$\bullet$};

% Add AR node before dWO
\node[arnode] (AR_WO) [left=0.8cm of dWO_calc] {AR};
\draw[dpcomm] ([yshift=-0.5cm]AR_WO.south) -- ([yshift=-0.05cm]AR_WO.south);
\draw[dpcomm] ([yshift=0.05cm]AR_WO.north) -- ([yshift=0.5cm]AR_WO.north);
\node[font=\tiny, red!70, align=center] at ([yshift=-0.65cm]AR_WO.south) {DP\\nodes};
\node[font=\tiny, red!70, align=center] at ([yshift=0.65cm]AR_WO.north) {DP\\nodes};

\node[align=center, left=0.8cm of AR_WO]
  (dWO_GRAD) {$\mathbf{d}\widetilde{\mathbf{W}}_{O}$\\$[D,D]$};
\draw[dpgradweight] (dWO_calc) -- (AR_WO);
\draw[gradflow] (AR_WO) -- (dWO_GRAD);

\node[auxnode] (T_AO_in) [right=1.4cm of dWO_calc] {T};
\node[dimlabel] (AO_in_local_label) [right=1.6cm of T_AO_in] {$\mathbf{AO}_{\text{cat}}$\\$[B,S,D]$};

\node[auxnode] (C) at (C_center) {dC};
\node[auxnode] (R) at (R_center) {R};

\node[mulnode] (dPV_AS_calc) at (dPV_AS_calc_center) {$\bullet$};
\node[auxnode] (dSoft) at (dSoft_center) {dS};
\node[auxnode] (dSM_calc) at (dSM_calc_center) {dSM};

\node[mulnode] (dQK_calc_Q) at (dQK_calc_Q_center) {$\bullet$};
\node[mulnode] (dQX_proj_calc) [left=5.8cm of dQK_calc_Q] {$\bullet$};

\node[mulnode] (dQK_calc_K) at (dQK_calc_K_center) {$\bullet$};
\node[mulnode] (dKX_proj_calc) [left=5.8cm of dQK_calc_K] {$\bullet$};

\node[dimlabel] (K_BWD_input) at (K_BWD_input_center) {$\mathbf{K}$\\$[B,N_H,S,D_h]$};
\node[auxnode] (T_Q_bwd) at (T_Q_bwd_center) {T};
\node[dimlabel] (Q_BWD_input) [below=0.35cm of T_Q_bwd] {$\mathbf{Q}$\\$[B,N_H,S,D_h]$};

\node[dimlabel] (V_FWD) [above=1.7cm of dPV_AS_calc] {$\mathbf{V}$\\$[B,N_H,S,D_h]$};
\node[auxnode] (T_V_bwd) [below=0.25cm of V_FWD] {T};

\node[mulnode] (dV_calc) at (dV_calc_center) {$\bullet$};
\node[auxnode] (T_AS_bwd) [right=1.5cm of dV_calc] {T};
\node[dimlabel] (AS_BWD_for_V) [right=0.5cm of T_AS_bwd] {$\mathbf{AS}$\\$[B,N_H,S,S]$};

\node[auxnode] (R_V_bwd) at (R_V_bwd_center) {R};
\node[mulnode] (dVX_calc) at (dVX_calc_center) {$\bullet$};

\node[auxnode] (T_WV) [above=0.35cm of dVX_calc] {T};
\node[dimlabel] (WV_BWD) [above=0.3cm of T_WV] {$\widetilde{\mathbf{W}}_{V}$\\$[D,D]$};

\node[sumnode] (Sum_dBO) [below=1.5cm of ProjGradSplit] {$\sum_{B, S}$};

% Add AR node before dBO
\node[arnode] (AR_BO) [below=0.6cm of Sum_dBO] {AR};
\draw[dpcomm] ([xshift=-0.4cm]AR_BO.west) -- ([xshift=-0.05cm]AR_BO.west);
\draw[dpcomm] ([xshift=0.05cm]AR_BO.east) -- ([xshift=0.4cm]AR_BO.east);
\node[font=\tiny, red!70, align=center] at ([xshift=-0.55cm]AR_BO.west) {DP\\nodes};
\node[font=\tiny, red!70, align=center] at ([xshift=0.55cm]AR_BO.east) {DP\\nodes};

\node[align=center] (dBO) [below=0.6cm of AR_BO] {$\mathbf{d}\widetilde{\mathbf{b}}_{O}$\\$[D]$};
\draw[dpgradweight] (Sum_dBO) -- (AR_BO);
\draw[gradflow] (AR_BO) -- (dBO);

\draw[gradflow] (Grad_Aout_B) -- (DO)
  node[dimlabel, midway, above]{$\mathbf{dA}_{\text{out}}$\\$[B,S,D]$};

\draw[gradflow] (DO) -- (dOProj)
  node[dimlabel, pos=0.25, above]{$\mathbf{dAO}_{\text{bias}}$\\$=\mathbf{dAO}_{\text{lin}}$\\$[B,S,D]$};

\draw[gradflow] (ProjGradSplit) -- (dWO_calc.south);
\draw[gradflow] (ProjGradSplit) -- ([yshift=-0.75cm]ProjGradSplit) -| (Sum_dBO.north);

\draw[gradflow] (dOProj) -- (C)
  node[dimlabel, midway, above]{$\mathbf{dAO}_{\text{cat}}$\\$[B,S,D]$};
\draw[gradflow] (C) -- (R)
  node[dimlabel, midway, above]{$[B,S,N_H,D_h]$};

\coordinate (R_split_point) at ($(dPV_AS_calc)!0.5!(R)$);
\draw[gradflow] (R.west) -- (dPV_AS_calc.east);
\draw[flow2] (R_split_point) -- (dV_calc.north)
  node[dimlabel, midway, right]{$\mathbf{dAO}_{\text{heads}}$\\$[B,N_H,S,D_h]$};

\draw[gradflow] (V_FWD.south) -- (T_V_bwd.north);
\draw[flow2] (T_V_bwd.south) -- (dPV_AS_calc.north)
  node[dimlabel, midway, right]{$[B,N_H,D_h,S]$};
\draw[gradflow] (dPV_AS_calc.west) -- (dSoft.east)
  node[dimlabel, midway, above]{$\mathbf{dAS}$\\$[B,N_H,S,S]$};

\node (AS_BWD_dS) [dimlabel, above=0.5cm of dSoft] {$\mathbf{AS}$\\$[B,N_H,S,S]$};
\draw[gradflow] (AS_BWD_dS.south) -- (dSoft.north);
\draw[gradflow] (dSoft.west) -- (dSM_calc.east)
  node[dimlabel, midway, above]{$[B,N_H,S,S]$};

\coordinate (dA_Split_X) at ($(dSM_calc_center)!0.5!(dQK_calc_Q_center)$);
\coordinate (dA_Split) at (dA_Split_X |- dQK_calc_Q.east);
\draw[gradflow] (dSM_calc.west) -- (dQK_calc_Q.east);
\draw[flow2] (dA_Split) -- (dA_Split |- dQK_calc_K.east) -- (dQK_calc_K.east)
  node[dimlabel, pos=-1.5, above, yshift=15]{$\mathbf{dA}$\\$[B,N_H,S,S]$};

\draw[flow2] (K_BWD_input.south) -- (dQK_calc_Q.north);
\draw[gradweight] (dQK_calc_Q) -- (dQX_proj_calc)
  node[dimlabel, midway, below]{$\mathbf{dQ}$\\$[B,N_H,S,D_h]$};

\node[auxnode] (T_WQ_bwd) [above=0.45cm of dQX_proj_calc] {T};
\node[dimlabel] (WQ_bwd) [above=0.45cm of T_WQ_bwd] {$\widetilde{\mathbf{W}}_{Q}$\\$[D,D]$};
\draw[flow] (WQ_bwd) -- (T_WQ_bwd);
\draw[flow2] (T_WQ_bwd.south) -- (dQX_proj_calc.north)
  node[dimlabel, midway, right]{$[D,D]$};

\draw[flow] (Q_BWD_input.north) -- (T_Q_bwd.south);
\draw[flow] (T_Q_bwd.north) -- (dQK_calc_K.south)
  node[dimlabel, pos=0.55, right]{$[B,N_H,D_h,S]$};

\node[auxnode] (T_dK) at ($(dQK_calc_K)!0.35!(dKX_proj_calc)$) {T};
\node[auxnode] (R_dK_mid) at ($(T_dK)!0.5!(dKX_proj_calc)$) {R};

\draw[gradweight] (dQK_calc_K) -- (T_dK)
  node[dimlabel, midway, above]{$\mathbf{dK^T}$\\$[B,N_H,D_h,S]$};
\draw[gradweight] (T_dK) -- (R_dK_mid)
  node[dimlabel, midway, above]{$[B,S,N_H,D_h]$};
\draw[gradweight] (R_dK_mid) -- (dKX_proj_calc)
  node[dimlabel, midway, above]{$\mathbf{dK}$\\$[B,S,D]$};

\node[auxnode] (T_WK_bwd) [above=0.55cm of dKX_proj_calc] {T};
\node[dimlabel] (WK_bwd) [above=0.45cm of T_WK_bwd] {$\widetilde{\mathbf{W}}_{K}$\\$[D,D]$};
\draw[gradflow] (WK_bwd) -- (T_WK_bwd);
\draw[flow2] (T_WK_bwd.south) -- (dKX_proj_calc.north);
\draw[gradflow] (AS_BWD_for_V.west) -- (T_AS_bwd.east);
\draw[gradflow] (T_AS_bwd.west) -- (dV_calc.east)
  node[dimlabel, midway, above]{$[B,N_H,S,S]$};
\draw[gradflow] (dV_calc.west) -- (R_V_bwd.east)
  node[dimlabel, midway, above]{$\mathbf{dV}$\\$[B,N_H,S,D_h]$};
\draw[gradflow] (R_V_bwd) -- (dVX_calc.east)
  node[dimlabel, midway, above]{$[B,S,D]$};

\draw[gradflow] (WV_BWD) -- (T_WV);
\draw[flow2] (T_WV) -- (dVX_calc.north)
  node[dimlabel, midway, right]{$[D,D]$};

\node[addnode] (Sum_dXnorm) [left=1.8cm of dKX_proj_calc] {$+$};

\draw[gradweight] (dQX_proj_calc.west) -| node[dimlabel, pos=0.7, left]{$\mathbf{dX}_{\text{norm},\mathbf{Q}}$\\$[B,S,D]$} (Sum_dXnorm.north);
\draw[gradweight] (dKX_proj_calc.west) -- node[dimlabel, midway, above]{$\mathbf{dX}_{\text{norm},\mathbf{K}}$\\$[B,S,D]$} (Sum_dXnorm.east);
\draw[gradweight] (dVX_calc.west) -| node[dimlabel, pos=0.9, left]{$\mathbf{dX}_{\text{norm},\mathbf{V}}$\\$[B,S,D]$} (Sum_dXnorm.south);

\coordinate (dV_branch) at ($(R_V_bwd.west)!0.52!(dVX_calc.east)$);
\node[mulnode] (dWV_mul) at ($(dV_branch)+(0,-1.6cm)$) {$\bullet$};
\draw[flow2] (dV_branch) -- (dWV_mul.north);

\node[auxnode] (T_Xnorm) [right=1.1cm of dWV_mul] {T};
\node[dimlabel] (Xnorm_local) [right=0.8cm of T_Xnorm] {$\mathbf{X}_{\text{norm}}$\\$[B,S,D]$};
\draw[gradflow] (Xnorm_local) -- (T_Xnorm);
\draw[gradflow] (T_Xnorm.west) -- (dWV_mul.east)
  node[dimlabel, midway, above]{$[B,D,S]$};

% Add AR node before dWV
\node[arnode] (AR_WV) [left=0.8cm of dWV_mul] {AR};
\draw[dpcomm] ([yshift=-0.5cm]AR_WV.south) -- ([yshift=-0.05cm]AR_WV.south);
\draw[dpcomm] ([yshift=0.05cm]AR_WV.north) -- ([yshift=0.5cm]AR_WV.north);
\node[font=\tiny, red!70, align=center] at ([yshift=-0.65cm]AR_WV.south) {DP\\nodes};
\node[font=\tiny, red!70, align=center] at ([yshift=0.65cm]AR_WV.north) {DP\\nodes};

\node[align=center] (dWV_out) [left=0.8cm of AR_WV] {$\mathbf{d}\widetilde{\mathbf{W}}_{V}$\\$[D,D]$};
\draw[dpgradweight] (dWV_mul.west) -- (AR_WV);
\draw[gradflow] (AR_WV) -- (dWV_out);

\coordinate (dQ_branch) at ($(dQK_calc_Q.east)!0.50!(dQX_proj_calc.west)$);
\node[mulnode] (dWQ_mul) at ($(dQ_branch)+(0,3.5cm)$) {$\bullet$};
\node[auxnode] (R_dQ_for_WQ) at ($(dWQ_mul)+(0,-1.0cm)$) {R};
\draw[gradflow]  (dQ_branch) -- (R_dQ_for_WQ.south);
\draw[flow2] (R_dQ_for_WQ.north) -- (dWQ_mul.south)
  node[dimlabel, midway, right]{$[B,S,D]$};

\node[auxnode] (T_XnormQ) [right=1.1cm of dWQ_mul] {T};
\node[dimlabel] (Xnorm_localQ) [right=0.5cm of T_XnormQ] {$\mathbf{X}_{\text{norm}}$\\$[B,S,D]$};
\draw[gradflow] (Xnorm_localQ) -- (T_XnormQ);
\draw[gradflow] (T_XnormQ.west) -- (dWQ_mul.east)
  node[dimlabel, midway, above]{$[B,D,S]$};

% Add AR node before dWQ
\node[arnode] (AR_WQ) [left=0.8cm of dWQ_mul] {AR};
\draw[dpcomm] ([yshift=-0.5cm]AR_WQ.south) -- ([yshift=-0.05cm]AR_WQ.south);
\draw[dpcomm] ([yshift=0.05cm]AR_WQ.north) -- ([yshift=0.5cm]AR_WQ.north);
\node[font=\tiny, red!70, align=center] at ([yshift=-0.65cm]AR_WQ.south) {DP\\nodes};
\node[font=\tiny, red!70, align=center] at ([yshift=0.65cm]AR_WQ.north) {DP\\nodes};

\node[align=center] (dWQ_out) [left=0.8cm of AR_WQ] {$\mathbf{d}\widetilde{\mathbf{W}}_{Q}$\\$[D,D]$};
\draw[dpgradweight] (dWQ_mul.west) -- (AR_WQ);
\draw[gradflow] (AR_WQ) -- (dWQ_out);

\coordinate (dK_branch) at ($(R_dK_mid)!0.52!(dKX_proj_calc)$);
\node[mulnode] (dWK_mul) at ($(dK_branch)+(0,-1.7cm)$) {$\bullet$};
\draw[flow2]  (dK_branch) -- (dWK_mul.north);

\node[auxnode] (T_XnormK) [right=1.2cm of dWK_mul] {T};
\node[dimlabel, right=0.5cm of T_XnormK] (Xnorm_localK) {$\mathbf{X}_{\text{norm}}$\\$[B,S,D]$};
\draw[gradflow] (Xnorm_localK) -- (T_XnormK);
\draw[gradflow] (T_XnormK.west) -- (dWK_mul.east)
  node[dimlabel, midway, above]{$[B,D,S]$};

% Add AR node before dWK
\node[arnode] (AR_WK) [left=0.4cm of dWK_mul] {AR};
\draw[dpcomm] ([yshift=-0.5cm]AR_WK.south) -- ([yshift=-0.05cm]AR_WK.south);
\draw[dpcomm] ([yshift=0.05cm]AR_WK.north) -- ([yshift=0.5cm]AR_WK.north);
\node[font=\tiny, red!70, align=center] at ([yshift=-0.65cm]AR_WK.south) {DP\\nodes};
\node[font=\tiny, red!70, align=center] at ([yshift=0.65cm]AR_WK.north) {DP\\nodes};

\node[align=center] (dWK_out) [left=0.4cm of AR_WK] {$\mathbf{d}\widetilde{\mathbf{W}}_{K}$\\$[D,D]$};
\draw[dpgradweight] (dWK_mul.west) -- (AR_WK);
\draw[gradflow] (AR_WK) -- (dWK_out);

\draw[gradflow] (WO_BWD) -- (T_WO);
\draw[flow2] (T_WO) -- (dOProj)
  node[dimlabel, midway, right]{$\widetilde{\mathbf{W}}_{O}^{T}$\\$[D,D]$};
\draw[gradflow] (AO_in_local_label) -- (T_AO_in);
\draw[flow2] (T_AO_in) -- (dWO_calc.east)
  node[dimlabel, midway, above]{$[B,D,S]$};

\node[auxnode] (dLN) [left=1.7cm of Sum_dXnorm] {dLN};
\draw[gradweight] (Sum_dXnorm.west) -- node[dimlabel, midway, above]
  {$\mathbf{dX}_{\text{norm}}$\\$[B,S,D]$} (dLN.east);

\node (dX_OUT) [align=center, left=1.0cm of dLN] {$\mathbf{dX}$\\$[B,S,D]$};
\draw[gradweight] (dLN.west) -- (dX_OUT);

\node[dimlabel] (LNCache) [below=0.9cm of dLN] {$\mathbf{X}$\\$[B,S,D]$};
\draw[gradflow] (LNCache.north) -- (dLN.south);

% Legend and explanation
    \node[align=left, font=\scriptsize, text width=8cm] at (-5, -9.5) {
      \textbf{AR (All-Reduce)} synchronizes weight gradients across data parallel nodes\\[4pt]
      \textbf{All-Reduce Communication Cost per MHA layer:}\\
      \quad • Total parameters: $4D^2 + D$ (W$_Q$, W$_K$, W$_V$, W$_O$, b$_O$)\\
      \quad • \textbf{Naive:} $2(N_{DP}-1) \times (4D^2 + D)$ per node\\
      \quad • \textbf{Ring:} $2\frac{N_{DP}-1}{N_{DP}} \times (4D^2 + D)$ per node\\[2pt]
      {\scriptsize (Gradients averaged across $N_{DP}$ data parallel nodes)}
    };

\end{scope}
\end{tikzpicture}
}
  \caption{Multi-head attention backward pass under data parallelism. Each
  replica computes local gradients w.r.t.\ $W_Q$, $W_K$, $W_V$, and $W_O$
  using its own mini-batch. Red dashed arrows indicate All-Reduce
  operations that aggregate these local gradients across all data-parallel
  replicas to form the global gradients used by the optimizer. The
  internal structure of the backward graph matches the single-node case.}
  \label{fig:mha_backward_dp}
\end{figure}
\end{landscape}

% ------------------------ 7.4 MLP Backward under DP -------------------
\subsection{MLP Backward under DP}

The situation for the MLP block is similar. Locally, each replica performs
exactly the same backward computations as in
Figure~\ref{fig:single_node_mlp_backward}: gradients propagate from
$\mathrm{d}\mathbf{Y}$ through the down-projection, activation,
up-projection, and (optional) layer normalization back to $\mathbf{H}$,
while accumulating parameter gradients for
$W_{\text{up}}, W_{\text{down}}, \mathbf{b}_{\text{up}},
\mathbf{b}_{\text{down}}$.

Under data parallelism, each replica $d$ obtains its own local gradients
$\nabla W_{\text{up}}^{(d)}, \nabla W_{\text{down}}^{(d)}$ and similarly
for the biases. These are then synchronized across replicas via
All-Reduce:
\[
  \nabla W_{\text{up}}
    = \frac{1}{N_D} \sum_{d=0}^{N_D-1} \nabla W_{\text{up}}^{(d)},\quad
  \nabla W_{\text{down}}
    = \frac{1}{N_D} \sum_{d=0}^{N_D-1} \nabla W_{\text{down}}^{(d)}.
\]
Bias gradients are treated analogously. Because the MLP input $\mathbf{H}$
is replicated across devices, no additional communication is needed for
$\mathrm{d}\mathbf{H}$ itself: each replica computes the same functional
backward map but with different data, and the effect of aggregation is
entirely captured by averaging the parameter gradients.

Figure~\ref{fig:mlp_backward_dp} marks the locations of these All-Reduce
operations in the MLP backward graph.

\begin{figure}[p]
  % no \centering here to avoid compilation issues
  \noindent
\resizebox{\linewidth}{!}{%
\begin{tikzpicture}[
    >=stealth,
    auxnode/.style={draw, rectangle, fill=white, minimum height=6mm, inner sep=2pt, font=\footnotesize, align=center},
    mulnode/.style={draw, circle, fill=white, minimum size=6mm, font=\footnotesize, align=center},
    addnode/.style={draw, circle, fill=white, minimum size=6mm, font=\footnotesize, align=center},
    sumnode/.style={draw, circle, fill=white, minimum size=6mm, font=\tiny, align=center},
    arnode/.style={draw, rectangle, fill=red!30, minimum height=6mm, inner sep=2pt, font=\footnotesize, align=center, thick, draw=red!70},
    flow_rev/.style={<-, thick, black!85},
    flow_dw/.style={->, thick, black!85},
    flow_act/.style={double, ->, thick, black!85},
    dimlabel/.style={font=\tiny, inner sep=1pt, align=center},
    gradlabel/.style={font=\tiny\bfseries, inner sep=1pt, align=center},
    dpgradweight/.style={->, thick, black!85},
    dpcomm/.style={<->, thick, red!70, dashed}
]
    \node[font=\Large\bfseries] at (5, 10) {MLP Backward Pass (Data Parallel)};

    \pgfmathsetmacro{\backwardoffset}{0.0}

    \node (d_MOut) at (12.6, \backwardoffset) {$\mathrm{d}\mathbf{Y}$};
    \node[auxnode] (d_Drop2) [left=1.8cm of d_MOut] {dDO};
    \draw[flow_rev] (d_Drop2) -- (d_MOut)
      node[dimlabel, midway, below]{\shortstack{$\mathrm{d}\mathbf{Y}$\\$[B,S,D]$}};

    \coordinate (split2) at ($(d_Drop2.west) + (-1.5cm, 0)$);
    \coordinate (branch_dUproj) at ($(split2) + (-1.2cm, 0)$);

    \node[sumnode] (d_SumB2) [above=0.8cm of split2] {$\sum_{B, S}$};
    \node (d_Bdown) [above=0.8cm of d_SumB2] {$\mathrm{d}\tilde{\mathbf{b}}_{\text{down}}$};
    \draw[dpgradweight] (d_SumB2) -- (d_Bdown) node[dimlabel, midway, right]{$[D]$};

    \draw[flow_rev] (d_SumB2) -- (split2);

    \node[mulnode] (d_L2Mul_in) [left=2.2cm of split2] {$\bullet$};
    \draw[flow_rev] (d_L2Mul_in) -- (d_Drop2)
      node[gradlabel, midway, below]{\shortstack{$\mathrm{d}\mathbf{Z}_{\text{down}}=\mathrm{d}\mathbf{A}_{\text{out}}$\\$[B,S,D]$}};

    \node (W_down_T) [below=1.0cm of d_L2Mul_in] {$\tilde{\mathbf{W}}_{\text{down}}^{T}$};
    \draw[flow_act] (W_down_T.north) -- (d_L2Mul_in)
      node[dimlabel, midway, right]{$[D, D_{ff}]$};

    \coordinate (L2Mul_w_y) at ($(d_L2Mul_in) + (0, 3.5cm)$);
    \node[mulnode] (d_L2Mul_w) at (L2Mul_w_y) {$\bullet$};

    % Add AR node before d_Wdown
    \node[arnode] (AR_down) [above=0.8cm of d_L2Mul_w] {AR};

    % Add communication arrows for AR_down
    \draw[dpcomm] ([xshift=-1.2cm]AR_down.west) -- ([xshift=-0.1cm]AR_down.west);
    \draw[dpcomm] ([xshift=0.1cm]AR_down.east) -- ([xshift=1.2cm]AR_down.east);
    \node[font=\tiny, red!70, align=center] at ([xshift=-1.5cm]AR_down.west) {DP\\nodes};
    \node[font=\tiny, red!70, align=center] at ([xshift=1.5cm]AR_down.east) {DP\\nodes};
    \node[font=\tiny, red!70, below=0.1cm of AR_down] {$2DD_{ff}$};

    \node (d_Wdown) [above=0.8cm of AR_down] {$\mathrm{d}\tilde{\mathbf{W}}_{\text{down}}$};
    \draw[dpgradweight] (d_L2Mul_w) -- (AR_down) node[dimlabel, midway, right]{$[D_{ff}, D]$};
    \draw[flow_dw] (AR_down) -- (d_Wdown) node[dimlabel, midway, right]{$[D_{ff}, D]$};

    \draw[flow_act] (branch_dUproj.north) |- (d_L2Mul_w.east);

    \node[auxnode] (Uin_T) at ($(d_L2Mul_w.west) + (-1.5cm, 0)$) {T};
    \draw[flow_dw] (Uin_T) -- (d_L2Mul_w)
      node[dimlabel, midway, below]{\shortstack{$\mathbf{U}_{\text{in}}^T$\\$[B, D_{ff}, S]$}};
    \node (Uin_aux) [left=1.8cm of Uin_T] {$\mathbf{U}_{\text{in}}$};
    \draw[flow_dw] (Uin_aux) -- (Uin_T) node[dimlabel, midway, above]{\shortstack{$[B,S,D_{ff}]$}};

    \node[auxnode] (d_Drop1) [left=1.8cm of d_L2Mul_in] {dDO};
    \draw[flow_rev] (d_Drop1) -- (d_L2Mul_in)
      node[dimlabel, midway, below]{\shortstack{$\mathrm{d}\mathbf{U}_{\text{in}}$\\$[B,S,D_{ff}]$}};

    \node[auxnode] (d_Act) [left=1.8cm of d_Drop1] {dGL};
    \draw[flow_rev] (d_Act) -- (d_Drop1)
      node[dimlabel, midway, below]{\shortstack{$\mathrm{d}\mathbf{H}_{\text{inter}}$\\$[B,S,D_{ff}]$}};

    \coordinate (split1) at ($(d_Act.west) + (-1.5cm, 0)$);
    \coordinate (branch_dHpre) at ($(split1) + (-1.2cm, 0)$);

    \node[sumnode] (d_SumB1) [above=0.8cm of split1] {$\sum_{B, S}$};
    \node (d_Bup) [above=0.8cm of d_SumB1] {$\mathrm{d}\tilde{\mathbf{b}}_{\text{up}}$};
    \draw[dpgradweight] (d_SumB1) -- (d_Bup) node[dimlabel, midway, right]{$[D_{ff}]$};

    \draw[flow_rev] (d_SumB1) -- (split1);

    \node[mulnode] (d_L1Mul_in) [left=2.2cm of split1] {$\bullet$};
    \draw[flow_rev] (d_L1Mul_in) -- (d_Act)
      node[gradlabel, midway, below]{\shortstack{$\mathrm{d}\mathbf{Z}_{\text{up}}=\mathrm{d}\mathbf{A}_{\text{up}}$\\$[B,S,D_{ff}]$}};

    \node (W_up_T) [below=1.0cm of d_L1Mul_in] {$\tilde{\mathbf{W}}_{\text{up}}^{T}$};
    \draw[flow_act] (W_up_T.north) -- (d_L1Mul_in)
      node[dimlabel, midway, right]{$[D_{ff}, D]$};

    \coordinate (L1Mul_w_y) at ($(d_L1Mul_in) + (0, 3.5cm)$);
    \node[mulnode] (d_L1Mul_w) at (L1Mul_w_y) {$\bullet$};

    % Add AR node before d_Wup
    \node[arnode] (AR_up) [above=0.8cm of d_L1Mul_w] {AR};

    % Add communication arrows for AR_up
    \draw[dpcomm] ([xshift=-1.2cm]AR_up.west) -- ([xshift=-0.1cm]AR_up.west);
    \draw[dpcomm] ([xshift=0.1cm]AR_up.east) -- ([xshift=1.2cm]AR_up.east);
    \node[font=\tiny, red!70, align=center] at ([xshift=-1.5cm]AR_up.west) {DP\\nodes};
    \node[font=\tiny, red!70, align=center] at ([xshift=1.5cm]AR_up.east) {DP\\nodes};
    \node[font=\tiny, red!70, below=0.1cm of AR_up] {$2DD_{ff}$};

    \node (d_Wup) [above=0.8cm of AR_up] {$\mathrm{d}\tilde{\mathbf{W}}_{\text{up}}$};
    \draw[dpgradweight] (d_L1Mul_w) -- (AR_up) node[dimlabel, midway, right]{$[D, D_{ff}]$};
    \draw[flow_dw] (AR_up) -- (d_Wup) node[dimlabel, midway, right]{$[D, D_{ff}]$};

    \draw[flow_act] (branch_dHpre.north) |- (d_L1Mul_w.east);

    \node[auxnode] (Znorm_T) at ($(d_L1Mul_w.west) + (-1.5cm, 0)$) {T};
    \draw[flow_dw] (Znorm_T) -- (d_L1Mul_w)
      node[dimlabel, midway, below]{\shortstack{$\mathbf{H}^T$\\$[B, D, S]$}};
    \node (Znorm_aux) [left=1.8cm of Znorm_T] {$\mathbf{H}$};
    \draw[flow_dw] (Znorm_aux) -- (Znorm_T) node[dimlabel, midway, above]{\shortstack{$[B,S,D]$}};

    \node[auxnode] (d_LN2) [left=1.8cm of d_L1Mul_in] {dLN};
    \draw[flow_rev] (d_LN2) -- (d_L1Mul_in)
      node[dimlabel, midway, below]{\shortstack{$\mathrm{d}\mathbf{H}$\\$[B,S,D]$}};

    \node (d_MIn) [left=1.8cm of d_LN2] {$\mathrm{d}\mathbf{X}$};
    \draw[flow_rev] (d_MIn) -- (d_LN2)
      node[dimlabel, midway, below]{\shortstack{$\mathrm{d}\mathbf{X}$\\$[B,S,D]$}};

    % Legend and explanation
    \node[align=left, font=\scriptsize, text width=8cm] at (-2, -3.5) {
      \textbf{AR (All-Reduce)} synchronizes weight gradients across data parallel nodes\\[4pt]
      \textbf{MLP All-Reduce Cost:} $\sim 2DD_{ff}$ parameters (W$_{\text{up}}$, W$_{\text{down}}$)\\
      \quad • \textbf{Naive:} $2(N_{DP}-1) \times 2DD_{ff}$ per node\\
      \quad • \textbf{Ring:} $2\frac{N_{DP}-1}{N_{DP}} \times 2DD_{ff}$ per node\\[2pt]
      {\scriptsize (Gradients averaged across $N_{DP}$ data parallel nodes)}
    };

\end{tikzpicture}%
}
  \caption{MLP backward pass under data parallelism. Each replica computes
  local gradients for the up- and down-projection weights and biases using
  its own mini-batch. Red boxes and dashed arrows indicate All-Reduce
  operations that aggregate these local parameter gradients across
  replicas. The structure of the backward computation inside each replica
  is identical to the single-node graph.}
  \label{fig:mlp_backward_dp}
\end{figure}


% ------------------------ 7.5 Communication and Memory ---------------
\subsection{Communication and Memory Considerations}

Compared to the single-node model in Section~\ref{sec:sn}:

\begin{itemize}
  \item \textbf{Memory per device.}
        Each device stores a full copy of the model parameters and
        optimizer states. Activation memory per device is reduced by a
        factor of $N_D$ because each replica sees only a fraction of the
        global batch.
  \item \textbf{Communication pattern.}
        DP introduces communication only at gradient synchronization time
        (All-Reduce over parameter gradients). There is no communication
        on the forward path and no communication for activations.
  \item \textbf{Scalability.}
        Increasing $N_D$ increases the effective batch size and reduces
        the per-device compute and activation memory, but the cost of
        All-Reduce grows with the number of replicas and the total
        parameter size.
  \item \textbf{Combination with other parallelism.}
        In practice, DP is often combined with tensor parallelism (and
        sometimes pipeline parallelism). In such settings, each data
        parallel group contains several tensor-parallel shards; gradient
        synchronization is then performed across data-parallel groups,
        while tensor-parallel collectives remain local to each group.
\end{itemize}

From the perspective of the computation graphs in Section~\ref{sec:sn}, data
parallelism is therefore the least invasive form of parallelism: it keeps
the per-layer forward and backward structure unchanged and adds only
gradient All-Reduces on top.
