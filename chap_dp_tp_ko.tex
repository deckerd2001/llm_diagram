% ==========================================================
% 8. 데이터 병렬화와 텐서 병렬화의 결합
% ==========================================================
\section{데이터 병렬화와 텐서 병렬화의 결합}
\label{sec:dp_tp}

앞 장들에서는 텐서 병렬화(TP, Section~\ref{sec:tp})와
데이터 병렬화(DP, Section~\ref{sec:dp})를 각각 따로 살펴보았다.
실제 대규모 학습 환경에서는 거의 항상 이 둘을 함께 사용한다.

\begin{itemize}
  \item \textbf{텐서 병렬화}는 큰 가중치 행렬과 활성값을
        \emph{모델 샤드(model shard)} 내부의 $N_T$개 디바이스에
        나누어 저장·계산한다.
  \item \textbf{데이터 병렬화}는 그 모델 샤드 전체를 $N_D$개 복제하여
        서로 다른 미니배치를 처리하고, 기울기를 동기화한다.
\end{itemize}

이 장의 목표는, 앞에서 정의한 단일 노드 및 TP/DP-only 계산 그래프 위에
DP+TP를 어떻게 “겹쳐서” 올려놓는지 개념적으로 정리하는 것이다.
즉, \emph{모델 차원 병렬화(TP)}와 \emph{배치 차원 병렬화(DP)}가
어떻게 직교하는지, 그리고 어디에서 집합 통신이 발생하는지를
시각적으로 이해할 수 있도록 하는 데 초점을 맞춘다.

% ------------------------ 8.1 Parallel Groups -------------------------
\subsection{병렬 그룹과 레이아웃}

우리는 전체 디바이스들을 개념적으로 2차원 격자 위에 배치한다.

\begin{itemize}
  \item \textbf{TP 축}: 각 행(row) 안의 $N_T$개 디바이스는
        하나의 텐서 병렬 그룹을 이루어 하나의 모델 샤드를 공동으로 저장한다.
  \item \textbf{DP 축}: 열(column) 방향으로는, 그 TP 그룹의
        $N_D$개 복제본이 서로 다른 배치 조각을 처리한다.
\end{itemize}

다르게 말하면 다음과 같다.

- 각 TP 그룹은 Section~\ref{sec:tp}의 “TP-only” 모델처럼 동작하지만,
  전체 배치가 아니라 \emph{로컬 배치 조각}만을 본다.
- 각 DP “열”은 Section~\ref{sec:dp}의 “DP-only” 설정과 비슷하지만,
  한 디바이스 대신 \emph{TP 그룹 전체}가 하나의 복제본 역할을 한다.

전역 배치 크기가 $B$라면, 이를 데이터 병렬 그룹에 대해
다음과 같이 나눈다.
\[
  B = N_D \cdot B_{\text{local}},
\]
따라서 DP 복제본 $d$는
$\mathbf{X}_d \in \mathbb{R}^{B_{\text{local}} \times S \times D}$를
입력으로 본다.
각 복제본 내부에서는 $\mathbf{X}_d$가
Section~\ref{sec:tp}에서와 동일한 TP-샤딩 모델로 들어간다:
단일 노드의 큰 matmul들은 $N_T$ 샤드에 걸친
컬럼/로우 병렬 쌍으로 대체된다.

Figure~\ref{fig:dp_tp_overall_flow}는 이러한 레이아웃을 요약한 그림이다.
이 그림은 단일 노드 개요(Figure~\ref{fig:single_node_overall})와
TP 개요(Figure~\ref{fig:tp_overall_flow})를
DP+TP 버전으로 확장한 것으로 읽으면 된다.

\begin{figure}[htbp]
  \centering
  \resizebox{\linewidth}{!}{%
\begin{tikzpicture}[
    node distance=2.5cm,
    >=stealth,
    block/.style={rectangle, draw=black, fill=white, text width=5em, text centered, rounded corners, minimum height=8em, font=\bfseries},
    blockstack/.style={rectangle, draw=black!60, fill=white, text width=5em, text centered, rounded corners, minimum height=8em},
    forward/.style={-{Stealth[length=2mm]}, thick, black},
    backward/.style={-{Stealth[length=2mm]}, thick, black, densely dashed},
    tpcomm/.style={{Stealth[length=1.5mm]}-{Stealth[length=1.5mm]}, thick, blue!70, densely dotted},
    dpcomm/.style={<->, thick, red!70, dashed},
    io/.style={text centered, font=\bfseries}
]
    % Title
    \node[font=\Large\bfseries] at (10, 14) {Transformer Flow with TP + DP (2 DP replicas × 3 TP GPUs)};

    % ========== DP Replica 0 (Upper) ==========
    \node[font=\normalsize\bfseries, anchor=west] at (-1, 11.5) {DP Rank 0};

    \node (input_0) [io] at (0, 10) {$\mathbf{X}_0$};
    \node (encoding_0) [block, right of=input_0, yshift=-3em] {Input\\Encoding};

    % MHA blocks (3 stacked) - TP Group 0
    \node (mha3_0) [blockstack, right of=encoding_0, xshift=0.3cm, yshift=0.3cm] {};
    \node (mha2_0) [blockstack, right of=encoding_0, xshift=0.15cm, yshift=0.15cm] {};
    \node (mha_0) [block, right of=encoding_0] {MHA};
    \node[font=\tiny, anchor=north east] at (mha_0.north east) {GPU 0};

    % FFN blocks (3 stacked) - TP Group 0
    \node (mlp3_0) [blockstack, right of=mha_0, xshift=0.3cm, yshift=0.3cm] {};
    \node (mlp2_0) [blockstack, right of=mha_0, xshift=0.15cm, yshift=0.15cm] {};
    \node (mlp_0) [block, right of=mha_0] {FFN};
    \node[font=\tiny, anchor=north east] at (mlp_0.north east) {GPU 0};

    \node (output_0) [block, right of=mlp_0] {Output\\Projection};
    \node (pred_0) [io, right of=output_0, yshift=3em] {$\mathbf{Y}_0$};
    \node (loss_0) [align=center, io, right of=output_0] {\small LOSS:\\$\mathcal{L}_0$};
    \node (gradient_0) [io, right of=output_0, yshift=-3em] {$\mathbf{dY}_0$};

    % Forward arrows - DP Rank 0
    \draw [forward] (input_0) -- ([yshift=3em]encoding_0.west);
    \draw [forward] ([yshift=3em]encoding_0.east) -- ([yshift=3em]mha_0.west);
    \draw [forward] ([yshift=3em]mha_0.east) -- ([yshift=3em]mlp_0.west);
    \draw [forward] ([yshift=3em]mlp_0.east) -- ([yshift=3em]output_0.west);
    \draw [forward] ([yshift=3em]output_0.east) -- (pred_0);
    \draw [forward] (pred_0) -- (loss_0);
    \draw [backward] (loss_0) -- (gradient_0);

    % Backward arrows - DP Rank 0
    \draw [backward] (gradient_0) -- ([yshift=-3em]output_0.east);
    \draw [backward] ([yshift=-3em]output_0.west) -- ([yshift=-3em]mlp_0.east);
    \draw [backward] ([yshift=-3em]mlp_0.west) -- ([yshift=-3em]mha_0.east);
    \draw [backward] ([yshift=-3em]mha_0.west) -- ([yshift=-3em]encoding_0.east);

    % TP Communications - DP Rank 0
    % Forward All-Reduce
    \draw [tpcomm] (mha_0.north east) -- ([xshift=0.3cm, yshift=0.3cm]mha_0.north east);
    \node[font=\tiny, blue!70, anchor=south] at ([xshift=0.15cm, yshift=0.35cm]mha_0.north east) {TP all\_reduce(X)};

    \draw [tpcomm] (mlp_0.north east) -- ([xshift=0.3cm, yshift=0.3cm]mlp_0.north east);
    \node[font=\tiny, blue!70, anchor=south] at ([xshift=0.15cm, yshift=0.35cm]mlp_0.north east) {TP all\_reduce(X)};

    % Backward All-Reduce - dX
    \draw [tpcomm] ([yshift=-0.2cm]mlp_0.south west) -- ([xshift=0.3cm, yshift=0.1cm]mlp_0.south west);
    \node[font=\tiny, blue!70, anchor=north] at ([xshift=0.15cm, yshift=-0.25cm]mlp_0.south west) {TP all\_reduce(dX)};

    \draw [tpcomm] ([yshift=-0.2cm]mha_0.south west) -- ([xshift=0.3cm, yshift=0.1cm]mha_0.south west);
    \node[font=\tiny, blue!70, anchor=north] at ([xshift=0.15cm, yshift=-0.25cm]mha_0.south west) {TP all\_reduce(dX)};

    % Backward All-Gather for gradients - positioned on right side, before DP all-reduce
    \draw [dpcomm] ([xshift=-0.5cm, yshift=-1.2cm]mlp_0.south east) -- ([xshift=-0.2cm, yshift=-0.9cm]mlp_0.south east);
    \node[font=\tiny, red!70, anchor=west, align=left] at ([xshift=-0.15cm, yshift=-1.05cm]mlp_0.south east) {(1) TP all\_gather\\$(\nabla W_{\text{up},\text{down}})$};

    \draw [dpcomm] ([xshift=-0.5cm, yshift=-1.2cm]mha_0.south east) -- ([xshift=-0.2cm, yshift=-0.9cm]mha_0.south east);
    \node[font=\tiny, red!70, anchor=west, align=left] at ([xshift=-0.15cm, yshift=-1.05cm]mha_0.south east) {(1) TP all\_gather\\$(\nabla W_{Q,K,V,O})$};

    % Brace for layer repetition - DP Rank 0
    \draw[decorate, decoration={brace, amplitude=10pt}]
        ([yshift=5.0em]mha_0.north west) -- ([xshift=0.3cm, yshift=5.0em]mlp_0.north east)
        node[midway, above=12pt, font=\normalsize] {$N$ layers};

    % ========== DP Replica 1 (Lower) ==========
    \node[font=\normalsize\bfseries, anchor=west] at (-1, 5.5) {DP Rank 1};

    \node (input_1) [io] at (0, 4) {$\mathbf{X}_1$};
    \node (encoding_1) [block, right of=input_1, yshift=-3em] {Input\\Encoding};

    % MHA blocks (3 stacked) - TP Group 1
    \node (mha3_1) [blockstack, right of=encoding_1, xshift=0.3cm, yshift=0.3cm] {};
    \node (mha2_1) [blockstack, right of=encoding_1, xshift=0.15cm, yshift=0.15cm] {};
    \node (mha_1) [block, right of=encoding_1] {MHA};
    \node[font=\tiny, anchor=north east] at (mha_1.north east) {GPU 3};

    % FFN blocks (3 stacked) - TP Group 1
    \node (mlp3_1) [blockstack, right of=mha_1, xshift=0.3cm, yshift=0.3cm] {};
    \node (mlp2_1) [blockstack, right of=mha_1, xshift=0.15cm, yshift=0.15cm] {};
    \node (mlp_1) [block, right of=mha_1] {FFN};
    \node[font=\tiny, anchor=north east] at (mlp_1.north east) {GPU 3};

    \node (output_1) [block, right of=mlp_1] {Output\\Projection};
    \node (pred_1) [io, right of=output_1, yshift=3em] {$\mathbf{Y}_1$};
    \node (loss_1) [align=center, io, right of=output_1] {\small LOSS:\\$\mathcal{L}_1$};
    \node (gradient_1) [io, right of=output_1, yshift=-3em] {$\mathbf{dY}_1$};

    % Forward arrows - DP Rank 1
    \draw [forward] (input_1) -- ([yshift=3em]encoding_1.west);
    \draw [forward] ([yshift=3em]encoding_1.east) -- ([yshift=3em]mha_1.west);
    \draw [forward] ([yshift=3em]mha_1.east) -- ([yshift=3em]mlp_1.west);
    \draw [forward] ([yshift=3em]mlp_1.east) -- ([yshift=3em]output_1.west);
    \draw [forward] ([yshift=3em]output_1.east) -- (pred_1);
    \draw [forward] (pred_1) -- (loss_1);
    \draw [backward] (loss_1) -- (gradient_1);

    % Backward arrows - DP Rank 1
    \draw [backward] (gradient_1) -- ([yshift=-3em]output_1.east);
    \draw [backward] ([yshift=-3em]output_1.west) -- ([yshift=-3em]mlp_1.east);
    \draw [backward] ([yshift=-3em]mlp_1.west) -- ([yshift=-3em]mha_1.east);
    \draw [backward] ([yshift=-3em]mha_1.west) -- ([yshift=-3em]encoding_1.east);

    % TP Communications - DP Rank 1
    % Forward All-Reduce
    \draw [tpcomm] (mha_1.north east) -- ([xshift=0.3cm, yshift=0.3cm]mha_1.north east);
    \node[font=\tiny, blue!70, anchor=south] at ([xshift=0.15cm, yshift=0.35cm]mha_1.north east) {TP all\_reduce(X)};

    \draw [tpcomm] (mlp_1.north east) -- ([xshift=0.3cm, yshift=0.3cm]mlp_1.north east);
    \node[font=\tiny, blue!70, anchor=south] at ([xshift=0.15cm, yshift=0.35cm]mlp_1.north east) {TP all\_reduce(X)};

    % Backward All-Reduce - dX
    \draw [tpcomm] ([yshift=-0.2cm]mlp_1.south west) -- ([xshift=0.3cm, yshift=0.1cm]mlp_1.south west);
    \node[font=\tiny, blue!70, anchor=north] at ([xshift=0.15cm, yshift=-0.25cm]mlp_1.south west) {TP all\_reduce(dX)};

    \draw [tpcomm] ([yshift=-0.2cm]mha_1.south west) -- ([xshift=0.3cm, yshift=0.1cm]mha_1.south west);
    \node[font=\tiny, blue!70, anchor=north] at ([xshift=0.15cm, yshift=-0.25cm]mha_1.south west) {TP all\_reduce(dX)};

    % Backward All-Gather for gradients - positioned on right side, before DP all-reduce
    % \draw [dpcomm] ([xshift=0.5cm, yshift=-1.2cm]mlp_1.south east) -- ([xshift=0.8cm, yshift=-0.9cm]mlp_1.south east);
    % \node[font=\tiny, red!70, anchor=west, align=left] at ([xshift=0.85cm, yshift=-1.05cm]mlp_1.south east) {(1) TP all\_gather\\$(\nabla W_{\text{up},\text{down}})$};

    % \draw [dpcomm] ([xshift=0.5cm, yshift=-1.2cm]mha_1.south east) -- ([xshift=0.8cm, yshift=-0.9cm]mha_1.south east);
    % \node[font=\tiny, red!70, anchor=west, align=left] at ([xshift=0.85cm, yshift=-1.05cm]mha_1.south east) {(1) TP all\_gather\\$(\nabla W_{Q,K,V,O})$};

    % ========== DP Communications (Between TP Groups) ==========
    \draw [dpcomm] (mha_0.south) -- (mha_1.north) node[midway, right, font=\tiny, align=left, red!70] {(2) DP All-Reduce\\$(\nabla W_{Q,K,V,O})$};
    \draw [dpcomm] (mlp_0.south) -- (mlp_1.north) node[midway, right, font=\tiny, align=left, red!70] {(2) DP All-Reduce\\$(\nabla W_{\text{up},\text{down}})$};

    % Communication Order Annotation
    \node[font=\scriptsize, align=left, text width=7.5cm, draw=black!30, fill=yellow!10, rounded corners] at ([xshift=15.5cm, yshift=10em]encoding_1.south) {
        \textbf{Communication in TP+DP:}\\
        \textbf{Forward:} \textcolor{blue!70}{TP all\_reduce(X)} - sync activations\\
        \textbf{Backward:}\\
        \hspace{0.3em}\textcolor{blue!70}{• TP all\_reduce(dX)} - sync activation grads\\
        \hspace{0.3em}\textcolor{red!70}{• (1) TP all\_gather($\nabla W$)} - collect weight grads\\
        \hspace{0.6em}within each TP group\\
        \hspace{0.3em}\textcolor{red!70}{• (2) DP all-reduce($\nabla W$)} - average grads\\
        \hspace{0.6em}across DP groups\\[0.2em]
        \textit{Note: (1) must complete before (2)}
    };

    % Labels (Legend)
    \coordinate (legend) at ([xshift=11.5cm, yshift=12em]input_0);

    % Forward
    \draw[forward, -{Stealth[length=1.2mm,width=1.4mm]}, line width=0.3pt] (legend) -- ++(0.8,0)
      node[right, font=\scriptsize] {Forward};

    % Backward
    \draw[backward, -{Stealth[length=1.2mm,width=1.4mm]}, line width=0.3pt] ([yshift=-0.4cm]legend) -- ++(0.8,0)
      node[right, font=\scriptsize] {Backward};

    % TP Comm
    \draw[tpcomm, line width=0.3pt] ([yshift=-0.8cm]legend) -- ++(0.8,0)
      node[right, font=\scriptsize] {TP Comm};

    % DP Comm
    \draw[dpcomm, line width=0.3pt] ([yshift=-1.2cm]legend) -- ++(0.8,0)
      node[right, font=\scriptsize] {DP Comm};

\end{tikzpicture}%
}
  \caption{데이터 병렬화와 텐서 병렬화가 결합된 전체 트랜스포머 레이어.
  가로 방향으로는 각 텐서 병렬 그룹($N_T$ 디바이스)이
  하나의 샤딩된 레이어(Section~\ref{sec:tp})를 함께 구현한다.
  세로 방향으로는 그러한 그룹 $N_D$개가 데이터 병렬 격자를 이룬다.
  각 행은 서로 다른 미니배치 조각을 처리하며,
  기울기는 행들 사이에서 동기화된다.
  All-Reduce와 All-Gather 집합 통신이 사용되는 위치가
  그림에 함께 표시된다.}
  \label{fig:dp_tp_overall_flow}
\end{figure}

% ------------------------ 8.2 From Single-Node to TP to DP+TP --------
\subsection{단일 노드에서 텐서 병렬화, 그리고 데이터+텐서 병렬화까지}

DP+TP를 이해하는 한 가지 좋은 방법은,
Section~\ref{sec:sn}의 단일 노드 계산을 기준으로
\emph{점진적인 수정 단계}로 보는 것이다.

\paragraph{단일 노드 (Section~\ref{sec:sn}).}
모든 파라미터와 활성값이 하나의 디바이스에 있다.
각 선형 레이어는 하나의 matmul로 구현되며,
집합 통신은 전혀 필요 없다.

\paragraph{텐서 병렬화만 적용 (Section~\ref{sec:tp}).}
큰 행렬을 $N_T$ 디바이스에 나눈다.

\begin{itemize}
  \item \textbf{컬럼 병렬(linear)}:
        가중치를 출력 차원 방향으로 나눈다.
        각 샤드는 출력의 일부 슬라이스만 계산하고,
        이후 연산은 이 샤딩된 표현을 그대로 사용하거나,
        All-Gather를 통해 전체 텐서를 다시 조립한다.
  \item \textbf{로우 병렬(linear)}:
        가중치와 입력을 입력 차원 방향으로 나눈다.
        각 샤드는 부분 결과를 계산하고,
        All-Reduce로 부분 출력들을 합산한다.
\end{itemize}

하나의 TP 그룹 내부에서, 모든 디바이스는 같은 배치를 보며,
활성값은 피처 차원 방향으로 샤딩된다.
All-Reduce와 (필요하다면) All-Gather는
\emph{레이어 내부} 계산 그래프에 등장한다.

\paragraph{데이터 병렬화만 적용 (Section~\ref{sec:dp}).}
단일 노드 그래프는 그대로 두고, 이를 $N_D$ 디바이스에 복제한다.
각 디바이스는 서로 다른 미니배치 조각을 처리한다.
집합 통신은 역전파 끝에서 파라미터 기울기에 대해 수행하는
All-Reduce 한 번뿐이다.

\paragraph{데이터 병렬화와 텐서 병렬화를 결합.}
DP+TP에서는 위 두 가지 수정을 동시에 적용한다.

\begin{itemize}
  \item 각 데이터 병렬 복제본은 $N_T$ 디바이스로 구성된
        하나의 TP 그룹이다. 이 그룹 내부에서는
        Section~\ref{sec:tp}의 모든 TP 샤딩과 집합 통신이 그대로 적용된다.
  \item 데이터 병렬 복제본들 사이에서는
        Section~\ref{sec:dp}에서와 같이 기울기 All-Reduce를 수행하되,
        이제는 각 가중치 행렬의 \emph{TP 샤드}를 단위로 한다.
\end{itemize}

즉, 각 MHA/MLP/출력 블록의 \emph{샤드 단위} 내부 구조는
단일 TP-only 설정 때와 동일하고,
변하는 것은 “각 텐서에 몇 개의 디바이스가 관여하는지”와
“어떤 집합 통신이 어떤 디바이스를 잇는지”뿐이다.

% ------------------------ 8.3 Forward Pass under DP+TP ---------------
\subsection{데이터+텐서 병렬 환경에서의 순전파}

DP+TP 환경에서 한 TP 샤드가 보는 관점에서,
순전파는 기본적으로 TP-only 모델
(Section~\ref{sec:tp})과 같다.

\begin{itemize}
  \item 입력은 로컬 미니배치 조각
        $\mathbf{X}_d \in \mathbb{R}^{B_{\text{local}} \times S \times D}$이다.
  \item 컬럼 병렬 및 로우 병렬 선형 레이어가
        TP-only에서와 동일한 방식으로 적용된다.
  \item 중간 활성값(예: 헤드별 Q/K/V, MLP 중간 피처)은
        $N_T$ 디바이스에 샤딩된다.
\end{itemize}

여기에 \emph{추가}되는 것은, 어떤 이후 연산이
“샤딩되지 않은 전체 텐서”를 필요로 할 때
\textbf{활성값에 대한 All-Gather}를 사용하는 것이다.

대표적인 예시는 다음과 같다.

\begin{itemize}
  \item \textbf{어휘 병렬 출력 프로젝션.}
        최종 LM head 가중치 $\mathbf{W}_{\text{lm}}$이
        어휘 차원 방향으로 TP 디바이스에 샤딩되어 있다면,
        각 샤드는 어휘의 일부분에 대한 logits만 생성한다.
        전체 어휘에 대한 softmax/손실을 계산하려면,
        모든 샤드에서 나온 logits를 All-Gather로 모아
        크기가 $[B_{\text{local}}, S, V]$인 전체 텐서를
        구성해야 한다.
  \item \textbf{샤딩된 임베딩.}
        토큰 임베딩 행렬이 어휘 차원 방향으로 샤딩되어 있다면,
        임베딩 lookup 결과 역시 샤딩된 형태로 나온다.
        이후 레이어가 샤딩되지 않은
        $[B_{\text{local}}, S, D]$ 표현을 기대한다면,
        그 전에 All-Gather를 통해 임베딩을 합쳐줘야 한다.
  \item \textbf{샤딩되지 않은 연산들.}
        은닉 전체 차원에 작용하는 연산
        (예: 전역 정규화, 샤딩되지 않은 잔차 브랜치 등)은
        원칙적으로 전체 텐서를 필요로 하므로,
        그 전에 샤딩된 활성값을 All-Gather해야 할 수 있다.
        물론 모델을 설계할 때, 이러한 연산들이
        샤드 로컬로 동작하거나 TP 친화적인 변형으로 대체되도록
        구조를 잡으면, All-Gather를 줄일 수 있다.
\end{itemize}

정리하면, DP+TP 순전파에서 TP 그룹 내부의 All-Reduce/All-Gather 위치와
텐서 모양은 TP-only와 거의 같고,
데이터 병렬은 “어떤 미니배치 조각을 보느냐”만 바꾼다.
All-Gather 위치는 “샤딩된 텐서를 전역적으로 봐야 하는 연산”에 의해
결정된다.

% ------------------------ 8.4 Backward Pass and Gradient Sync --------
\subsection{데이터+텐서 병렬 환경에서의 역전파와 기울기 동기화}

DP+TP 환경에서 역전파는 다음 두 가지 요소를 결합한 것이다.

\begin{itemize}
  \item Section~\ref{sec:tp}에서의 TP-only 역전파 그래프
        (레이어 내부의 활성값/부분 기울기에 대한 All-Reduce와 All-Gather).
  \item Section~\ref{sec:dp}에서의 DP-only 기울기 동기화
        (이제는 각 TP 샤드 파라미터에 대해 적용).
\end{itemize}

특정 파라미터 텐서 $W$에 대해 보다 구체적으로 보면 다음과 같다.

\begin{itemize}
  \item \textbf{TP-only에서:}
    \begin{itemize}
      \item $W$는 TP 그룹 내부의 $N_T$ 디바이스에 대해
            $W^{(t)}$ 형태의 샤드로 나뉘어 저장된다
            ($t = 0,\dots,N_T-1$).
      \item 로컬 matmul 역전파를 통해 각 샤드에서
            로컬 기울기 $\nabla W^{(t)}$가 계산된다.
      \item $W$가 로우 병렬 구성에 등장한다면,
            입력에 대한 부분 기울기는 $t$에 대한 All-Reduce로 합쳐진다.
    \end{itemize}
  \item \textbf{DP+TP에서:}
    \begin{itemize}
      \item 위의 TP-only 로직이 각 데이터 병렬 복제본 $d$에 대해
            독립적으로 적용되어, 복제본 $d$의 샤드 $t$에 대해
            $\nabla W^{(t,d)}$를 얻는다.
      \item 그 다음, 각 샤드 $t$에 대해 복제본 인덱스 $d$에 대한
            All-Reduce를 수행한다:
            \[
              \nabla W^{(t)}
                = \frac{1}{N_D}
                  \sum_{d=0}^{N_D-1} \nabla W^{(t,d)}.
            \]
            이는 데이터 병렬 기울기 동기화이지만,
            이제 모든 TP 샤드에 대해 \emph{병렬로} 수행된다고 볼 수 있다.
    \end{itemize}
\end{itemize}

따라서 각 파라미터 샤드 $W^{(t)}$는 두 종류의 집합 통신에 참여한다.

\begin{itemize}
  \item \textbf{TP 그룹 내부 통신}:
        $t$에 대한 All-Reduce/All-Gather
        (로우/컬럼 병렬 로직에 따라).
  \item \textbf{DP 복제본 간 통신}:
        각 샤드 인덱스 $t$에 대해 $d$에 대한 All-Reduce
        (데이터 병렬 기울기 동기화).
\end{itemize}

이 조합 덕분에, 역전파가 끝났을 때
모든 DP 복제본의 모든 TP 그룹이
자신의 로컬 샤드 $W^{(t)}$에 대해
동일하고 올바르게 합산된 기울기를 가지게 되며,
전체 시스템에서 옵티마이저 업데이트가 일관되게 수행된다.

% ------------------------ 8.5 Communication Summary ------------------
\subsection{통신 요약 및 TP/DP 단독 설정과의 비교}

마지막으로, TP-only (Section~\ref{sec:tp}) 및 DP-only
(Section~\ref{sec:dp})와 비교했을 때
DP+TP의 주요 차이를 정리해 보자.

\paragraph{TP-only와 비교.}

\begin{itemize}
  \item \textbf{레이어별 샤딩 패턴은 동일.}
        하나의 TP 그룹 내부에서, 순전파/역전파 그래프 구조,
        All-Reduce/All-Gather의 위치, 텐서 모양은
        TP-only 설정과 완전히 동일하다.
  \item \textbf{DP 복제본 방향의 기울기 All-Reduce 추가.}
        각 파라미터 샤드는 이제 DP 축에 대해
        한 번 더 All-Reduce에 참여하여,
        서로 다른 미니배치 조각에서 온 기울기를 평균낸다.
  \item \textbf{로컬 활성값은 변하지 않음.}
        TP 그룹 내부의 활성 메모리 사용량과 계산 패턴은
        DP를 추가하더라도 변하지 않는다.
        DP는 각 복제본이 어떤 샘플을 보는지와,
        파라미터 기울기를 어떻게 합치는지만 바꾼다.
\end{itemize}

\paragraph{DP-only와 비교.}

\begin{itemize}
  \item \textbf{각 복제본 안의 샤딩된 모델.}
        DP-only에서는 각 복제본이 모델 전체를 그대로 가지고 있지만,
        DP+TP에서는 각 복제본이 TP-샤딩된 모델을 가진다.
        즉, 파라미터와 일부 활성값이 $N_T$ 디바이스에 나누어져 있다.
  \item \textbf{더 빈번한 집합 통신.}
        DP-only에서는 각 파라미터 텐서에 대해
        스텝당 한 번의 All-Reduce만 수행한다.
        반면 DP+TP에서는 여기에 더해,
        대부분의 레이어(MHA, MLP, 임베딩, LM head 등) 내부에
        TP 그룹 내부용 All-Reduce/All-Gather가 추가된다.
  \item \textbf{디바이스당 메모리·연산량.}
        TP는 디바이스당 파라미터·활성 메모리를
        대략 $1/N_T$만큼 줄여 주지만,
        그 대가로 레이어 내부 통신이 늘어난다.
        DP는 배치를 나누어 디바이스당 활성 메모리를
        추가로 $1/N_D$ 정도 줄여 준다.
\end{itemize}

전체적으로 DP+TP는 다음과 같이 이해할 수 있다.

\begin{quote}
  “Section~\ref{sec:tp}의 TP-샤딩된 트랜스포머 레이어를 가져와,
  Section~\ref{sec:dp}에서처럼 서로 다른 미니배치 조각에 대해 실행한 뒤,
  각 파라미터 샤드마다 복제본 간 기울기를 평균내기 위한
  All-Reduce를 하나 더 추가한다.”
\end{quote}

TP로 인해 샤딩된 활성값이 전역적으로 보여야 하는 곳
(예: 전체 어휘에 대한 logits)에서는 All-Gather가 사용되고,
뿐만 아니라 TP의 부분 결과를 합치거나
데이터 병렬 복제본 사이에서 기울기를 집계할 때는
All-Reduce가 사용된다.
