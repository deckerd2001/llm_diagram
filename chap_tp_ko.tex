% ==========================================================
% 6. 텐서 병렬화 (TP)
% ==========================================================
\section{텐서 병렬화}
\label{sec:tp}

텐서 병렬화에서는 각 레이어의 파라미터를 하나 이상의 텐서 차원을 따라
여러 디바이스에 나누어 저장한다. 모든 디바이스에 동일한 가중치 행렬 전체를
복제하는 대신, 각 디바이스는 가중치의 일부(shard)만을 보유하고,
그에 대응되는 활성값의 일부에 대해서만 연산을 수행한다.
이후 집합 통신(예: All-Reduce, All-Gather)을 사용하여
이 부분 결과들을 모아, Section~\ref{sec:sn}에서 정의한
단일 노드 모델과 동일한 모양의 텐서를 복원한다.

텐서 병렬도의 크기를 $N_T$로 두고,
디바이스 인덱스는 $t \in \{0,\dots,N_T-1\}$로 표기한다.
이 장에서는 Section~\ref{sec:sn}에서의 단일 노드 계산을
이들 $N_T$개 디바이스로 어떻게 나누어 담는지에 초점을 맞춘다.

\textbf{핵심 아이디어는 다음과 같다.}
\begin{itemize}
  \item 큰 가중치 행렬은 행 또는 열 방향으로 나누어,
        각 디바이스가 전체 행렬 $W$ 대신 부분 행렬 $W^{(t)}$만 가지도록 한다.
  \item 각 디바이스는 자신이 가진 부분 가중치와 부분 활성값만을 사용해
        \emph{지역(local) 부분 결과}를 계산한다.
  \item 집합 통신(All-Reduce, All-Gather 등)을 통해
        각 디바이스의 부분 결과를 모아,
        단일 노드 계산 그래프에서 나타나는 것과 동일한 텐서를 재구성한다.
  \item 역전파는 순전파에서의 분할 및 통신 패턴을 그대로 따라가며,
        파라미터와 입력에 대한 기울기가 올바르게 합산되도록 한다.
  \item 디바이스당 메모리 사용량은 대략 $N_T$배 감소하지만,
        각 “글로벌” 행렬 곱은 적어도 한 번의 집합 통신을 필요로 한다.
\end{itemize}

\subsection{텐서 병렬 분할 개요}

큰 그림에서 보면, 텐서 병렬화는 Section~\ref{sec:sn}에서 등장하는
모든 큰 행렬 곱을, 여러 디바이스에서 병렬로 수행되는
작은 행렬 곱들의 집합으로 바꾸는 것으로 볼 수 있다.
선형 레이어
\[
  \mathbf{Y} = \mathbf{X} W + \mathbf{b}, \qquad
  \mathbf{X} \in \mathbb{R}^{B \times D_{\text{in}}},\;
  W \in \mathbb{R}^{D_{\text{in}} \times D_{\text{out}}}
\]
를 분할하는 기본 패턴은 두 가지이다.

\begin{itemize}
  \item \textbf{컬럼 병렬(출력 축 분할) 선형}:
        출력 차원을 기준으로 $W$를 쪼개어
        $W = [W^{(0)},\dots,W^{(N_T-1)}]$,
        $W^{(t)} \in \mathbb{R}^{D_{\text{in}} \times D_{\text{out}}^{(t)}}$
        로 둔다.
        각 디바이스는
        \[
          \mathbf{Y}^{(t)} = \mathbf{X} W^{(t)} + \mathbf{b}^{(t)}
        \]
        를 계산하고, 전체 출력은 단순히 이어 붙여 얻는다:
        \[
          \mathbf{Y} = \mathrm{Concat}_t \mathbf{Y}^{(t)}.
        \]
        순전파 경로에서는 별도의 집합 통신이 필요 없지만,
        이후 레이어에서 이 분할된 축을 다시 모으거나(All-Gather),
        합산(All-Reduce)해야 할 수도 있다.
  \item \textbf{로우 병렬(입력 축 분할) 선형}:
        입력 차원을 기준으로 $W$를 나누고,
        활성값 $\mathbf{X}$도 같은 방식으로 분할한다.
        각 디바이스는 $\mathbf{X}^{(t)}$와 $W^{(t)}$를 가지고 있으며,
        \[
          W^{(t)} \in \mathbb{R}^{D_{\text{in}}^{(t)} \times D_{\text{out}}},
          \qquad
          \mathbf{Y}^{(t)} = \mathbf{X}^{(t)} W^{(t)}
        \]
        를 계산한다.
        이후 $t$에 대한 All-Reduce를 통해 전체 출력을 얻는다:
        \[
          \mathbf{Y} = \sum_{t=0}^{N_T-1} \mathbf{Y}^{(t)}.
        \]
\end{itemize}

트랜스포머 내부에서는 이 두 패턴을 적절히 조합하여,
대부분의 연산이 각 디바이스 로컬로 유지되도록 하고,
레이어당 필요한 All-Reduce/All-Gather 호출 수를 최소화한다.
이러한 스킴을 단일 트랜스포머 블록에 적용한 개략적인 구조는
Figure~\ref{fig:tp_overall_flow}에 나와 있으며,
단일 노드 개요 그림(Figure~\ref{fig:single_node_overall})와 비교해서
보면 이해하기 쉽다.

\begin{figure}[htbp]
  \centering
  \resizebox{\linewidth}{!}{%
\begin{tikzpicture}[
    node distance=2.5cm,
    >=stealth,
    block/.style={rectangle, draw=black, fill=white, text width=5em, text centered, rounded corners, minimum height=8em, font=\bfseries},
    blockstack/.style={rectangle, draw=black!60, fill=white, text width=5em, text centered, rounded corners, minimum height=8em},
    forward/.style={-{Stealth[length=2mm]}, thick, black},
    backward/.style={-{Stealth[length=2mm]}, thick, black, densely dashed},
    allreduce/.style={{Stealth[length=1.5mm]}-{Stealth[length=1.5mm]}, thick, red!70, densely dotted},
    io/.style={text centered, font=\bfseries}
]
    % Title
    % \node[font=\Large\bfseries] at (10, 6) {Transformer Overall Flow (TP with 3 GPUs)};

    % Forward path nodes (horizontal)
    \node (input) [io] {$\mathbf{X}$};
    \node (encoding) [block, right of=input, yshift=-3em] {Input\\Encoding};

    % MHA blocks (3 stacked) - GPU 2 (back)
    \node (mha3) [blockstack, right of=encoding, xshift=0.3cm, yshift=0.3cm] {};
    % MHA blocks (3 stacked) - GPU 1 (middle)
    \node (mha2) [blockstack, right of=encoding, xshift=0.15cm, yshift=0.15cm] {};
    % MHA blocks (3 stacked) - GPU 0 (front)
    \node (mha) [block, right of=encoding] {MHA};

    % Small GPU labels for MHA
    \node[font=\tiny, anchor=north east] at (mha.north east) {GPU 0};

    % FFN blocks (3 stacked) - GPU 2 (back)
    \node (mlp3) [blockstack, right of=mha, xshift=0.3cm, yshift=0.3cm] {};
    % FFN blocks (3 stacked) - GPU 1 (middle)
    \node (mlp2) [blockstack, right of=mha, xshift=0.15cm, yshift=0.15cm] {};
    % FFN blocks (3 stacked) - GPU 0 (front)
    \node (mlp) [block, right of=mha] {FFN};

    % Small GPU labels for FFN
    \node[font=\tiny, anchor=north east] at (mlp.north east) {GPU 0};

    \node (output) [block, right of=mlp] {Output\\Projection};
    \node (pred) [io, right of=output, yshift=3em] {$\mathbf{Y}$};
    \node (loss) [align=center, io, right of=output] {\small LOSS:\\$\mathcal{L}(\mathbf{Y,Y_\text{targets}})$};
    \node (gradient) [io, right of=output, yshift=-3em] {$\mathbf{dY}(=\frac{dL}{dY})$};

    % All-Reduce arrows (Backward drawn first, then blocks, then Forward on top)
    % Backward All-Reduce - FFN (left-bottom corner, lower position)
    \draw [allreduce] ([yshift=-0.2cm]mlp.south west) -- ([xshift=0.3cm, yshift=0.1cm]mlp.south west);
    \node[font=\tiny, red!70, anchor=north] at ([xshift=0.15cm, yshift=-0.25cm]mlp.south west) {all\_reduce(dX)};

    % Backward All-Reduce - MHA (left-bottom corner, lower position)
    \draw [allreduce] ([yshift=-0.2cm]mha.south west) -- ([xshift=0.3cm, yshift=0.1cm]mha.south west);
    \node[font=\tiny, red!70, anchor=north] at ([xshift=0.15cm, yshift=-0.25cm]mha.south west) {all\_reduce(dX)};

    % Redraw blocks to cover backward All-Reduce arrows
    \draw [draw=black!60, fill=white, rounded corners] (mha3.south west) rectangle (mha3.north east);
    \draw [draw=black!60, fill=white, rounded corners] (mha2.south west) rectangle (mha2.north east);
    \draw [draw=black, fill=white, rounded corners, line width=0.4pt] (mha.south west) rectangle (mha.north east);
    \node[font=\bfseries] at (mha.center) {MHA};
    \node[font=\tiny, anchor=north east] at (mha.north east) {GPU 0};

    \draw [draw=black!60, fill=white, rounded corners] (mlp3.south west) rectangle (mlp3.north east);
    \draw [draw=black!60, fill=white, rounded corners] (mlp2.south west) rectangle (mlp2.north east);
    \draw [draw=black, fill=white, rounded corners, line width=0.4pt] (mlp.south west) rectangle (mlp.north east);
    \node[font=\bfseries] at (mlp.center) {FFN};
    \node[font=\tiny, anchor=north east] at (mlp.north east) {GPU 0};

    % Forward All-Reduce - MHA (right-top corner) - drawn after blocks to appear on top
    \draw [allreduce] (mha.north east) -- ([xshift=0.3cm, yshift=0.3cm]mha.north east);
    \node[font=\tiny, red!70, anchor=south] at ([xshift=0.15cm, yshift=0.35cm]mha.north east) {all\_reduce(X)};

    % Forward All-Reduce - FFN (right-top corner) - drawn after blocks to appear on top
    \draw [allreduce] (mlp.north east) -- ([xshift=0.3cm, yshift=0.3cm]mlp.north east);
    \node[font=\tiny, red!70, anchor=south] at ([xshift=0.15cm, yshift=0.35cm]mlp.north east) {all\_reduce(X)};

    % Forward arrows (upper part of blocks)
    \draw [forward] (input) -- ([yshift=3em]encoding.west);
    \draw [forward] ([yshift=3em]encoding.east) -- ([yshift=3em]mha.west);
    \draw [forward] ([yshift=3em]mha.east) -- ([yshift=3em]mlp.west);
    \draw [forward] ([yshift=3em]mlp.east) -- ([yshift=3em]output.west);
    \draw [forward] ([yshift=3em]output.east) -- (pred);
    \draw [forward] (pred) -- (loss);
    \draw [backward] (loss) -- (gradient);

    % Backward arrows (lower part of blocks)
    \draw [backward] (gradient) -- ([yshift=-3em]output.east);
    \draw [backward] ([yshift=-3em]output.west) -- ([yshift=-3em]mlp.east);
    \draw [backward] ([yshift=-3em]mlp.west) -- ([yshift=-3em]mha.east);
    \draw [backward] ([yshift=-3em]mha.west) -- ([yshift=-3em]encoding.east);

    % Brace for layer repetition (horizontal - using mlp with x offset to match mlp3)
    \draw[decorate, decoration={brace, amplitude=10pt}]
        ([yshift=4.8em]mha.north west) -- ([xshift=0.3cm, yshift=4.8em]mlp.north east)
        node[midway, above=12pt, font=\normalsize] {$N$ layers};

    % Labels (Legend)
    \coordinate (legend) at ([xshift=-1.5cm, yshift=6.5em]input);
    \draw[forward] (legend) -- ++(0.8,0) node[right, font=\normalsize] {Forward};
    \draw[backward] ([yshift=-0.6cm]legend) -- ++(0.8,0) node[right, font=\normalsize] {Backward};
    \draw[allreduce] ([yshift=-1.2cm]legend) -- ++(0.8,0) node[right, font=\normalsize] {All-Reduce};

    % TP explanation
    \node[font=\scriptsize, align=left, text width=6cm] at ([xshift=2cm, yshift=-5.5em]encoding.south) {
        \textbf{Tensor Parallelism:}\\
        • Each GPU processes a shard of the weight matrix\\
        • All-Reduce synchronizes partial results\\
        • Forward: after row-parallel ops\\
        • Backward: after column-parallel ops
    };

\end{tikzpicture}%
}
  \caption{텐서 병렬화가 적용된 전체 트랜스포머 레이어.
  단일 노드 모델의 큰 선형 레이어는 $N_T$개의 작은 matmul로 나뉘어
  서로 다른 디바이스에서 실행된다.
  색깔 화살표는 부분 결과를 모으기 위해 집합 통신
  (예: All-Reduce, All-Gather)이 필요한 위치를 나타내고,
  나머지 로컬 계산은 Figure~\ref{fig:single_node_overall}과
  구조적으로 동일하다.}
  \label{fig:tp_overall_flow}
\end{figure}

% ------------------------ 6.1 MHA with Tensor Parallelism -------------
\subsection{텐서 병렬화된 MHA}

이제 텐서 병렬화를 멀티헤드 어텐션(MHA) 블록에 적용해 보자.
Section~\ref{sec:sn}.2에서 보았듯이, 단일 노드의 어텐션 블록은
입력 $\mathbf{X} \in [B,S,D]$를
Q/K/V 프로젝션, 스케일된 내적 어텐션, 출력 프로젝션, 잔차 연결을 통해
$\mathbf{A}_{\text{out}} \in [B,S,D]$로 사상한다.

텐서 병렬화에서는 이 계산을 $N_T$개 디바이스에 다음과 같이 나눈다.
\begin{itemize}
  \item 각 디바이스는 일부 어텐션 헤드, 또는 동등하게
        Q/K/V 프로젝션 출력 채널의 부분 집합만을 담당한다.
  \item softmax와 값(value) 기반 가중합은 각 디바이스가
        자신이 담당하는 헤드에 대해서만 로컬로 계산한다.
  \item 출력 프로젝션은 로우 병렬(입력 분할) 형태로 구현하여,
        각 디바이스의 부분 결과를 All-Reduce로 합산한 뒤,
        단일 노드와 동일한 $\mathbf{A}_{\text{out}}$을 복원한다.
\end{itemize}

\subsubsection{순전파}

입력은 단일 노드와 마찬가지로
\[
  \mathbf{X} \in \mathbb{R}^{B \times S \times D}
\]
이고, $N_H$개의 헤드와 각 헤드 차원 $D_h$에 대해
$D = N_H D_h$를 가정한다.
텐서 병렬화에서는 헤드를 $N_T$ 디바이스에 나누어,
각 디바이스가 $N_H^{(t)}$개 헤드와 차원 $D_h^{(t)}$를 갖도록 한다
(이들의 합이 전체 $N_H$, $D_h$가 되도록).

\paragraph{(1) 정규화와 공유 입력.}
먼저 레이어 정규화를 적용한다.
\[
  \mathbf{X}_{\text{norm}} = \mathrm{LN}(\mathbf{X})
  \in \mathbb{R}^{B \times S \times D}.
\]
$\mathbf{X}_{\text{norm}}$는 텐서 병렬 디바이스 전체에 \emph{공유}되며,
각 디바이스는 동일한 $\mathbf{X}_{\text{norm}}$를 본다.

\paragraph{(2) Q/K/V 컬럼 병렬 프로젝션.}
Q/K/V 프로젝션은 컬럼 병렬 선형으로 구현하여,
각 디바이스가 출력 채널의 일부만 갖도록 한다.
\[
  W_Q = [W_Q^{(0)},\dots,W_Q^{(N_T-1)}], \quad
  W_K = [W_K^{(0)},\dots,W_K^{(N_T-1)}], \quad
  W_V = [W_V^{(0)},\dots,W_V^{(N_T-1)}],
\]
\[
  W_Q^{(t)}, W_K^{(t)}, W_V^{(t)}
  \in \mathbb{R}^{D \times D_{\text{head}}^{(t)}},
\]
와 같이 두면, 각 디바이스는 다음을 계산한다.
\[
  \mathbf{Q}^{(t)} = \mathbf{X}_{\text{norm}} W_Q^{(t)}, \quad
  \mathbf{K}^{(t)} = \mathbf{X}_{\text{norm}} W_K^{(t)}, \quad
  \mathbf{V}^{(t)} = \mathbf{X}_{\text{norm}} W_V^{(t)}.
\]
여기서 $D_{\text{head}}^{(t)}$는 디바이스 $t$가 담당하는
헤드 차원의 총합이다.
헤드 차원을 명시적으로 드러내면,
\[
  \mathbf{Q}^{(t)}, \mathbf{K}^{(t)}, \mathbf{V}^{(t)}
  \in \mathbb{R}^{B \times N_H^{(t)} \times S \times D_h}.
\]

\paragraph{(3) 로컬 스케일드 어텐션.}
각 디바이스는 자신이 담당하는 헤드에 대해서만
스케일된 내적 어텐션을 계산한다.
\[
  \mathbf{S}^{(t)} = \frac{\mathbf{Q}^{(t)} (\mathbf{K}^{(t)})^{\top}}
                           {\sqrt{D_h^{(t)}}},\qquad
  \mathbf{A}_S^{(t)} = \mathrm{Softmax}(\mathrm{Mask}(\mathbf{S}^{(t)})),
\]
\[
  \mathbf{A}_{\text{heads}}^{(t)} = \mathbf{A}_S^{(t)} \mathbf{V}^{(t)}.
\]
이 과정에는 디바이스 간 통신이 필요 없다.

\paragraph{(4) 헤드 결합과 로우 병렬 출력 프로젝션.}
각 디바이스는 자신이 담당하는 헤드들만 모아
\[
  \mathbf{A}_{\text{cat}}^{(t)}
  \in \mathbb{R}^{B \times S \times D^{(t)}}
\]
를 만든다. 이후 출력 프로젝션을 로우 병렬 선형으로 구현한다.
\[
  W_O
    = \begin{bmatrix} W_O^{(0)} \\ \vdots \\ W_O^{(N_T-1)} \end{bmatrix},\qquad
  W_O^{(t)} \in \mathbb{R}^{D^{(t)} \times D}.
\]
각 디바이스는
\[
  \mathbf{A}_{\text{lin}}^{(t)}
    = \mathbf{A}_{\text{cat}}^{(t)} W_O^{(t)} + \mathbf{b}_O^{(t)}
\]
를 계산하고, $t$에 대한 All-Reduce를 통해
\[
  \mathbf{A}_{\text{lin}}
    = \sum_{t=0}^{N_T-1} \mathbf{A}_{\text{lin}}^{(t)}
\]
을 얻는다.
이제 모든 디바이스가 동일한 $\mathbf{A}_{\text{lin}}$을 가지므로,
이후 드롭아웃과 입력 $\mathbf{X}$와의 잔차 연결은
각 디바이스에서 로컬로 적용할 수 있다.

이 순서와 All-Reduce의 위치는
Figure~\ref{fig:mha_forward_tp}에 명시적으로 표시되어 있다.

\begin{landscape}
\begin{figure}[p]
  \centering
  \resizebox{\linewidth}{!}{%
\begin{tikzpicture}[
  every node/.style={transform shape},
  >=stealth,
  auxnode/.style={draw, rectangle, fill=white, minimum height=6mm, inner sep=2pt, font=\footnotesize, align=center},
  mulnode/.style={draw, circle, fill=white, minimum size=6mm, font=\footnotesize, align=center},
  addnode/.style={draw, circle, fill=white, minimum size=6mm, font=\footnotesize, align=center},
  allreduce/.style={draw, rectangle, fill=red!30, minimum height=6mm, inner sep=2pt, font=\footnotesize, align=center, thick, draw=red!70},
  flow/.style={->, thick, black!85},
  flow2/.style={->, double, thick, black!85},
  commflow/.style={<->, thick, red!70, dashed},
  dimlabel/.style={font=\tiny, inner sep=0.5pt, align=center}
]
% \node[font=\Large\bfseries] at (9, 4.5) {Multi-Head Attention Forward Pass (Node $i$)};

\node (Input) at (0.5, 0) [align=center] {$\mathbf{X}$\\$[B,S,D]$};
\node[auxnode] (LN) [right=0.5cm of Input] {LN};

\node[mulnode] (Proj_Q) [right=2.0cm of LN, yshift=3.2cm] {$\bullet$};
\node[auxnode] (R_Q) [right=1.5cm of Proj_Q] {R};

\node[mulnode] (Proj_K) [right=2.0cm of LN, yshift=0cm] {$\bullet$};
\node[auxnode] (R_K) [right=1.5cm of Proj_K] {R};

\node[mulnode] (Proj_V) [right=2.0cm of LN, yshift=-3.2cm] {$\bullet$};
\node[auxnode] (R_V) [right=1.5cm of Proj_V] {R};

\node[dimlabel] (WQ) [align=center, below=1.0cm of Proj_Q] {$\widetilde{\mathbf{W}}_{Q}^{(i)}$\\$[D,N_{HN}D_h]$};
\node[dimlabel] (WK) [align=center, below=1.0cm of Proj_K] {$\widetilde{\mathbf{W}}_{K}^{(i)}$\\$[D,N_{HN}D_h]$};
\node[dimlabel] (WV) [align=center, below=1.0cm of Proj_V] {$\widetilde{\mathbf{W}}_{V}^{(i)}$\\$[D,N_{HN}D_h]$};

\node[auxnode] (T_K) [right=2.0cm of R_K] {T};
\node[mulnode] (QK) [right=3.2cm of R_Q, yshift=-1.6cm] {$\bullet$};
\node[auxnode] (SM) [right=2.0cm of QK] {SM};
\node[auxnode] (Soft) [right=2.0cm of SM] {S};
\node[mulnode] (PV) [right=2.4cm of Soft, yshift=-1.6cm] {$\bullet$};

\node[auxnode] (R_Merge) [right=2.0cm of PV] {R};
\node[auxnode] (Cat) [right=2.0cm of R_Merge] {C};

\node[mulnode] (OProj) [right=2.5cm of Cat] {$\bullet$};
\node[dimlabel] (WO_FWD) [align=center, below=1.0cm of OProj] {$\widetilde{\mathbf{W}}_{O}^{(i)}$\\$[N_{HN}D_h,D]$};
\node[allreduce] (AR) [right=2.0cm of OProj] {AR};
\node[
  align=center,
  below=2.5cm of AR,
  text width=5.5cm % 적당한 값으로 조정
] (AR_info) {%
  \textbf{All Reduce Comm.}:\\[2pt]
  \begin{itemize}
    \item \textbf{Naive:} $2(N_T-1) \times [B,S,D]$
    \item \textbf{Ring:} $2\frac{N_T-1}{N_T} \times [B,S,D]$
  \end{itemize}
};
\node[addnode] (AddB) [right=2.0cm of AR] {+};
\node[dimlabel] (BO) [align=center, below=1.0cm of AddB] {$\widetilde{\mathbf{b}}_{O}$\\$[D]$};
\node[auxnode] (Drop) [right=1.5cm of AddB] {DO};
\node (Aout) [align=center, right=0.5cm of Drop] {$\mathbf{A}_{\text{out}}$\\$[B,S,D]$};

\draw[flow] (Input) -- (LN);

\draw[flow] (LN.east) -- ++(0.3,0) |- (Proj_Q.west);
\draw[flow] (LN) -- (Proj_K.west) node[dimlabel, midway, above]{$\mathbf{X}_{\text{norm}}$\\$[B,S,D]$};
\draw[flow] (LN.east) -- ++(0.3,0) |- (Proj_V.west);

\draw[flow2] (WQ) -- (Proj_Q);
\draw[flow2] (WK) -- (Proj_K);
\draw[flow2] (WV) -- (Proj_V);

\draw[flow] (Proj_Q) -- (R_Q) node[dimlabel, midway, above]{$[B,S,N_{HN}D_h]$};
\draw[flow] (Proj_K) -- (R_K) node[dimlabel, midway, above]{$[B,S,N_{HN}D_h]$};
\draw[flow] (Proj_V) -- (R_V) node[dimlabel, midway, above]{$[B,S,N_{HN}D_h]$};

\draw[flow] (R_Q) -| (QK) node[dimlabel, near start, above]{$\mathbf{Q}_i$\\$[B,N_{HN},S,D_h]$};
\draw[flow] (R_K) -- (T_K) node[dimlabel, midway, above]{$\mathbf{K}_i$\\$[B,N_{HN},S,D_h]$};
\draw[flow2] (T_K) -| (QK) node[dimlabel, near end, right]{$\mathbf{K}_i^{T}$\\$[B,N_{HN},D_h,S]$};

\draw[flow] (QK) -- (SM) node[dimlabel, midway, above]{$\mathbf{A}_i$\\$[B,N_{HN},S,S]$};
\draw[flow] (SM) -- (Soft) node[dimlabel, midway, above]{$[B,N_{HN},S,S]$};
\draw[flow] (Soft) -| (PV) node[dimlabel, near start, above]{$\mathbf{AS}_i$\\$[B,N_{HN},S,S]$};
\draw[flow2] (R_V) -| (PV) node[dimlabel, pos=0.08, above]{$\mathbf{V}_i$\\$[B,N_{HN},S,D_h]$};

\draw[flow] (PV) -- (R_Merge) node[dimlabel, midway, above]{$\mathbf{AO}_{\text{h},i}$\\$[B,N_{HN},S,D_h]$};
\draw[flow] (R_Merge) -- (Cat) node[dimlabel, midway, above]{$[B,S,N_{HN},D_h]$};
\draw[flow] (Cat) -- (OProj) node[dimlabel, midway, above]{$\mathbf{AO}_{\text{c},i}$\\$[B,S,N_{HN}D_h]$};
\draw[flow2] (WO_FWD) -- (OProj);
\draw[flow] (OProj) -- (AR) node[dimlabel, midway, above]{$\mathbf{AO}_{\text{l},i}$\\$[B,S,D]$};

% All-Reduce communication arrows
\draw[commflow] (AR.north) -- ++(0, 2.0) node[midway, right, font=\tiny]{Node $j$};
\draw[commflow] (AR.south) -- ++(0, -2.4) node[midway, right, font=\tiny]{Node $k$};

\draw[flow] (AR) -- (AddB) node[dimlabel, midway, above]{$[B,S,D]$};
\draw[flow] (BO) -- (AddB);
\draw[flow] (AddB) -- (Drop) node[dimlabel, midway, above]{$[B,S,D]$};
\draw[flow] (Drop) -- (Aout);
\end{tikzpicture}%
}
  \caption{텐서 병렬화된 MHA의 순전파.
  Q/K/V 프로젝션은 컬럼 병렬 선형으로 구현되어,
  각 디바이스가 일부 헤드만을 소유한다.
  각 디바이스는 자신이 담당하는 헤드에 대해 어텐션을 로컬로 계산하고,
  헤드 출력들을 합친 뒤, 로우 병렬 출력 프로젝션을 적용한다.
  이후 All-Reduce를 통해 모든 디바이스에서
  단일 노드와 동일한 $\mathbf{A}_{\text{out}}$를 복원한다.}
  \label{fig:mha_forward_tp}
\end{figure}
\end{landscape}

\subsubsection{역전파}

텐서 병렬 MHA의 역전파는
Section~\ref{sec:sn}.2의 단일 노드 역전파와 같은 상위 구조를 따르되,
기울기가 샤딩되어 있고, 집합 통신이 명시적으로 들어간다는 점만 다르다.
각 디바이스에서 $\mathrm{d}\mathbf{A}_{\text{out}}$로부터 시작하여,
다음과 같은 단계가 진행된다.

\begin{itemize}
  \item \textbf{출력 프로젝션 역전파}:
        로우 병렬 출력 프로젝션은
        $\mathrm{d}\mathbf{A}_{\text{lin}}$으로부터
        로컬 기울기 $\mathrm{d}\mathbf{A}_{\text{cat}}^{(t)}$와
        파라미터 기울기 $\mathrm{d}W_O^{(t)}, \mathrm{d}\mathbf{b}_O^{(t)}$를 계산한다.
  \item \textbf{헤드 역전파}:
        각 디바이스는 자신이 담당하는 헤드를 따라 역전파를 수행하여
        $\mathrm{d}\mathbf{V}^{(t)}$, $\mathrm{d}\mathbf{A}_S^{(t)}$를 얻고,
        다시 스케일된 내적 및 소프트맥스 역전파를 통해
        $\mathrm{d}\mathbf{Q}^{(t)}$, $\mathrm{d}\mathbf{K}^{(t)}$를 계산한다.
  \item \textbf{Q/K/V 프로젝션 역전파}:
        컬럼 병렬 Q/K/V 선형 레이어는
        로컬 파라미터 기울기
        $\mathrm{d}W_Q^{(t)}, \mathrm{d}W_K^{(t)}, \mathrm{d}W_V^{(t)}$와
        정규화된 입력에 대한 부분 기울기
        $\mathrm{d}\mathbf{X}_{\text{norm}}^{(t)}$를 계산한다.
  \item \textbf{입력 기울기에 대한 All-Reduce}:
        $\mathbf{X}_{\text{norm}}$는 모든 디바이스에 공유되므로,
        모든 Q/K/V shard에서 나온 기울기를 합산해야 한다:
        \[
          \mathrm{d}\mathbf{X}_{\text{norm}}
            = \sum_{t=0}^{N_T-1} \mathrm{d}\mathbf{X}_{\text{norm}}^{(t)},
        \]
        이는 $t$에 대한 All-Reduce로 구현된다.
  \item \textbf{레이어 정규화 역전파}:
        마지막으로 레이어 정규화 역전파를 통해
        $\mathrm{d}\mathbf{X}_{\text{norm}}$을
        원래 입력 $\mathbf{X}$에 대한 기울기
        $\mathrm{d}\mathbf{X}$로 변환한다.
\end{itemize}

이 과정은 Figure~\ref{fig:mha_backward_tp}에서
전체 계산 그래프로 펼쳐져 있고,
각 All-Reduce/All-Gather의 위치가 명시되어 있다.

\begin{landscape}
\begin{figure}[p]
  % no \centering here to avoid compilation issues
  \par\vspace{1cm}
\noindent
\resizebox{\linewidth}{!}{%
\begin{tikzpicture}[
  every node/.style={transform shape},
  >=stealth,
  auxnode/.style={draw, rectangle, fill=white, minimum height=6mm, inner sep=2pt, font=\footnotesize, align=center},
  mulnode/.style={draw, circle, fill=white, minimum size=6mm, font=\footnotesize, align=center},
  addnode/.style={draw, circle, fill=white, minimum size=6mm, font=\footnotesize, align=center},
  sumnode/.style={draw, circle, fill=white, minimum size=6mm, font=\tiny, align=center},
  allreduce/.style={draw, rectangle, fill=red!30, minimum height=6mm, inner sep=2pt, font=\footnotesize, align=center, thick, draw=red!70},
  flow/.style={->, thick, black!85},
  flow2/.style={->, double, thick, black!85},
  commflow/.style={<->, thick, red!70, dashed},
  dimlabel/.style={font=\scriptsize, inner sep=1pt, align=center},
  gradflow/.style={->, thick, black!85},
  gradweight/.style={->, thick, black!85}
]

\begin{scope}[xscale=1.45, yscale=1.3]

\def\yoffset{-1.0}
\def\dVXyoffset{-6.5}

\coordinate (Grad_Aout_B) at (17.5, \yoffset);
\coordinate (dDrop_center) at (15.9, \yoffset);
\coordinate (ProjGradSplit) at (14.3, \yoffset);
\coordinate (dOProj_center) at (12.7, \yoffset);
\coordinate (C_center) at (11.1, \yoffset);
\coordinate (R_center) at (9.0, \yoffset);
\coordinate (dPV_AS_calc_center) at (7.5, \yoffset);
\coordinate (dSoft_center) at (5.3, \yoffset);
\coordinate (dSM_calc_center) at (3.3, \yoffset);
\coordinate (dV_calc_center) at (8.2, \dVXyoffset+\yoffset);
\coordinate (R_V_bwd_center) at (5.9, \dVXyoffset+\yoffset);
\coordinate (dVX_calc_center) at (3.9, \dVXyoffset+\yoffset);
\coordinate (dQK_calc_Q_center) at (1.7, \yoffset);
\coordinate (dQK_calc_K_center) at (1.7, -3.3+\yoffset);
\coordinate (K_BWD_input_center) at (1.7, 1.0+\yoffset);
\coordinate (T_Q_bwd_center) at (1.7, -4.0+\yoffset);

\node[font=\Large\bfseries] at (8, 4.6+\yoffset) {Multi-Head Attention Backward Pass (Node $i$)};

\node (Grad_Aout_B) at (18.5, \yoffset) {$\mathbf{dA}_{\text{out}}$\\$[B,S,D]$};
\node[auxnode] (DO) at (dDrop_center) {DO};
\node[mulnode] (dOProj) at (dOProj_center) {$\bullet$};

\node[auxnode] (T_WO) [below=1.0cm of dOProj] {T};
\node[dimlabel] (WO_BWD) [below=0.5cm of T_WO] {$\widetilde{\mathbf{W}}_{O}^{(i)}$\\$[N_{HN}D_h,D]$};

\node[mulnode] (dWO_calc) at ($(ProjGradSplit)+(0, 1.7)$) {$\bullet$};
\node[align=center, left=1.1cm of dWO_calc]
  (dWO_GRAD) {$\mathbf{d}\widetilde{\mathbf{W}}_{O}^{(i)}$\\$[N_{HN}D_h,D]$};
\node[auxnode] (T_AO_in) [right=1.5cm of dWO_calc] {T};
\node[dimlabel] (AO_in_local_label) [right=1.7cm of T_AO_in] {$\mathbf{AO}_{\text{cat},i}$\\$[B,S,$\\$N_{HN}D_h]$};

\node[auxnode] (C) at (C_center) {dC};
\node[auxnode] (R) at (R_center) {R};

\node[mulnode] (dPV_AS_calc) at (dPV_AS_calc_center) {$\bullet$};
\node[auxnode] (dSoft) at (dSoft_center) {dS};
\node[auxnode] (dSM_calc) at (dSM_calc_center) {dSM};

\node[mulnode] (dQK_calc_Q) at (dQK_calc_Q_center) {$\bullet$};
\node[mulnode] (dQX_proj_calc) [left=6.0cm of dQK_calc_Q] {$\bullet$};

\node[mulnode] (dQK_calc_K) at (dQK_calc_K_center) {$\bullet$};
\node[mulnode] (dKX_proj_calc) [left=6.0cm of dQK_calc_K] {$\bullet$};

\node[dimlabel] (K_BWD_input) at (K_BWD_input_center) {$\mathbf{K}_i$\\$[B,N_{HN},$\\$S,D_h]$};
\node[auxnode] (T_Q_bwd) at (T_Q_bwd_center) {T};
\node[dimlabel] (Q_BWD_input) [below=0.4cm of T_Q_bwd] {$\mathbf{Q}_i$\\$[B,N_{HN},$\\$S,D_h]$};

\node[dimlabel] (V_FWD) [above=1.8cm of dPV_AS_calc] {$\mathbf{V}_i$\\$[B,N_{HN},$\\$S,D_h]$};
\node[auxnode] (T_V_bwd) [below=0.3cm of V_FWD] {T};

\node[mulnode] (dV_calc) at (dV_calc_center) {$\bullet$};
\node[auxnode] (T_AS_bwd) [right=1.6cm of dV_calc] {T};
\node[dimlabel] (AS_BWD_for_V) [right=0.6cm of T_AS_bwd] {$\mathbf{AS}_i$\\$[B,N_{HN},$\\$S,S]$};

\node[auxnode] (R_V_bwd) at (R_V_bwd_center) {R};
\node[mulnode] (dVX_calc) at (dVX_calc_center) {$\bullet$};

\node[auxnode] (T_WV) [above=0.4cm of dVX_calc] {T};
\node[dimlabel] (WV_BWD) [above=0.35cm of T_WV] {$\widetilde{\mathbf{W}}_{V}^{(i)}$\\$[D,N_{HN}D_h]$};

\node[sumnode] (Sum_dBO) [below=1.6cm of ProjGradSplit] {$\sum_{B, S}$};
\node (dBO) [align=center, below=0.75cm of Sum_dBO] {$\mathbf{d}\widetilde{\mathbf{b}}_{O}$\\$[D]$};

\draw[gradflow] (Grad_Aout_B) -- (DO)
  node[dimlabel, midway, above]{$\mathbf{dA}_{\text{out}}$\\$[B,S,D]$};

\draw[gradflow] (DO) -- (dOProj)
  node[dimlabel, pos=0.25, above]{$\mathbf{dAO}_{\text{bias},i}$\\$=\mathbf{dAO}_{\text{lin},i}$\\$[B,S,D]$};

\draw[gradflow] (ProjGradSplit) -- (dWO_calc.south);
\draw[gradflow] (ProjGradSplit) -- ([yshift=-0.75cm]ProjGradSplit) -| (Sum_dBO.north);

\draw[gradflow] (dOProj) -- (C)
  node[dimlabel, midway, above]{$\mathbf{dAO}_{\text{cat},i}$\\$[B,S,$\\$N_{HN}D_h]$};
\draw[gradflow] (C) -- (R)
  node[dimlabel, midway, above]{$[B,S,$\\$N_{HN},D_h]$};

\coordinate (R_split_point) at ($(dPV_AS_calc)!0.5!(R)$);
\draw[gradflow] (R.west) -- (dPV_AS_calc.east);
\draw[flow2] (R_split_point) -- (dV_calc.north)
  node[dimlabel, midway, right]{$\mathbf{dAO}_{\text{heads},i}$\\$[B,N_{HN},$\\$S,D_h]$};

\draw[gradflow] (V_FWD.south) -- (T_V_bwd.north);
\draw[flow2] (T_V_bwd.south) -- (dPV_AS_calc.north)
  node[dimlabel, midway, right]{$[B,N_{HN},$\\$D_h,S]$};
\draw[gradflow] (dPV_AS_calc.west) -- (dSoft.east)
  node[dimlabel, midway, above]{$\mathbf{dAS}_i$\\$[B,N_{HN},$\\$S,S]$};

\node (AS_BWD_dS) [dimlabel, above=0.6cm of dSoft] {$\mathbf{AS}_i$\\$[B,N_{HN},$\\$S,S]$};
\draw[gradflow] (AS_BWD_dS.south) -- (dSoft.north);
\draw[gradflow] (dSoft.west) -- (dSM_calc.east)
  node[dimlabel, midway, above]{$[B,N_{HN},$\\$S,S]$};

\coordinate (dA_Split_X) at ($(dSM_calc_center)!0.5!(dQK_calc_Q_center)$);
\coordinate (dA_Split) at (dA_Split_X |- dQK_calc_Q.east);
\draw[gradflow] (dSM_calc.west) -- (dQK_calc_Q.east);
\draw[flow2] (dA_Split) -- (dA_Split |- dQK_calc_K.east) -- (dQK_calc_K.east)
  node[dimlabel, pos=-1.5, above, yshift=15]{$\mathbf{dA}_i$\\$[B,N_{HN},$\\$S,S]$};

\draw[flow2] (K_BWD_input.south) -- (dQK_calc_Q.north);
\draw[gradweight] (dQK_calc_Q) -- (dQX_proj_calc)
  node[dimlabel, midway, below]{$\mathbf{dQ}_i$\\$[B,N_{HN},$\\$S,D_h]$};

\node[auxnode] (T_WQ_bwd) [above=0.5cm of dQX_proj_calc] {T};
\node[dimlabel] (WQ_bwd) [above=0.5cm of T_WQ_bwd] {$\widetilde{\mathbf{W}}_{Q}^{(i)}$\\$[D,N_{HN}D_h]$};
\draw[flow] (WQ_bwd) -- (T_WQ_bwd);
\draw[flow2] (T_WQ_bwd.south) -- (dQX_proj_calc.north)
  node[dimlabel, midway, right]{$[D,$\\$N_{HN}D_h]$};

\draw[flow] (Q_BWD_input.north) -- (T_Q_bwd.south);
\draw[flow] (T_Q_bwd.north) -- (dQK_calc_K.south)
  node[dimlabel, pos=0.55, right]{$[B,N_{HN},$\\$D_h,S]$};

\node[auxnode] (T_dK) at ($(dQK_calc_K)!0.35!(dKX_proj_calc)$) {T};
\node[auxnode] (R_dK_mid) at ($(T_dK)!0.5!(dKX_proj_calc)$) {R};

\draw[gradweight] (dQK_calc_K) -- (T_dK)
  node[dimlabel, midway, above]{$\mathbf{dK}_i^T$\\$[B,N_{HN},$\\$D_h,S]$};
\draw[gradweight] (T_dK) -- (R_dK_mid)
  node[dimlabel, midway, above]{$[B,S,$\\$N_{HN},D_h]$};
\draw[gradweight] (R_dK_mid) -- (dKX_proj_calc)
  node[dimlabel, midway, above]{$\mathbf{dK}_i$\\$[B,S,$\\$N_{HN}D_h]$};

\node[auxnode] (T_WK_bwd) [above=0.45cm of dKX_proj_calc] {T};
\node[dimlabel] (WK_bwd) [above=0.45cm of T_WK_bwd] {$\widetilde{\mathbf{W}}_{K}^{(i)}$\\$[D,N_{HN}D_h]$};
\draw[gradflow] (WK_bwd) -- (T_WK_bwd);
\draw[flow2] (T_WK_bwd.south) -- (dKX_proj_calc.north);

\draw[gradflow] (AS_BWD_for_V.west) -- (T_AS_bwd.east);
\draw[gradflow] (T_AS_bwd.west) -- (dV_calc.east)
  node[dimlabel, midway, above]{$[B,N_{HN},$\\$S,S]$};
\draw[gradflow] (dV_calc.west) -- (R_V_bwd.east)
  node[dimlabel, midway, above]{$\mathbf{dV}_i$\\$[B,N_{HN},$\\$S,D_h]$};
\draw[gradflow] (R_V_bwd) -- (dVX_calc.east)
  node[dimlabel, midway, above]{$[B,S,$\\$N_{HN}D_h]$};

\draw[gradflow] (WV_BWD) -- (T_WV);
\draw[flow2] (T_WV) -- (dVX_calc.north)
  node[dimlabel, midway, left]{$[D,N_{HN}D_h]$};

\node[addnode] (Sum_dXnorm) [left=1.9cm of dKX_proj_calc] {$+$};

\draw[gradweight] (dQX_proj_calc.west) -| node[dimlabel, pos=0.7, left]{$\mathbf{dX}_{\text{norm},\mathbf{Q},i}$\\$[B,S,D]$} (Sum_dXnorm.north);
\draw[gradweight] (dKX_proj_calc.west) -- node[dimlabel, midway, above]{$\mathbf{dX}_{\text{norm},\mathbf{K},i}$\\$[B,S,D]$} (Sum_dXnorm.east);
\draw[gradweight] (dVX_calc.west) -| node[dimlabel, pos=0.9, left]{$\mathbf{dX}_{\text{norm},\mathbf{V},i}$\\$[B,S,D]$} (Sum_dXnorm.south);

\coordinate (dV_branch) at ($(R_V_bwd.west)!0.52!(dVX_calc.east)$);
\node[mulnode] (dWV_mul) at ($(dV_branch)+(0,-1.7cm)$) {$\bullet$};
\draw[flow2] (dV_branch) -- (dWV_mul.north);

\node[auxnode] (T_Xnorm) [right=1.2cm of dWV_mul] {T};
\node[dimlabel] (Xnorm_local) [right=0.9cm of T_Xnorm] {$\mathbf{X}_{\text{norm}}$\\$[B,S,D]$};
\draw[gradflow] (Xnorm_local) -- (T_Xnorm);
\draw[gradflow] (T_Xnorm.west) -- (dWV_mul.east)
  node[dimlabel, midway, above]{$[B,D,S]$};
\node (dWV_out) [align=center, left=1.1cm of dWV_mul] {$\mathbf{d}\widetilde{\mathbf{W}}_{V}^{(i)}$\\$[D,N_{HN}D_h]$};
\draw[gradweight] (dWV_mul.west) -- (dWV_out);

\coordinate (dQ_branch) at ($(dQK_calc_Q.east)!0.50!(dQX_proj_calc.west)$);
\node[mulnode] (dWQ_mul) at ($(dQ_branch)+(0,3.8cm)$) {$\bullet$};
\node[auxnode] (R_dQ_for_WQ) at ($(dWQ_mul)+(0,-1.0cm)$) {R};
\draw[gradflow]  (dQ_branch) -- (R_dQ_for_WQ.south);
\draw[flow2] (R_dQ_for_WQ.north) -- (dWQ_mul.south)
  node[dimlabel, midway, right]{$[B,S,$\\$N_{HN}D_h]$};

\node[auxnode] (T_XnormQ) [right=1.2cm of dWQ_mul] {T};
\node[dimlabel] (Xnorm_localQ) [right=0.6cm of T_XnormQ] {$\mathbf{X}_{\text{norm}}$\\$[B,S,D]$};
\draw[gradflow] (Xnorm_localQ) -- (T_XnormQ);
\draw[gradflow] (T_XnormQ.west) -- (dWQ_mul.east)
  node[dimlabel, midway, above]{$[B,D,S]$};
\node (dWQ_out) [align=center, left=1.1cm of dWQ_mul] {$\mathbf{d}\widetilde{\mathbf{W}}_{Q}^{(i)}$\\$[D,N_{HN}D_h]$};
\draw[gradweight] (dWQ_mul.west) -- (dWQ_out);

\coordinate (dK_branch) at ($(R_dK_mid)!0.52!(dKX_proj_calc)$);
\node[mulnode] (dWK_mul) at ($(dK_branch)+(0,-1.7cm)$) {$\bullet$};
\draw[flow2]  (dK_branch) -- (dWK_mul.north);

\node[auxnode] (T_XnormK) [right=1.3cm of dWK_mul] {T};
\node[dimlabel, right=0.6cm of T_XnormK] (Xnorm_localK) {$\mathbf{X}_{\text{norm}}$\\$[B,S,D]$};
\draw[gradflow] (Xnorm_localK) -- (T_XnormK);
\draw[gradflow] (T_XnormK.west) -- (dWK_mul.east)
  node[dimlabel, midway, above]{$[B,D,S]$};
\node (dWK_out) [align=center, left=1.1cm of dWK_mul] {$\mathbf{d}\widetilde{\mathbf{W}}_{K}^{(i)}$\\$[D,N_{HN}D_h]$};
\draw[gradweight] (dWK_mul.west) -- (dWK_out);

\draw[gradweight] (Sum_dBO) -- (dBO);

\draw[gradflow] (WO_BWD) -- (T_WO);
\draw[flow2] (T_WO) -- (dOProj)
  node[dimlabel, midway, right]{$\widetilde{\mathbf{W}}_{O}^{(i)T}$\\$[D,$\\$N_{HN}D_h]$};
\draw[gradflow] (AO_in_local_label) -- (T_AO_in);
\draw[flow2] (T_AO_in) -- (dWO_calc.east)
  node[dimlabel, midway, above]{$[B,$\\$N_{HN}D_h,S]$};
\draw[gradweight] (dWO_calc) -- (dWO_GRAD);

\node[auxnode] (dLN) [left=1.8cm of Sum_dXnorm] {dLN};
\draw[gradweight] (Sum_dXnorm.west) -- node[dimlabel, midway, above]
  {$\mathbf{dX}_{\text{norm},i}$\\$[B,S,D]$} (dLN.east);

\node[allreduce] (AR) [left=1.6cm of dLN] {AR};
\node[
  align=center,
  below=2.5cm of AR,
  text width=5.5cm
] (AR_info) {%
  \textbf{All Reduce Comm.}:\\[2pt]
  \begin{itemize}
    \item \textbf{Naive:} $2(N_T-1) \times [B,S,D]$
    \item \textbf{Ring:} $2\frac{N_T-1}{N_T} \times [B,S,D]$
  \end{itemize}
};

% All-Reduce communication arrows
\draw[commflow] (AR.north) -- ++(0, 2.0) node[midway, right, font=\tiny]{Node $j$};
\draw[commflow] (AR.south) -- ++(0, -2.4) node[midway, right, font=\tiny]{Node $k$};

\draw[gradweight] (dLN.west) -- (AR.east) node[dimlabel, midway, above]{$\mathbf{dX}_i$\\$[B,S,D]$};

\node (dX_OUT) [align=center, left=1.1cm of AR] {$\mathbf{dX}$\\$[B,S,D]$};
\draw[gradweight] (AR.west) -- (dX_OUT);

\node[dimlabel] (LNCache) [below=1.0cm of dLN] {$\mathbf{X}$\\$[B,S,D]$};
\draw[gradflow] (LNCache.north) -- (dLN.south);

\end{scope}
\end{tikzpicture}
}
  \caption{텐서 병렬화된 MHA의 역전파.
  각 디바이스는 로컬 Q/K/V 프로젝션과 자신이 담당하는
  어텐션 헤드를 따라 역전파를 수행한다.
  정규화된 입력에 대한 기울기는 All-Reduce를 통해 모든 디바이스에서
  합산되며, 파라미터 기울기는 각 shard
  $W_Q^{(t)}, W_K^{(t)}, W_V^{(t)}, W_O^{(t)}$에 대해
  로컬로 누적된다.}
  \label{fig:mha_backward_tp}
\end{figure}
\end{landscape}

% ------------------------ 6.2 MLP with Tensor Parallelism -------------
\subsection{텐서 병렬화된 MLP}

피드포워드(MLP) 블록은 두 개의 선형 레이어를 가지기 때문에,
하나는 컬럼 병렬, 다른 하나는 로우 병렬로 구현하기에 특히 적합하다.
Section~\ref{sec:sn}.3에서 단일 노드 MLP는
$\mathbf{H} \in [B,S,D]$를 $\mathbf{Y} \in [B,S,D]$로 사상하며,
다음과 같은 구조를 가진다.
\[
  \mathbf{Z}_{\text{up}} = \mathbf{H} W_{\text{up}} + \mathbf{b}_{\text{up}},
  \quad \mathbf{U} = \phi(\mathbf{Z}_{\text{up}}),
\]
\[
  \mathbf{Z}_{\text{down}} = \mathbf{U} W_{\text{down}} + \mathbf{b}_{\text{down}},
  \quad \mathbf{Y} = \mathbf{H} + \mathrm{Dropout}(\mathbf{Z}_{\text{down}}),
\]
여기서 $W_{\text{up}} \in \mathbb{R}^{D \times D_{\text{ff}}}$,
$W_{\text{down}} \in \mathbb{R}^{D_{\text{ff}} \times D}$이다.

텐서 병렬화에서는 $\mathbf{H}$가 모든 디바이스에 공유된다고 가정하고,
상향(up) 선형은 컬럼 병렬, 하향(down) 선형은 로우 병렬로 구현한다.

\subsubsection{순전파}

\paragraph{(1) 컬럼 병렬 상향 프로젝션.}
상향 선형 레이어를
\[
  W_{\text{up}} = [W_{\text{up}}^{(0)},\dots,W_{\text{up}}^{(N_T-1)}],\qquad
  W_{\text{up}}^{(t)} \in \mathbb{R}^{D \times D_{\text{ff}}^{(t)}}
\]
로 나눈다.
각 디바이스는 공유 입력 $\mathbf{H}$에 대해
\[
  \mathbf{Z}_{\text{up}}^{(t)}
    = \mathbf{H} W_{\text{up}}^{(t)} + \mathbf{b}_{\text{up}}^{(t)},
  \qquad
  \mathbf{Z}_{\text{up}}^{(t)} \in \mathbb{R}^{B \times S \times D_{\text{ff}}^{(t)}}
\]
를 계산한다.
전체 $\mathbf{Z}_{\text{up}}$는 개념적으로
\[
  \mathbf{Z}_{\text{up}}
    = \mathrm{Concat}_t \mathbf{Z}_{\text{up}}^{(t)}
\]
로 구성되지만, 이후 연산이 헤드/피처 축을 따라 원소별로 작동하는 경우,
실제로는 각 shard만 가지고 있어도 된다.

\paragraph{(2) 로컬 비선형 활성 함수.}
비선형 함수 $\phi$는 각 디바이스에서 shard별로 적용된다.
\[
  \mathbf{U}^{(t)} = \phi(\mathbf{Z}_{\text{up}}^{(t)}).
\]

\paragraph{(3) 로우 병렬 하향 프로젝션.}
하향 선형 레이어는 입력 차원 방향으로 분할한다.
\[
  W_{\text{down}}
    = \begin{bmatrix}
        W_{\text{down}}^{(0)} \\
        \vdots \\
        W_{\text{down}}^{(N_T-1)}
      \end{bmatrix},
  \qquad
  W_{\text{down}}^{(t)} \in \mathbb{R}^{D_{\text{ff}}^{(t)} \times D}.
\]
각 디바이스는 로컬 shard에 대해
\[
  \mathbf{Z}_{\text{down}}^{(t)}
    = \mathbf{U}^{(t)} W_{\text{down}}^{(t)} + \mathbf{b}_{\text{down}}^{(t)}
\]
를 계산한다.
이후 $t$에 대한 All-Reduce를 수행하여 전체 down-projection 출력을 얻는다.
\[
  \mathbf{Z}_{\text{down}}
    = \sum_{t=0}^{N_T-1} \mathbf{Z}_{\text{down}}^{(t)}.
\]

\paragraph{(4) 드롭아웃과 잔차 연결.}
모든 디바이스가 동일한 $\mathbf{Z}_{\text{down}}$와
$\mathbf{H}$를 가지므로, 드롭아웃과 잔차 연결
\[
  \mathbf{Y} = \mathbf{H} + \mathrm{Dropout}(\mathbf{Z}_{\text{down}})
\]
은 각 디바이스에서 로컬로 수행할 수 있다.

이 전체 순서는 Figure~\ref{fig:mlp_forward_tp}에 요약되어 있으며,
순전파 경로에서 유일한 집합 통신은
down-projection 이후의 All-Reduce이다.

\begin{figure}[htbp]
  \centering
  \resizebox{\linewidth}{!}{%
\begin{tikzpicture}[
    >=stealth,
    auxnode/.style={draw, rectangle, fill=white, minimum height=6mm, inner sep=2pt, font=\footnotesize, align=center},
    mulnode/.style={draw, circle, fill=white, minimum size=6mm, font=\footnotesize, align=center},
    addnode/.style={draw, circle, fill=white, minimum size=6mm, font=\footnotesize, align=center},
    allreduce/.style={draw, rectangle, fill=red!30, minimum height=6mm, inner sep=2pt, font=\footnotesize, align=center, thick, draw=red!70},
    sumnode/.style={draw, circle, fill=white, minimum size=6mm, font=\tiny, align=center},
    flow/.style={->, thick, black!85},
    flow2/.style={double, ->, thick, black!85},
    commflow/.style={<->, thick, red!70, dashed},
    dimlabel/.style={font=\tiny, inner sep=1pt, align=center}
]
    \node[font=\Large\bfseries] at (11, 2.8) {MLP Forward Pass (Node $i$)};

    \pgfmathsetmacro{\verticaloffset}{-0.5}

    \node            (MIn)   at (0,\verticaloffset) {$\mathbf{X}$};
    \node[auxnode]   (LN2)   [right=1.6cm of MIn] {LN};
    \node[mulnode]   (L1Mul) [right=2.0cm of LN2] {$\bullet$};
    \node[dimlabel]  (Wup)   [below=1.0cm of L1Mul] {$\tilde{\mathbf{W}}_{\text{up}}^{(i)}$\\$[D, D_{ff}/N_T]$\\(col-split)};
    \node[addnode]   (AddB1) [right=1.8cm of L1Mul] {+};
    \node[dimlabel]  (Bup)   [below=1.0cm of AddB1] {$\tilde{\mathbf{b}}_{\text{up}}^{(i)}$\\$[D_{ff}/N_T]$};
    \node[auxnode]   (Act)   [right=1.8cm of AddB1] {GL};
    \node[auxnode]   (Drop1) [right=1.8cm of Act] {DO};
    \node[mulnode]   (L2Mul) [right=2.0cm of Drop1] {$\bullet$};
    \node[dimlabel]  (Wdown) [below=1.0cm of L2Mul] {$\tilde{\mathbf{W}}_{\text{down}}^{(i)}$\\$[D_{ff}/N_T, D]$\\(row-split)};
    \node[allreduce] (AR)    [right=1.8cm of L2Mul] {AR};
    \node[
      align=center,
      below=2.5cm of AR,
      text width=5.2cm
    ] (AR_info) {%
      \textbf{All Reduce Comm.}:\\[2pt]
      \begin{itemize}
        \item \textbf{Naive:} $2(N_T-1) \times [B,S,D]$
        \item \textbf{Ring:} $2\frac{N_T-1}{N_T} \times [B,S,D]$
      \end{itemize}
    };
    \node[addnode]   (AddB2) [right=1.8cm of AR] {+};
    \node[dimlabel]  (Bdown) [below=1.0cm of AddB2] {$\tilde{\mathbf{b}}_{\text{down}}$\\$[D]$};
    \node[auxnode]   (Drop2) [right=1.6cm of AddB2] {DO};
    \node            (MOut)  [right=1.6cm of Drop2] {$\mathbf{Y}$};

    % All-Reduce communication arrows
    \draw[commflow] (AR.north) -- ++(0, 1.8) node[midway, right, font=\tiny]{Node $j$};
    \draw[commflow] (AR.south) -- ++(0, -2.1) node[midway, right, font=\tiny]{Node $k$};

    \draw[flow] (MIn) -- (LN2) node[dimlabel, midway, above]{\shortstack{$\mathbf{X}$\\$[B,S,D]$}};
    \draw[flow] (LN2) -- (L1Mul) node[dimlabel, midway, above]{\shortstack{$\mathbf{H}$\\$[B,S,D]$}};
    \draw[flow2] (Wup) -- (L1Mul);
    \draw[flow] (L1Mul) -- (AddB1) node[dimlabel, midway, above]{\shortstack{$\mathbf{Z}_{\text{up},i}$\\$[B,S,D_{ff}/N_T]$}};
    \draw[flow] (Bup) -- (AddB1);
    \draw[flow] (AddB1) -- (Act) node[dimlabel, midway, above]{\shortstack{$\mathbf{A}_{\text{up},i}$\\$[B,S,D_{ff}/N_T]$}};
    \draw[flow] (Act) -- (Drop1) node[dimlabel, midway, above]{\shortstack{$\mathbf{H}_{\text{inter},i}$\\$[B,S,D_{ff}/N_T]$}};
    \draw[flow] (Drop1) -- (L2Mul) node[dimlabel, midway, above]{\shortstack{$\mathbf{U}_{\text{in},i}$\\$[B,S,D_{ff}/N_T]$}};
    \draw[flow2] (Wdown) -- (L2Mul);
    \draw[flow] (L2Mul) -- (AR) node[dimlabel, midway, above]{\shortstack{$\mathbf{Z}_{\text{down},i}$\\$[B,S,D]$}};
    \draw[flow] (AR) -- (AddB2) node[dimlabel, midway, above]{\shortstack{$[B,S,D]$}};
    \draw[flow] (Bdown) -- (AddB2);
    \draw[flow] (AddB2) -- (Drop2) node[dimlabel, midway, above]{\shortstack{$\mathbf{A}_{\text{out}}$\\$[B,S,D]$}};
    \draw[flow] (Drop2) -- (MOut) node[dimlabel, midway, above]{\shortstack{$\mathbf{Y}$\\$[B,S,D]$}};

\end{tikzpicture}%
}
  \caption{텐서 병렬화된 MLP의 순전파.
  상향 프로젝션은 컬럼 병렬 선형으로 구현되어,
  각 디바이스가 중간 피처의 일부만을 보유한다.
  하향 프로젝션은 로우 병렬로 구현되며,
  디바이스 간 All-Reduce를 통해 전체
  $\mathbf{Z}_{\text{down}}$를 복원한 뒤,
  드롭아웃과 잔차 연결이 로컬로 적용된다.}
  \label{fig:mlp_forward_tp}
\end{figure}

\subsubsection{역전파}

텐서 병렬 MLP의 역전파는
단일 노드 역전파 그래프(Figure~\ref{fig:single_node_mlp_backward})와
동일한 구조를 사용하되,
파라미터와 활성값이 샤딩되어 있고,
적절한 위치에 All-Reduce가 들어간다는 점만 다르다.

각 디바이스에서 $\mathrm{d}\mathbf{Y}$로부터 시작하여
다음 순서로 진행된다.

\begin{enumerate}
  \item \textbf{잔차 및 드롭아웃 역전파}:
        단일 노드와 마찬가지로, 기울기는 항등 경로와
        최종 드롭아웃 경로로 나뉘어,
        각 디바이스에서 $\mathrm{d}\mathbf{Z}_{\text{down}}$을 얻는다.
  \item \textbf{하향 프로젝션 역전파 (로우 병렬)}:
        각 디바이스는 자신의 shard $W_{\text{down}}^{(t)}$와
        로컬 활성값 $\mathbf{U}^{(t)}$를 사용해
        \[
          \mathrm{d}\mathbf{U}^{(t)},\quad
          \mathrm{d}W_{\text{down}}^{(t)},\quad
          \mathrm{d}\mathbf{b}_{\text{down}}^{(t)}
        \]
        를 계산한다.
        $\mathrm{d}\mathbf{U}^{(t)}$를 구하는 데에는
        별도의 통신이 필요 없다.
  \item \textbf{활성 함수 역전파}:
        비선형 함수 $\phi$의 역전파를 각 디바이스에서 원소별로 적용하여
        $\mathrm{d}\mathbf{Z}_{\text{up}}^{(t)}$를 얻는다.
  \item \textbf{상향 프로젝션 역전파 (컬럼 병렬)}:
        컬럼 병렬 상향 프로젝션에 대해,
        각 디바이스는 $W_{\text{up}}^{(t)}$에 대한 로컬 기울기와
        입력에 대한 부분 기울기 $\mathrm{d}\mathbf{H}^{(t)}$를 계산한다.
        $\mathbf{H}$는 모든 디바이스에 공유되므로,
        이들을 합산해야 한다:
        \[
          \mathrm{d}\mathbf{H}
            = \sum_{t=0}^{N_T-1} \mathrm{d}\mathbf{H}^{(t)},
        \]
        이는 $t$에 대한 All-Reduce로 구현된다.
  \item \textbf{레이어 정규화 역전파(있는 경우)}:
        MLP 앞에 레이어 정규화가 있는 pre-LN 구조라면,
        $\mathrm{d}\mathbf{H}$는 추가적인 레이어 정규화 역전파를
        거쳐 이전 블록으로 전달된다.
\end{enumerate}

이러한 역전파 과정은 Figure~\ref{fig:mlp_backward_tp}에
전체 계산 그래프로 정리되어 있으며,
입력 기울기 $\mathrm{d}\mathbf{H}$에 대한 All-Reduce 위치가
명확히 표시되어 있다.

\begin{figure}[htbp]
  \centering
  \resizebox{\linewidth}{!}{%
\begin{tikzpicture}[
    >=stealth,
    auxnode/.style={draw, rectangle, fill=white, minimum height=6mm, inner sep=2pt, font=\footnotesize, align=center},
    mulnode/.style={draw, circle, fill=white, minimum size=6mm, font=\footnotesize, align=center},
    addnode/.style={draw, circle, fill=white, minimum size=6mm, font=\footnotesize, align=center},
    sumnode/.style={draw, circle, fill=white, minimum size=6mm, font=\tiny, align=center},
    allreduce/.style={draw, rectangle, fill=red!30, minimum height=6mm, inner sep=2pt, font=\footnotesize, align=center, thick, draw=red!70},
    flow_rev/.style={<-, thick, black!85},
    flow_dw/.style={->, thick, black!85},
    flow_act/.style={double, ->, thick, black!85},
    commflow/.style={<->, thick, red!70, dashed},
    dimlabel/.style={font=\tiny, inner sep=1pt, align=center},
    gradlabel/.style={font=\tiny\bfseries, inner sep=1pt, align=center}
]
    % \node[font=\Large\bfseries] at (5, 10) {MLP Backward Pass (Node $i$)};

    \pgfmathsetmacro{\backwardoffset}{0.0}

    \node (d_MOut) at (12.6, \backwardoffset) {$\mathrm{d}\mathbf{Y}$};
    \node[auxnode] (d_Drop2) [left=1.8cm of d_MOut] {dDO};
    \draw[flow_rev] (d_Drop2) -- (d_MOut)
      node[dimlabel, midway, below]{\shortstack{$\mathrm{d}\mathbf{Y}$\\$[B,S,D]$}};

    \coordinate (split2) at ($(d_Drop2.west) + (-1.5cm, 0)$);
    \coordinate (branch_dUproj) at ($(split2) + (-1.2cm, 0)$);

    \node[sumnode] (d_SumB2) [above=0.8cm of split2] {$\sum_{B, S}$};
    \node (d_Bdown) [above=0.8cm of d_SumB2] {$\mathrm{d}\tilde{\mathbf{b}}_{\text{down}}$};
    \draw[flow_dw] (d_SumB2) -- (d_Bdown) node[dimlabel, midway, right]{$[D]$};

    \draw[flow_rev] (d_SumB2) -- (split2);

    \node[mulnode] (d_L2Mul_in) [left=2.2cm of split2] {$\bullet$};
    \draw[flow_rev] (d_L2Mul_in) -- (d_Drop2)
      node[gradlabel, midway, below]{\shortstack{$\mathrm{d}\mathbf{Z}_{\text{down}}$\\$=\mathrm{d}\mathbf{A}_{\text{out}}$\\$[B,S,D]$}};

    \node[dimlabel] (W_down_T) [align=center, below=1.0cm of d_L2Mul_in] {$\tilde{\mathbf{W}}_{\text{down}}^{(i)T}$\\$[D, D_{ff}/N_T]$\\(row-split)};
    \draw[flow_act] (W_down_T.north) -- (d_L2Mul_in);

    \coordinate (L2Mul_w_y) at ($(d_L2Mul_in) + (0, 2.5cm)$);
    \node[mulnode] (d_L2Mul_w) at (L2Mul_w_y) {$\bullet$};
    \node (d_Wdown) [align=center, above=0.9cm of d_L2Mul_w] {$\mathrm{d}\tilde{\mathbf{W}}_{\text{down}}^{(i)}$\\$[D_{ff}/N_T, D]$};
    \draw[flow_dw] (d_L2Mul_w) -- (d_Wdown);

    \draw[flow_act] (branch_dUproj.north) |- (d_L2Mul_w.east);

    \node[auxnode] (Uin_T) at ($(d_L2Mul_w.west) + (-1.5cm, 0)$) {T};
    \draw[flow_dw] (Uin_T) -- (d_L2Mul_w)
      node[dimlabel, midway, below]{\shortstack{$\mathbf{U}_{\text{in},i}^T$\\$[B, D_{ff}/N_T, S]$}};
    \node (Uin_aux) [left=1.8cm of Uin_T] {$\mathbf{U}_{\text{in},i}$};
    \draw[flow_dw] (Uin_aux) -- (Uin_T) node[dimlabel, midway, above]{\shortstack{$[B,S,$$D_{ff}/N_T]$}};

    \node[auxnode] (d_Drop1) [left=1.8cm of d_L2Mul_in] {dDO};
    \draw[flow_rev] (d_Drop1) -- (d_L2Mul_in)
      node[dimlabel, midway, below]{\shortstack{$\mathrm{d}\mathbf{U}_{\text{in},i}$\\$[B,S,$$D_{ff}/N_T]$}};

    \node[auxnode] (d_Act) [left=1.8cm of d_Drop1] {dGL};
    \draw[flow_rev] (d_Act) -- (d_Drop1)
      node[dimlabel, midway, below]{\shortstack{$\mathrm{d}\mathbf{H}_{\text{inter},i}$\\$[B,S,$$D_{ff}/N_T]$}};

    \coordinate (split1) at ($(d_Act.west) + (-1.5cm, 0)$);
    \coordinate (branch_dHpre) at ($(split1) + (-1.2cm, 0)$);

    \node[sumnode] (d_SumB1) [above=0.8cm of split1] {$\sum_{B, S}$};
    \node[dimlabel] (d_Bup) [above=0.9cm of d_SumB1] {$\mathrm{d}\tilde{\mathbf{b}}_{\text{up}}^{(i)}$\\$[D_{ff}/N_T]$};
    \draw[flow_dw] (d_SumB1) -- (d_Bup);

    \draw[flow_rev] (d_SumB1) -- (split1);

    \node[mulnode] (d_L1Mul_in) [left=2.2cm of split1] {$\bullet$};
    \draw[flow_rev] (d_L1Mul_in) -- (d_Act)
      node[gradlabel, midway, below]{\shortstack{$\mathrm{d}\mathbf{Z}_{\text{up},i}$\\$=\mathrm{d}\mathbf{A}_{\text{up},i}$\\$[B,S,D_{ff}/N_T]$}};

    \node[dimlabel] (W_up_T) [align=center, below=1.0cm of d_L1Mul_in] {$\tilde{\mathbf{W}}_{\text{up}}^{(i)T}$\\$[D_{ff}/N_T, D]$\\(col-split)};
    \draw[flow_act] (W_up_T.north) -- (d_L1Mul_in);

    \coordinate (L1Mul_w_y) at ($(d_L1Mul_in) + (0, 2.5cm)$);
    \node[mulnode] (d_L1Mul_w) at (L1Mul_w_y) {$\bullet$};
    \node (d_Wup) [align=center, above=0.9cm of d_L1Mul_w] {$\mathrm{d}\tilde{\mathbf{W}}_{\text{up}}^{(i)}$\\$[D, D_{ff}/N_T]$};
    \draw[flow_dw] (d_L1Mul_w) -- (d_Wup);

    \draw[flow_act] (branch_dHpre.north) |- (d_L1Mul_w.east);

    \node[auxnode] (Znorm_T) at ($(d_L1Mul_w.west) + (-1.5cm, 0)$) {T};
    \draw[flow_dw] (Znorm_T) -- (d_L1Mul_w)
      node[dimlabel, midway, below]{\shortstack{$\mathbf{H}^T$\\$[B, D, S]$}};
    \node (Znorm_aux) [left=1.8cm of Znorm_T] {$\mathbf{H}$};
    \draw[flow_dw] (Znorm_aux) -- (Znorm_T) node[dimlabel, midway, above]{\shortstack{$[B,S,D]$}};

    \node[allreduce] (AR) [left=1.8cm of d_L1Mul_in] {AR};
    \node[
      align=center,
      below=2.0cm of AR,
      text width=5.2cm
    ] (AR_info) {%
      \textbf{All Reduce Comm.}:\\[2pt]
      \begin{itemize}
        \item \textbf{Naive:} $2(N_T-1) \times [B,S,D]$
        \item \textbf{Ring:} $2\frac{N_T-1}{N_T} \times [B,S,D]$
      \end{itemize}
    };

    % All-Reduce communication arrows
    \draw[commflow] (AR.north) -- ++(0, 1.8) node[midway, right, font=\tiny]{Node $j$};
    \draw[commflow] (AR.south) -- ++(0, -2.1) node[midway, right, font=\tiny]{Node $k$};

    \draw[flow_rev] (AR) -- (d_L1Mul_in)
      node[dimlabel, midway, below]{\shortstack{$\mathrm{d}\mathbf{H}_i$\\$[B,S,D]$}};

    \node[auxnode] (d_LN2) [left=1.8cm of AR] {dLN};
    \draw[flow_rev] (d_LN2) -- (AR)
      node[dimlabel, midway, below]{\shortstack{$\mathrm{d}\mathbf{H}$\\$[B,S,D]$}};

    \node (d_MIn) [left=1.8cm of d_LN2] {$\mathrm{d}\mathbf{X}$};
    \draw[flow_rev] (d_MIn) -- (d_LN2)
      node[dimlabel, midway, below]{\shortstack{$\mathrm{d}\mathbf{X}$\\$[B,S,D]$}};
\end{tikzpicture}%
}
  \caption{텐서 병렬화된 MLP의 역전파.
  로우 병렬 down-projection, 활성 함수, 컬럼 병렬 up-projection,
  (필요 시) 레이어 정규화를 역순으로 따라가며,
  각 shard 파라미터와 입력에 대한 기울기를 계산한다.
  특히 up-projection에서 얻은 입력 기울기
  $\mathrm{d}\mathbf{H}^{(t)}$는 All-Reduce를 통해 합산되어
  전역 $\mathrm{d}\mathbf{H}$를 형성한다.}
  \label{fig:mlp_backward_tp}
\end{figure}

\subsection{요약}

요약하면, 텐서 병렬화는 단일 노드 트랜스포머의 각 큰 선형 연산을
컬럼 병렬 / 로우 병렬 패턴으로 구현함으로써,
\begin{itemize}
  \item 디바이스당 파라미터·활성값 메모리를 줄이고,
  \item 각 디바이스에서의 로컬 계산 구조는 단일 노드와 거의 동일하게 유지하며,
  \item 필요한 최소한의 All-Reduce / All-Gather를 통해
        단일 노드와 동일한 결과를 얻는다.
\end{itemize}

모든 경우에, 각 디바이스에서의 로컬 계산 그래프는
Section~\ref{sec:sn}에서 설명한 단일 노드 계산과 구조적으로 동일하다.
달라지는 것은 \emph{파라미터와 활성값을 어떻게 샤딩하는지, 그리고
어디에서 집합 통신을 수행하는지}뿐이다.
이 때문에 텐서 병렬화는 단일 노드 트랜스포머를 자연스럽게 확장한 형태로,
단일 가속기의 메모리에 들어가지 않는 대규모 모델에 특히 적합하다.
