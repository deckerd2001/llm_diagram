% ==========================================================
% 6. Tensor Parallelism (TP)
% ==========================================================
\section{Tensor Parallelism (TP)}
\label{sec:tp}

In tensor parallelism, the parameters of each layer are partitioned across
multiple devices along one or more tensor dimensions. Instead of
replicating the full weight matrix on every device, each device holds only
a shard of the weights and computes on a corresponding shard of the
activations. Collective communication (e.g., All-Reduce, All-Gather) is
then used to assemble partial results into the same shapes as in the
single-node model of Section~\ref{sec:sn}.

We denote the tensor-parallel degree by $N_T$, and index devices by
$t \in \{0,\dots,N_T-1\}$. Throughout this section we focus on how the
single-node computations from Section~\ref{sec:sn} are decomposed across these
$N_T$ devices.

\textbf{Key ideas.}
\begin{itemize}
  \item Large weight matrices are split along rows or columns so that each
        device holds a sub-matrix $W^{(t)}$ instead of the full matrix $W$.
  \item Each device computes local partial outputs using its own shard.
  \item Collective operations (All-Reduce, All-Gather) combine local
        partial results across devices into the same tensors that appear
        in the single-node computation graph.
  \item The backward pass mirrors the sharding and communication pattern
        of the forward pass so that gradients with respect to parameters
        and inputs are correctly accumulated.
  \item Memory usage per device is reduced approximately by a factor of
        $N_T$, but each global matmul incurs at least one collective
        communication.
\end{itemize}

\subsection{Overview of Tensor-Parallel Sharding}

At a high level, tensor parallelism can be seen as replacing every large
matrix multiplication in Section~\ref{sec:sn} with a collection of smaller matmuls
that run in parallel on different devices. There are two basic patterns
for sharding a linear layer
\[
  \mathbf{Y} = \mathbf{X} W + \mathbf{b}, \qquad
  \mathbf{X} \in \mathbb{R}^{B \times D_{\text{in}}},\;
  W \in \mathbb{R}^{D_{\text{in}} \times D_{\text{out}}}:
\]
\begin{itemize}
  \item \textbf{Column-parallel (output-sharded) linear}: split $W$ along
        the output dimension,
        $W = [W^{(0)},\dots,W^{(N_T-1)}]$ with
        $W^{(t)} \in \mathbb{R}^{D_{\text{in}} \times D_{\text{out}}^{(t)}}$.
        Each device computes
        $\mathbf{Y}^{(t)} = \mathbf{X} W^{(t)} + \mathbf{b}^{(t)}$, and
        the global output is obtained by concatenation:
        \[
          \mathbf{Y} = \mathrm{Concat}_t \mathbf{Y}^{(t)}.
        \]
        No collective is needed on the forward path, but later layers may
        require an All-Gather or All-Reduce to reassemble or sum over
        shards.
  \item \textbf{Row-parallel (input-sharded) linear}: split $W$ along the
        input dimension and shard $\mathbf{X}$ accordingly. Each device
        holds $X^{(t)}$ and $W^{(t)}$ with
        $W^{(t)} \in \mathbb{R}^{D_{\text{in}}^{(t)} \times D_{\text{out}}}$,
        computes a partial product
        $\mathbf{Y}^{(t)} = \mathbf{X}^{(t)} W^{(t)}$, and an All-Reduce
        across $t$ produces the full output:
        \[
          \mathbf{Y} = \sum_{t=0}^{N_T-1} \mathbf{Y}^{(t)}.
        \]
\end{itemize}

These two patterns are composed inside the Transformer so that most
operations remain local to each device and only a small number of
All-Reduce/All-Gather calls are required per layer. An overview of this
scheme for a single Transformer block is shown in
Figure~\ref{fig:tp_overall_flow}, which should be compared to the
single-node overview in Figure~\ref{fig:single_node_overall}.

\begin{figure}[htbp]
  \centering
  \resizebox{\linewidth}{!}{%
\begin{tikzpicture}[
    node distance=2.5cm,
    >=stealth,
    block/.style={rectangle, draw=black, fill=white, text width=5em, text centered, rounded corners, minimum height=8em, font=\bfseries},
    blockstack/.style={rectangle, draw=black!60, fill=white, text width=5em, text centered, rounded corners, minimum height=8em},
    forward/.style={-{Stealth[length=2mm]}, thick, black},
    backward/.style={-{Stealth[length=2mm]}, thick, black, densely dashed},
    allreduce/.style={{Stealth[length=1.5mm]}-{Stealth[length=1.5mm]}, thick, red!70, densely dotted},
    io/.style={text centered, font=\bfseries}
]
    % Title
    % \node[font=\Large\bfseries] at (10, 6) {Transformer Overall Flow (TP with 3 GPUs)};

    % Forward path nodes (horizontal)
    \node (input) [io] {$\mathbf{X}$};
    \node (encoding) [block, right of=input, yshift=-3em] {Input\\Encoding};

    % MHA blocks (3 stacked) - GPU 2 (back)
    \node (mha3) [blockstack, right of=encoding, xshift=0.3cm, yshift=0.3cm] {};
    % MHA blocks (3 stacked) - GPU 1 (middle)
    \node (mha2) [blockstack, right of=encoding, xshift=0.15cm, yshift=0.15cm] {};
    % MHA blocks (3 stacked) - GPU 0 (front)
    \node (mha) [block, right of=encoding] {MHA};

    % Small GPU labels for MHA
    \node[font=\tiny, anchor=north east] at (mha.north east) {GPU 0};

    % FFN blocks (3 stacked) - GPU 2 (back)
    \node (mlp3) [blockstack, right of=mha, xshift=0.3cm, yshift=0.3cm] {};
    % FFN blocks (3 stacked) - GPU 1 (middle)
    \node (mlp2) [blockstack, right of=mha, xshift=0.15cm, yshift=0.15cm] {};
    % FFN blocks (3 stacked) - GPU 0 (front)
    \node (mlp) [block, right of=mha] {FFN};

    % Small GPU labels for FFN
    \node[font=\tiny, anchor=north east] at (mlp.north east) {GPU 0};

    \node (output) [block, right of=mlp] {Output\\Projection};
    \node (pred) [io, right of=output, yshift=3em] {$\mathbf{Y}$};
    \node (loss) [align=center, io, right of=output] {\small LOSS:\\$\mathcal{L}(\mathbf{Y,Y_\text{targets}})$};
    \node (gradient) [io, right of=output, yshift=-3em] {$\mathbf{dY}(=\frac{dL}{dY})$};

    % All-Reduce arrows (Backward drawn first, then blocks, then Forward on top)
    % Backward All-Reduce - FFN (left-bottom corner, lower position)
    \draw [allreduce] ([yshift=-0.2cm]mlp.south west) -- ([xshift=0.3cm, yshift=0.1cm]mlp.south west);
    \node[font=\tiny, red!70, anchor=north] at ([xshift=0.15cm, yshift=-0.25cm]mlp.south west) {all\_reduce(dX)};

    % Backward All-Reduce - MHA (left-bottom corner, lower position)
    \draw [allreduce] ([yshift=-0.2cm]mha.south west) -- ([xshift=0.3cm, yshift=0.1cm]mha.south west);
    \node[font=\tiny, red!70, anchor=north] at ([xshift=0.15cm, yshift=-0.25cm]mha.south west) {all\_reduce(dX)};

    % Redraw blocks to cover backward All-Reduce arrows
    \draw [draw=black!60, fill=white, rounded corners] (mha3.south west) rectangle (mha3.north east);
    \draw [draw=black!60, fill=white, rounded corners] (mha2.south west) rectangle (mha2.north east);
    \draw [draw=black, fill=white, rounded corners, line width=0.4pt] (mha.south west) rectangle (mha.north east);
    \node[font=\bfseries] at (mha.center) {MHA};
    \node[font=\tiny, anchor=north east] at (mha.north east) {GPU 0};

    \draw [draw=black!60, fill=white, rounded corners] (mlp3.south west) rectangle (mlp3.north east);
    \draw [draw=black!60, fill=white, rounded corners] (mlp2.south west) rectangle (mlp2.north east);
    \draw [draw=black, fill=white, rounded corners, line width=0.4pt] (mlp.south west) rectangle (mlp.north east);
    \node[font=\bfseries] at (mlp.center) {FFN};
    \node[font=\tiny, anchor=north east] at (mlp.north east) {GPU 0};

    % Forward All-Reduce - MHA (right-top corner) - drawn after blocks to appear on top
    \draw [allreduce] (mha.north east) -- ([xshift=0.3cm, yshift=0.3cm]mha.north east);
    \node[font=\tiny, red!70, anchor=south] at ([xshift=0.15cm, yshift=0.35cm]mha.north east) {all\_reduce(X)};

    % Forward All-Reduce - FFN (right-top corner) - drawn after blocks to appear on top
    \draw [allreduce] (mlp.north east) -- ([xshift=0.3cm, yshift=0.3cm]mlp.north east);
    \node[font=\tiny, red!70, anchor=south] at ([xshift=0.15cm, yshift=0.35cm]mlp.north east) {all\_reduce(X)};

    % Forward arrows (upper part of blocks)
    \draw [forward] (input) -- ([yshift=3em]encoding.west);
    \draw [forward] ([yshift=3em]encoding.east) -- ([yshift=3em]mha.west);
    \draw [forward] ([yshift=3em]mha.east) -- ([yshift=3em]mlp.west);
    \draw [forward] ([yshift=3em]mlp.east) -- ([yshift=3em]output.west);
    \draw [forward] ([yshift=3em]output.east) -- (pred);
    \draw [forward] (pred) -- (loss);
    \draw [backward] (loss) -- (gradient);

    % Backward arrows (lower part of blocks)
    \draw [backward] (gradient) -- ([yshift=-3em]output.east);
    \draw [backward] ([yshift=-3em]output.west) -- ([yshift=-3em]mlp.east);
    \draw [backward] ([yshift=-3em]mlp.west) -- ([yshift=-3em]mha.east);
    \draw [backward] ([yshift=-3em]mha.west) -- ([yshift=-3em]encoding.east);

    % Brace for layer repetition (horizontal - using mlp with x offset to match mlp3)
    \draw[decorate, decoration={brace, amplitude=10pt}]
        ([yshift=4.8em]mha.north west) -- ([xshift=0.3cm, yshift=4.8em]mlp.north east)
        node[midway, above=12pt, font=\normalsize] {$N$ layers};

    % Labels (Legend)
    \coordinate (legend) at ([xshift=-1.5cm, yshift=6.5em]input);
    \draw[forward] (legend) -- ++(0.8,0) node[right, font=\normalsize] {Forward};
    \draw[backward] ([yshift=-0.6cm]legend) -- ++(0.8,0) node[right, font=\normalsize] {Backward};
    \draw[allreduce] ([yshift=-1.2cm]legend) -- ++(0.8,0) node[right, font=\normalsize] {All-Reduce};

    % TP explanation
    \node[font=\scriptsize, align=left, text width=6cm] at ([xshift=2cm, yshift=-5.5em]encoding.south) {
        \textbf{Tensor Parallelism:}\\
        • Each GPU processes a shard of the weight matrix\\
        • All-Reduce synchronizes partial results\\
        • Forward: after row-parallel ops\\
        • Backward: after column-parallel ops
    };

\end{tikzpicture}%
}
  \caption{Overall Transformer layer with tensor parallelism. Each large
  linear layer from the single-node model is replaced by $N_T$ smaller
  matmuls on different devices. Colored arrows indicate where collective
  communication (e.g.\ All-Reduce, All-Gather) is required to assemble
  partial results, while local computations remain identical to those in
  Figure~\ref{fig:single_node_overall}.}
  \label{fig:tp_overall_flow}
\end{figure}

% ------------------------ 6.1 MHA with Tensor Parallelism -------------
\subsection{MHA with Tensor Parallelism}

We now apply tensor parallelism to the multi-head attention (MHA) block.
Recall from Section~\ref{sec:sn}.2 that the single-node attention block maps
$\mathbf{X} \in [B,S,D]$ to
$\mathbf{A}_{\text{out}} \in [B,S,D]$ via Q/K/V projections, scaled dot
product attention, an output projection, and a residual connection.

In tensor parallelism we partition these computations across $N_T$
devices in such a way that:
\begin{itemize}
  \item each device is responsible for a subset of the attention heads, or
        equivalently a subset of the output channels of the Q/K/V
        projections;
  \item the softmax and value-weighted sums are computed locally on each
        device for its own heads;
  \item the final output projection is implemented as a row-parallel
        linear, requiring an All-Reduce across devices to recover the
        same $\mathbf{A}_{\text{out}}$ as in the single-node case.
\end{itemize}

Figure~\ref{fig:mha_forward_tp} shows the forward data flow under this
sharding scheme and should be read as a “TP-annotated” version of
Figure~\ref{fig:single_node_mha_forward}. The corresponding backward
graph is shown in Figure~\ref{fig:mha_backward_tp}.

\subsubsection{Forward Pass}

For clarity we focus on one Transformer block and omit batch/sequence
dimensions in some formulas. We write the per-device Q/K/V projections as
column-parallel linears:
\[
  W_Q = [W_Q^{(0)},\dots,W_Q^{(N_T-1)}],\quad
  W_K = [W_K^{(0)},\dots,W_K^{(N_T-1)}],\quad
  W_V = [W_V^{(0)},\dots,W_V^{(N_T-1)}],
\]
where, on device $t$,
\[
  W_Q^{(t)}, W_K^{(t)}, W_V^{(t)}
    \in \mathbb{R}^{D \times D_h^{(t)}}, \qquad
  \sum_t D_h^{(t)} = D.
\]
Each device computes local projections:
\[
  \mathbf{Q}^{(t)} = \mathbf{X}_{\text{norm}} W_Q^{(t)},\quad
  \mathbf{K}^{(t)} = \mathbf{X}_{\text{norm}} W_K^{(t)},\quad
  \mathbf{V}^{(t)} = \mathbf{X}_{\text{norm}} W_V^{(t)},
\]
and reshapes them into its own subset of attention heads. Because heads
are independent, each device can compute the scaled dot-product attention
for its local heads without communication:
\[
  \mathbf{S}^{(t)} = \frac{\mathbf{Q}^{(t)} (\mathbf{K}^{(t)})^{\top}}
                           {\sqrt{D_h^{(t)}}},\qquad
  \mathbf{A}_S^{(t)} = \mathrm{Softmax}(\mathrm{Mask}(\mathbf{S}^{(t)})),
\]
\[
  \mathbf{A}_{\text{heads}}^{(t)} = \mathbf{A}_S^{(t)} \mathbf{V}^{(t)}.
\]

After computing local head outputs, each device forms a local
concatenation
$\mathbf{A}_{\text{cat}}^{(t)} \in \mathbb{R}^{B \times S \times D^{(t)}}$
over its subset of heads. The final output projection is implemented as a
\emph{row-parallel} linear:
\[
  W_O = \begin{bmatrix} W_O^{(0)} \\ \vdots \\ W_O^{(N_T-1)} \end{bmatrix},
  \quad W_O^{(t)} \in \mathbb{R}^{D^{(t)} \times D}.
\]
Each device computes a partial contribution
\[
  \mathbf{A}_{\text{lin}}^{(t)}
    = \mathbf{A}_{\text{cat}}^{(t)} W_O^{(t)} + \mathbf{b}_O^{(t)},
\]
and an All-Reduce across $t$ yields the full projected tensor
\[
  \mathbf{A}_{\text{lin}}
    = \sum_{t=0}^{N_T-1} \mathbf{A}_{\text{lin}}^{(t)}.
\]
Dropout and the residual connection with $\mathbf{X}$ are then applied
locally on each device, since after the All-Reduce every device holds the
same $\mathbf{A}_{\text{lin}}$ and $\mathbf{X}$.

These steps, together with the location of the All-Reduce, are annotated
explicitly in Figure~\ref{fig:mha_forward_tp}.

\begin{landscape}
\begin{figure}[p]
  \centering
  \resizebox{\linewidth}{!}{%
\begin{tikzpicture}[
  every node/.style={transform shape},
  >=stealth,
  auxnode/.style={draw, rectangle, fill=white, minimum height=6mm, inner sep=2pt, font=\footnotesize, align=center},
  mulnode/.style={draw, circle, fill=white, minimum size=6mm, font=\footnotesize, align=center},
  addnode/.style={draw, circle, fill=white, minimum size=6mm, font=\footnotesize, align=center},
  allreduce/.style={draw, rectangle, fill=red!30, minimum height=6mm, inner sep=2pt, font=\footnotesize, align=center, thick, draw=red!70},
  flow/.style={->, thick, black!85},
  flow2/.style={->, double, thick, black!85},
  commflow/.style={<->, thick, red!70, dashed},
  dimlabel/.style={font=\tiny, inner sep=0.5pt, align=center}
]
% \node[font=\Large\bfseries] at (9, 4.5) {Multi-Head Attention Forward Pass (Node $i$)};

\node (Input) at (0.5, 0) [align=center] {$\mathbf{X}$\\$[B,S,D]$};
\node[auxnode] (LN) [right=0.5cm of Input] {LN};

\node[mulnode] (Proj_Q) [right=2.0cm of LN, yshift=3.2cm] {$\bullet$};
\node[auxnode] (R_Q) [right=1.5cm of Proj_Q] {R};

\node[mulnode] (Proj_K) [right=2.0cm of LN, yshift=0cm] {$\bullet$};
\node[auxnode] (R_K) [right=1.5cm of Proj_K] {R};

\node[mulnode] (Proj_V) [right=2.0cm of LN, yshift=-3.2cm] {$\bullet$};
\node[auxnode] (R_V) [right=1.5cm of Proj_V] {R};

\node[dimlabel] (WQ) [align=center, below=1.0cm of Proj_Q] {$\widetilde{\mathbf{W}}_{Q}^{(i)}$\\$[D,N_{HN}D_h]$};
\node[dimlabel] (WK) [align=center, below=1.0cm of Proj_K] {$\widetilde{\mathbf{W}}_{K}^{(i)}$\\$[D,N_{HN}D_h]$};
\node[dimlabel] (WV) [align=center, below=1.0cm of Proj_V] {$\widetilde{\mathbf{W}}_{V}^{(i)}$\\$[D,N_{HN}D_h]$};

\node[auxnode] (T_K) [right=2.0cm of R_K] {T};
\node[mulnode] (QK) [right=3.2cm of R_Q, yshift=-1.6cm] {$\bullet$};
\node[auxnode] (SM) [right=2.0cm of QK] {SM};
\node[auxnode] (Soft) [right=2.0cm of SM] {S};
\node[mulnode] (PV) [right=2.4cm of Soft, yshift=-1.6cm] {$\bullet$};

\node[auxnode] (R_Merge) [right=2.0cm of PV] {R};
\node[auxnode] (Cat) [right=2.0cm of R_Merge] {C};

\node[mulnode] (OProj) [right=2.5cm of Cat] {$\bullet$};
\node[dimlabel] (WO_FWD) [align=center, below=1.0cm of OProj] {$\widetilde{\mathbf{W}}_{O}^{(i)}$\\$[N_{HN}D_h,D]$};
\node[allreduce] (AR) [right=2.0cm of OProj] {AR};
\node[
  align=center,
  below=2.5cm of AR,
  text width=5.5cm % 적당한 값으로 조정
] (AR_info) {%
  \textbf{All Reduce Comm.}:\\[2pt]
  \begin{itemize}
    \item \textbf{Naive:} $2(N_T-1) \times [B,S,D]$
    \item \textbf{Ring:} $2\frac{N_T-1}{N_T} \times [B,S,D]$
  \end{itemize}
};
\node[addnode] (AddB) [right=2.0cm of AR] {+};
\node[dimlabel] (BO) [align=center, below=1.0cm of AddB] {$\widetilde{\mathbf{b}}_{O}$\\$[D]$};
\node[auxnode] (Drop) [right=1.5cm of AddB] {DO};
\node (Aout) [align=center, right=0.5cm of Drop] {$\mathbf{A}_{\text{out}}$\\$[B,S,D]$};

\draw[flow] (Input) -- (LN);

\draw[flow] (LN.east) -- ++(0.3,0) |- (Proj_Q.west);
\draw[flow] (LN) -- (Proj_K.west) node[dimlabel, midway, above]{$\mathbf{X}_{\text{norm}}$\\$[B,S,D]$};
\draw[flow] (LN.east) -- ++(0.3,0) |- (Proj_V.west);

\draw[flow2] (WQ) -- (Proj_Q);
\draw[flow2] (WK) -- (Proj_K);
\draw[flow2] (WV) -- (Proj_V);

\draw[flow] (Proj_Q) -- (R_Q) node[dimlabel, midway, above]{$[B,S,N_{HN}D_h]$};
\draw[flow] (Proj_K) -- (R_K) node[dimlabel, midway, above]{$[B,S,N_{HN}D_h]$};
\draw[flow] (Proj_V) -- (R_V) node[dimlabel, midway, above]{$[B,S,N_{HN}D_h]$};

\draw[flow] (R_Q) -| (QK) node[dimlabel, near start, above]{$\mathbf{Q}_i$\\$[B,N_{HN},S,D_h]$};
\draw[flow] (R_K) -- (T_K) node[dimlabel, midway, above]{$\mathbf{K}_i$\\$[B,N_{HN},S,D_h]$};
\draw[flow2] (T_K) -| (QK) node[dimlabel, near end, right]{$\mathbf{K}_i^{T}$\\$[B,N_{HN},D_h,S]$};

\draw[flow] (QK) -- (SM) node[dimlabel, midway, above]{$\mathbf{A}_i$\\$[B,N_{HN},S,S]$};
\draw[flow] (SM) -- (Soft) node[dimlabel, midway, above]{$[B,N_{HN},S,S]$};
\draw[flow] (Soft) -| (PV) node[dimlabel, near start, above]{$\mathbf{AS}_i$\\$[B,N_{HN},S,S]$};
\draw[flow2] (R_V) -| (PV) node[dimlabel, pos=0.08, above]{$\mathbf{V}_i$\\$[B,N_{HN},S,D_h]$};

\draw[flow] (PV) -- (R_Merge) node[dimlabel, midway, above]{$\mathbf{AO}_{\text{h},i}$\\$[B,N_{HN},S,D_h]$};
\draw[flow] (R_Merge) -- (Cat) node[dimlabel, midway, above]{$[B,S,N_{HN},D_h]$};
\draw[flow] (Cat) -- (OProj) node[dimlabel, midway, above]{$\mathbf{AO}_{\text{c},i}$\\$[B,S,N_{HN}D_h]$};
\draw[flow2] (WO_FWD) -- (OProj);
\draw[flow] (OProj) -- (AR) node[dimlabel, midway, above]{$\mathbf{AO}_{\text{l},i}$\\$[B,S,D]$};

% All-Reduce communication arrows
\draw[commflow] (AR.north) -- ++(0, 2.0) node[midway, right, font=\tiny]{Node $j$};
\draw[commflow] (AR.south) -- ++(0, -2.4) node[midway, right, font=\tiny]{Node $k$};

\draw[flow] (AR) -- (AddB) node[dimlabel, midway, above]{$[B,S,D]$};
\draw[flow] (BO) -- (AddB);
\draw[flow] (AddB) -- (Drop) node[dimlabel, midway, above]{$[B,S,D]$};
\draw[flow] (Drop) -- (Aout);
\end{tikzpicture}%
}
  \caption{Multi-head attention forward pass with tensor parallelism. Q/K/V
  projections are implemented as column-parallel linears so that each
  device owns a subset of the heads. Attention for each local head is
  computed independently, and the resulting head outputs are combined
  using a row-parallel output projection followed by an All-Reduce to
  recover the same $\mathbf{A}_{\text{out}}$ as in the single-node
  computation.}
  \label{fig:mha_forward_tp}
\end{figure}
\end{landscape}

\subsubsection{Backward Pass}

The backward pass for tensor-parallel MHA follows the same high-level
structure as in Section~\ref{sec:sn}.2, but with sharded gradients and explicit
collectives. Starting from
$\mathrm{d}\mathbf{A}_{\text{out}}$ on each device, we have:

\begin{itemize}
  \item \textbf{Output projection backward.}
        The row-parallel output projection produces local gradients
        $\mathrm{d}\mathbf{A}_{\text{cat}}^{(t)}$ and parameter gradients
        $\mathrm{d}W_O^{(t)}, \mathrm{d}\mathbf{b}_O^{(t)}$ from
        $\mathrm{d}\mathbf{A}_{\text{lin}}$.
  \item \textbf{Head backward.}
        Each device backpropagates through its local heads, computing
        $\mathrm{d}\mathbf{V}^{(t)}$ and $\mathrm{d}\mathbf{A}_S^{(t)}$,
        and then $\mathrm{d}\mathbf{Q}^{(t)}$ and
        $\mathrm{d}\mathbf{K}^{(t)}$ through the scaled dot-product and
        softmax.
  \item \textbf{Q/K/V projection backward.}
        The column-parallel Q/K/V linears produce local gradients
        $\mathrm{d}W_Q^{(t)}, \mathrm{d}W_K^{(t)}, \mathrm{d}W_V^{(t)}$
        and partial contributions to the gradient w.r.t.\ the normalized
        input, $\mathrm{d}\mathbf{X}_{\text{norm}}^{(t)}$.
  \item \textbf{All-Reduce for input gradient.}
        Because $\mathbf{X}_{\text{norm}}$ is shared across devices,
        gradients from all Q/K/V shards must be summed:
        \[
          \mathrm{d}\mathbf{X}_{\text{norm}}
            = \sum_{t=0}^{N_T-1} \mathrm{d}\mathbf{X}_{\text{norm}}^{(t)},
        \]
        implemented as an All-Reduce over $t$.
  \item \textbf{Layer normalization backward.}
        Finally, layer normalization backward is applied locally on each
        device to produce $\mathrm{d}\mathbf{X}$ and gradients for the LN
        parameters, exactly as in the single-node setting.
\end{itemize}

Figure~\ref{fig:mha_backward_tp} expands these steps into a full
computation graph, with each All-Reduce/All-Gather explicitly indicated.

\begin{landscape}
\begin{figure}[p]
  % no \centering here to avoid compilation issues
  \par\vspace{1cm}
\noindent
\resizebox{\linewidth}{!}{%
\begin{tikzpicture}[
  every node/.style={transform shape},
  >=stealth,
  auxnode/.style={draw, rectangle, fill=white, minimum height=6mm, inner sep=2pt, font=\footnotesize, align=center},
  mulnode/.style={draw, circle, fill=white, minimum size=6mm, font=\footnotesize, align=center},
  addnode/.style={draw, circle, fill=white, minimum size=6mm, font=\footnotesize, align=center},
  sumnode/.style={draw, circle, fill=white, minimum size=6mm, font=\tiny, align=center},
  allreduce/.style={draw, rectangle, fill=red!30, minimum height=6mm, inner sep=2pt, font=\footnotesize, align=center, thick, draw=red!70},
  flow/.style={->, thick, black!85},
  flow2/.style={->, double, thick, black!85},
  commflow/.style={<->, thick, red!70, dashed},
  dimlabel/.style={font=\scriptsize, inner sep=1pt, align=center},
  gradflow/.style={->, thick, black!85},
  gradweight/.style={->, thick, black!85}
]

\begin{scope}[xscale=1.45, yscale=1.3]

\def\yoffset{-1.0}
\def\dVXyoffset{-6.5}

\coordinate (Grad_Aout_B) at (17.5, \yoffset);
\coordinate (dDrop_center) at (15.9, \yoffset);
\coordinate (ProjGradSplit) at (14.3, \yoffset);
\coordinate (dOProj_center) at (12.7, \yoffset);
\coordinate (C_center) at (11.1, \yoffset);
\coordinate (R_center) at (9.0, \yoffset);
\coordinate (dPV_AS_calc_center) at (7.5, \yoffset);
\coordinate (dSoft_center) at (5.3, \yoffset);
\coordinate (dSM_calc_center) at (3.3, \yoffset);
\coordinate (dV_calc_center) at (8.2, \dVXyoffset+\yoffset);
\coordinate (R_V_bwd_center) at (5.9, \dVXyoffset+\yoffset);
\coordinate (dVX_calc_center) at (3.9, \dVXyoffset+\yoffset);
\coordinate (dQK_calc_Q_center) at (1.7, \yoffset);
\coordinate (dQK_calc_K_center) at (1.7, -3.3+\yoffset);
\coordinate (K_BWD_input_center) at (1.7, 1.0+\yoffset);
\coordinate (T_Q_bwd_center) at (1.7, -4.0+\yoffset);

\node[font=\Large\bfseries] at (8, 4.6+\yoffset) {Multi-Head Attention Backward Pass (Node $i$)};

\node (Grad_Aout_B) at (18.5, \yoffset) {$\mathbf{dA}_{\text{out}}$\\$[B,S,D]$};
\node[auxnode] (DO) at (dDrop_center) {DO};
\node[mulnode] (dOProj) at (dOProj_center) {$\bullet$};

\node[auxnode] (T_WO) [below=1.0cm of dOProj] {T};
\node[dimlabel] (WO_BWD) [below=0.5cm of T_WO] {$\widetilde{\mathbf{W}}_{O}^{(i)}$\\$[N_{HN}D_h,D]$};

\node[mulnode] (dWO_calc) at ($(ProjGradSplit)+(0, 1.7)$) {$\bullet$};
\node[align=center, left=1.1cm of dWO_calc]
  (dWO_GRAD) {$\mathbf{d}\widetilde{\mathbf{W}}_{O}^{(i)}$\\$[N_{HN}D_h,D]$};
\node[auxnode] (T_AO_in) [right=1.5cm of dWO_calc] {T};
\node[dimlabel] (AO_in_local_label) [right=1.7cm of T_AO_in] {$\mathbf{AO}_{\text{cat},i}$\\$[B,S,$\\$N_{HN}D_h]$};

\node[auxnode] (C) at (C_center) {dC};
\node[auxnode] (R) at (R_center) {R};

\node[mulnode] (dPV_AS_calc) at (dPV_AS_calc_center) {$\bullet$};
\node[auxnode] (dSoft) at (dSoft_center) {dS};
\node[auxnode] (dSM_calc) at (dSM_calc_center) {dSM};

\node[mulnode] (dQK_calc_Q) at (dQK_calc_Q_center) {$\bullet$};
\node[mulnode] (dQX_proj_calc) [left=6.0cm of dQK_calc_Q] {$\bullet$};

\node[mulnode] (dQK_calc_K) at (dQK_calc_K_center) {$\bullet$};
\node[mulnode] (dKX_proj_calc) [left=6.0cm of dQK_calc_K] {$\bullet$};

\node[dimlabel] (K_BWD_input) at (K_BWD_input_center) {$\mathbf{K}_i$\\$[B,N_{HN},$\\$S,D_h]$};
\node[auxnode] (T_Q_bwd) at (T_Q_bwd_center) {T};
\node[dimlabel] (Q_BWD_input) [below=0.4cm of T_Q_bwd] {$\mathbf{Q}_i$\\$[B,N_{HN},$\\$S,D_h]$};

\node[dimlabel] (V_FWD) [above=1.8cm of dPV_AS_calc] {$\mathbf{V}_i$\\$[B,N_{HN},$\\$S,D_h]$};
\node[auxnode] (T_V_bwd) [below=0.3cm of V_FWD] {T};

\node[mulnode] (dV_calc) at (dV_calc_center) {$\bullet$};
\node[auxnode] (T_AS_bwd) [right=1.6cm of dV_calc] {T};
\node[dimlabel] (AS_BWD_for_V) [right=0.6cm of T_AS_bwd] {$\mathbf{AS}_i$\\$[B,N_{HN},$\\$S,S]$};

\node[auxnode] (R_V_bwd) at (R_V_bwd_center) {R};
\node[mulnode] (dVX_calc) at (dVX_calc_center) {$\bullet$};

\node[auxnode] (T_WV) [above=0.4cm of dVX_calc] {T};
\node[dimlabel] (WV_BWD) [above=0.35cm of T_WV] {$\widetilde{\mathbf{W}}_{V}^{(i)}$\\$[D,N_{HN}D_h]$};

\node[sumnode] (Sum_dBO) [below=1.6cm of ProjGradSplit] {$\sum_{B, S}$};
\node (dBO) [align=center, below=0.75cm of Sum_dBO] {$\mathbf{d}\widetilde{\mathbf{b}}_{O}$\\$[D]$};

\draw[gradflow] (Grad_Aout_B) -- (DO)
  node[dimlabel, midway, above]{$\mathbf{dA}_{\text{out}}$\\$[B,S,D]$};

\draw[gradflow] (DO) -- (dOProj)
  node[dimlabel, pos=0.25, above]{$\mathbf{dAO}_{\text{bias},i}$\\$=\mathbf{dAO}_{\text{lin},i}$\\$[B,S,D]$};

\draw[gradflow] (ProjGradSplit) -- (dWO_calc.south);
\draw[gradflow] (ProjGradSplit) -- ([yshift=-0.75cm]ProjGradSplit) -| (Sum_dBO.north);

\draw[gradflow] (dOProj) -- (C)
  node[dimlabel, midway, above]{$\mathbf{dAO}_{\text{cat},i}$\\$[B,S,$\\$N_{HN}D_h]$};
\draw[gradflow] (C) -- (R)
  node[dimlabel, midway, above]{$[B,S,$\\$N_{HN},D_h]$};

\coordinate (R_split_point) at ($(dPV_AS_calc)!0.5!(R)$);
\draw[gradflow] (R.west) -- (dPV_AS_calc.east);
\draw[flow2] (R_split_point) -- (dV_calc.north)
  node[dimlabel, midway, right]{$\mathbf{dAO}_{\text{heads},i}$\\$[B,N_{HN},$\\$S,D_h]$};

\draw[gradflow] (V_FWD.south) -- (T_V_bwd.north);
\draw[flow2] (T_V_bwd.south) -- (dPV_AS_calc.north)
  node[dimlabel, midway, right]{$[B,N_{HN},$\\$D_h,S]$};
\draw[gradflow] (dPV_AS_calc.west) -- (dSoft.east)
  node[dimlabel, midway, above]{$\mathbf{dAS}_i$\\$[B,N_{HN},$\\$S,S]$};

\node (AS_BWD_dS) [dimlabel, above=0.6cm of dSoft] {$\mathbf{AS}_i$\\$[B,N_{HN},$\\$S,S]$};
\draw[gradflow] (AS_BWD_dS.south) -- (dSoft.north);
\draw[gradflow] (dSoft.west) -- (dSM_calc.east)
  node[dimlabel, midway, above]{$[B,N_{HN},$\\$S,S]$};

\coordinate (dA_Split_X) at ($(dSM_calc_center)!0.5!(dQK_calc_Q_center)$);
\coordinate (dA_Split) at (dA_Split_X |- dQK_calc_Q.east);
\draw[gradflow] (dSM_calc.west) -- (dQK_calc_Q.east);
\draw[flow2] (dA_Split) -- (dA_Split |- dQK_calc_K.east) -- (dQK_calc_K.east)
  node[dimlabel, pos=-1.5, above, yshift=15]{$\mathbf{dA}_i$\\$[B,N_{HN},$\\$S,S]$};

\draw[flow2] (K_BWD_input.south) -- (dQK_calc_Q.north);
\draw[gradweight] (dQK_calc_Q) -- (dQX_proj_calc)
  node[dimlabel, midway, below]{$\mathbf{dQ}_i$\\$[B,N_{HN},$\\$S,D_h]$};

\node[auxnode] (T_WQ_bwd) [above=0.5cm of dQX_proj_calc] {T};
\node[dimlabel] (WQ_bwd) [above=0.5cm of T_WQ_bwd] {$\widetilde{\mathbf{W}}_{Q}^{(i)}$\\$[D,N_{HN}D_h]$};
\draw[flow] (WQ_bwd) -- (T_WQ_bwd);
\draw[flow2] (T_WQ_bwd.south) -- (dQX_proj_calc.north)
  node[dimlabel, midway, right]{$[D,$\\$N_{HN}D_h]$};

\draw[flow] (Q_BWD_input.north) -- (T_Q_bwd.south);
\draw[flow] (T_Q_bwd.north) -- (dQK_calc_K.south)
  node[dimlabel, pos=0.55, right]{$[B,N_{HN},$\\$D_h,S]$};

\node[auxnode] (T_dK) at ($(dQK_calc_K)!0.35!(dKX_proj_calc)$) {T};
\node[auxnode] (R_dK_mid) at ($(T_dK)!0.5!(dKX_proj_calc)$) {R};

\draw[gradweight] (dQK_calc_K) -- (T_dK)
  node[dimlabel, midway, above]{$\mathbf{dK}_i^T$\\$[B,N_{HN},$\\$D_h,S]$};
\draw[gradweight] (T_dK) -- (R_dK_mid)
  node[dimlabel, midway, above]{$[B,S,$\\$N_{HN},D_h]$};
\draw[gradweight] (R_dK_mid) -- (dKX_proj_calc)
  node[dimlabel, midway, above]{$\mathbf{dK}_i$\\$[B,S,$\\$N_{HN}D_h]$};

\node[auxnode] (T_WK_bwd) [above=0.45cm of dKX_proj_calc] {T};
\node[dimlabel] (WK_bwd) [above=0.45cm of T_WK_bwd] {$\widetilde{\mathbf{W}}_{K}^{(i)}$\\$[D,N_{HN}D_h]$};
\draw[gradflow] (WK_bwd) -- (T_WK_bwd);
\draw[flow2] (T_WK_bwd.south) -- (dKX_proj_calc.north);

\draw[gradflow] (AS_BWD_for_V.west) -- (T_AS_bwd.east);
\draw[gradflow] (T_AS_bwd.west) -- (dV_calc.east)
  node[dimlabel, midway, above]{$[B,N_{HN},$\\$S,S]$};
\draw[gradflow] (dV_calc.west) -- (R_V_bwd.east)
  node[dimlabel, midway, above]{$\mathbf{dV}_i$\\$[B,N_{HN},$\\$S,D_h]$};
\draw[gradflow] (R_V_bwd) -- (dVX_calc.east)
  node[dimlabel, midway, above]{$[B,S,$\\$N_{HN}D_h]$};

\draw[gradflow] (WV_BWD) -- (T_WV);
\draw[flow2] (T_WV) -- (dVX_calc.north)
  node[dimlabel, midway, left]{$[D,N_{HN}D_h]$};

\node[addnode] (Sum_dXnorm) [left=1.9cm of dKX_proj_calc] {$+$};

\draw[gradweight] (dQX_proj_calc.west) -| node[dimlabel, pos=0.7, left]{$\mathbf{dX}_{\text{norm},\mathbf{Q},i}$\\$[B,S,D]$} (Sum_dXnorm.north);
\draw[gradweight] (dKX_proj_calc.west) -- node[dimlabel, midway, above]{$\mathbf{dX}_{\text{norm},\mathbf{K},i}$\\$[B,S,D]$} (Sum_dXnorm.east);
\draw[gradweight] (dVX_calc.west) -| node[dimlabel, pos=0.9, left]{$\mathbf{dX}_{\text{norm},\mathbf{V},i}$\\$[B,S,D]$} (Sum_dXnorm.south);

\coordinate (dV_branch) at ($(R_V_bwd.west)!0.52!(dVX_calc.east)$);
\node[mulnode] (dWV_mul) at ($(dV_branch)+(0,-1.7cm)$) {$\bullet$};
\draw[flow2] (dV_branch) -- (dWV_mul.north);

\node[auxnode] (T_Xnorm) [right=1.2cm of dWV_mul] {T};
\node[dimlabel] (Xnorm_local) [right=0.9cm of T_Xnorm] {$\mathbf{X}_{\text{norm}}$\\$[B,S,D]$};
\draw[gradflow] (Xnorm_local) -- (T_Xnorm);
\draw[gradflow] (T_Xnorm.west) -- (dWV_mul.east)
  node[dimlabel, midway, above]{$[B,D,S]$};
\node (dWV_out) [align=center, left=1.1cm of dWV_mul] {$\mathbf{d}\widetilde{\mathbf{W}}_{V}^{(i)}$\\$[D,N_{HN}D_h]$};
\draw[gradweight] (dWV_mul.west) -- (dWV_out);

\coordinate (dQ_branch) at ($(dQK_calc_Q.east)!0.50!(dQX_proj_calc.west)$);
\node[mulnode] (dWQ_mul) at ($(dQ_branch)+(0,3.8cm)$) {$\bullet$};
\node[auxnode] (R_dQ_for_WQ) at ($(dWQ_mul)+(0,-1.0cm)$) {R};
\draw[gradflow]  (dQ_branch) -- (R_dQ_for_WQ.south);
\draw[flow2] (R_dQ_for_WQ.north) -- (dWQ_mul.south)
  node[dimlabel, midway, right]{$[B,S,$\\$N_{HN}D_h]$};

\node[auxnode] (T_XnormQ) [right=1.2cm of dWQ_mul] {T};
\node[dimlabel] (Xnorm_localQ) [right=0.6cm of T_XnormQ] {$\mathbf{X}_{\text{norm}}$\\$[B,S,D]$};
\draw[gradflow] (Xnorm_localQ) -- (T_XnormQ);
\draw[gradflow] (T_XnormQ.west) -- (dWQ_mul.east)
  node[dimlabel, midway, above]{$[B,D,S]$};
\node (dWQ_out) [align=center, left=1.1cm of dWQ_mul] {$\mathbf{d}\widetilde{\mathbf{W}}_{Q}^{(i)}$\\$[D,N_{HN}D_h]$};
\draw[gradweight] (dWQ_mul.west) -- (dWQ_out);

\coordinate (dK_branch) at ($(R_dK_mid)!0.52!(dKX_proj_calc)$);
\node[mulnode] (dWK_mul) at ($(dK_branch)+(0,-1.7cm)$) {$\bullet$};
\draw[flow2]  (dK_branch) -- (dWK_mul.north);

\node[auxnode] (T_XnormK) [right=1.3cm of dWK_mul] {T};
\node[dimlabel, right=0.6cm of T_XnormK] (Xnorm_localK) {$\mathbf{X}_{\text{norm}}$\\$[B,S,D]$};
\draw[gradflow] (Xnorm_localK) -- (T_XnormK);
\draw[gradflow] (T_XnormK.west) -- (dWK_mul.east)
  node[dimlabel, midway, above]{$[B,D,S]$};
\node (dWK_out) [align=center, left=1.1cm of dWK_mul] {$\mathbf{d}\widetilde{\mathbf{W}}_{K}^{(i)}$\\$[D,N_{HN}D_h]$};
\draw[gradweight] (dWK_mul.west) -- (dWK_out);

\draw[gradweight] (Sum_dBO) -- (dBO);

\draw[gradflow] (WO_BWD) -- (T_WO);
\draw[flow2] (T_WO) -- (dOProj)
  node[dimlabel, midway, right]{$\widetilde{\mathbf{W}}_{O}^{(i)T}$\\$[D,$\\$N_{HN}D_h]$};
\draw[gradflow] (AO_in_local_label) -- (T_AO_in);
\draw[flow2] (T_AO_in) -- (dWO_calc.east)
  node[dimlabel, midway, above]{$[B,$\\$N_{HN}D_h,S]$};
\draw[gradweight] (dWO_calc) -- (dWO_GRAD);

\node[auxnode] (dLN) [left=1.8cm of Sum_dXnorm] {dLN};
\draw[gradweight] (Sum_dXnorm.west) -- node[dimlabel, midway, above]
  {$\mathbf{dX}_{\text{norm},i}$\\$[B,S,D]$} (dLN.east);

\node[allreduce] (AR) [left=1.6cm of dLN] {AR};
\node[
  align=center,
  below=2.5cm of AR,
  text width=5.5cm
] (AR_info) {%
  \textbf{All Reduce Comm.}:\\[2pt]
  \begin{itemize}
    \item \textbf{Naive:} $2(N_T-1) \times [B,S,D]$
    \item \textbf{Ring:} $2\frac{N_T-1}{N_T} \times [B,S,D]$
  \end{itemize}
};

% All-Reduce communication arrows
\draw[commflow] (AR.north) -- ++(0, 2.0) node[midway, right, font=\tiny]{Node $j$};
\draw[commflow] (AR.south) -- ++(0, -2.4) node[midway, right, font=\tiny]{Node $k$};

\draw[gradweight] (dLN.west) -- (AR.east) node[dimlabel, midway, above]{$\mathbf{dX}_i$\\$[B,S,D]$};

\node (dX_OUT) [align=center, left=1.1cm of AR] {$\mathbf{dX}$\\$[B,S,D]$};
\draw[gradweight] (AR.west) -- (dX_OUT);

\node[dimlabel] (LNCache) [below=1.0cm of dLN] {$\mathbf{X}$\\$[B,S,D]$};
\draw[gradflow] (LNCache.north) -- (dLN.south);

\end{scope}
\end{tikzpicture}
}
  \caption{Multi-head attention backward pass with tensor parallelism.
  Each device backpropagates through its local Q/K/V projections and
  attention heads. Gradients with respect to the shared normalized input
  are summed across devices via All-Reduce, and parameter gradients are
  accumulated locally for each shard $W_Q^{(t)}, W_K^{(t)}, W_V^{(t)}$ and
  $W_O^{(t)}$.}
  \label{fig:mha_backward_tp}
\end{figure}
\end{landscape}

% ------------------------ 6.2 MLP with Tensor Parallelism -------------
\subsection{MLP with Tensor Parallelism}

The feed-forward (MLP) block is particularly well-suited to tensor
parallelism because its two linear layers can be implemented as a
column-parallel / row-parallel pair. Recall from Section~\ref{sec:sn}.3 that the
single-node MLP maps
$\mathbf{H} \in [B,S,D]$ to $\mathbf{Y} \in [B,S,D]$ via
\[
  \mathbf{Z}_{\text{up}} = \mathbf{H} W_{\text{up}} + \mathbf{b}_{\text{up}},
  \quad \mathbf{U} = \phi(\mathbf{Z}_{\text{up}}),
\]
\[
  \mathbf{Z}_{\text{down}} = \mathbf{U} W_{\text{down}} + \mathbf{b}_{\text{down}},
  \quad \mathbf{Y} = \mathbf{H} + \mathrm{Dropout}(\mathbf{Z}_{\text{down}}),
\]
where $W_{\text{up}} \in \mathbb{R}^{D \times D_{\text{ff}}}$ and
$W_{\text{down}} \in \mathbb{R}^{D_{\text{ff}} \times D}$.

In tensor parallelism we choose:
\begin{itemize}
  \item \textbf{Up-projection}: column-parallel (output-sharded).
  \item \textbf{Down-projection}: row-parallel (input-sharded).
\end{itemize}
This combination ensures that only one collective is required in the
forward pass and one in the backward pass, while keeping most of the
activation tensors local.

Figure~\ref{fig:mlp_forward_tp} shows the resulting forward data flow, and
Figure~\ref{fig:mlp_backward_tp} shows the corresponding backward
computation.

\subsubsection{Forward Pass}

We split the up-projection weight along its output dimension:
\[
  W_{\text{up}} = [W_{\text{up}}^{(0)},\dots,W_{\text{up}}^{(N_T-1)}],\quad
  W_{\text{up}}^{(t)} \in \mathbb{R}^{D \times D_{\text{ff}}^{(t)}},
  \qquad \sum_t D_{\text{ff}}^{(t)} = D_{\text{ff}}.
\]
Each device computes a local up-projection:
\[
  \mathbf{Z}_{\text{up}}^{(t)}
    = \mathbf{H} W_{\text{up}}^{(t)} + \mathbf{b}_{\text{up}}^{(t)},
  \qquad
  \mathbf{U}^{(t)} = \phi(\mathbf{Z}_{\text{up}}^{(t)}),
\]
where $\phi$ is the non-linearity (e.g., GELU). No communication is
required at this stage, because each device only needs its own local slice
$\mathbf{U}^{(t)}$ to proceed.

Next, we implement the down-projection as a row-parallel linear:
\[
  W_{\text{down}}
    = \begin{bmatrix} W_{\text{down}}^{(0)} \\ \vdots \\ W_{\text{down}}^{(N_T-1)} \end{bmatrix},
  \quad W_{\text{down}}^{(t)} \in \mathbb{R}^{D_{\text{ff}}^{(t)} \times D}.
\]
Each device computes a partial contribution
\[
  \mathbf{Z}_{\text{down}}^{(t)}
    = \mathbf{U}^{(t)} W_{\text{down}}^{(t)} + \mathbf{b}_{\text{down}}^{(t)},
\]
and an All-Reduce across $t$ yields the full down-projection output:
\[
  \mathbf{Z}_{\text{down}}
    = \sum_{t=0}^{N_T-1} \mathbf{Z}_{\text{down}}^{(t)}.
\]
Dropout and the residual connection with $\mathbf{H}$ are then applied
locally, as every device has the same $\mathbf{Z}_{\text{down}}$ and
$\mathbf{H}$ after the All-Reduce.

These steps are summarized in Figure~\ref{fig:mlp_forward_tp}, where the
only collective in the forward path is the All-Reduce after the
down-projection.

\begin{figure}[htbp]
  \centering
  \resizebox{\linewidth}{!}{%
\begin{tikzpicture}[
    >=stealth,
    auxnode/.style={draw, rectangle, fill=white, minimum height=6mm, inner sep=2pt, font=\footnotesize, align=center},
    mulnode/.style={draw, circle, fill=white, minimum size=6mm, font=\footnotesize, align=center},
    addnode/.style={draw, circle, fill=white, minimum size=6mm, font=\footnotesize, align=center},
    allreduce/.style={draw, rectangle, fill=red!30, minimum height=6mm, inner sep=2pt, font=\footnotesize, align=center, thick, draw=red!70},
    sumnode/.style={draw, circle, fill=white, minimum size=6mm, font=\tiny, align=center},
    flow/.style={->, thick, black!85},
    flow2/.style={double, ->, thick, black!85},
    commflow/.style={<->, thick, red!70, dashed},
    dimlabel/.style={font=\tiny, inner sep=1pt, align=center}
]
    \node[font=\Large\bfseries] at (11, 2.8) {MLP Forward Pass (Node $i$)};

    \pgfmathsetmacro{\verticaloffset}{-0.5}

    \node            (MIn)   at (0,\verticaloffset) {$\mathbf{X}$};
    \node[auxnode]   (LN2)   [right=1.6cm of MIn] {LN};
    \node[mulnode]   (L1Mul) [right=2.0cm of LN2] {$\bullet$};
    \node[dimlabel]  (Wup)   [below=1.0cm of L1Mul] {$\tilde{\mathbf{W}}_{\text{up}}^{(i)}$\\$[D, D_{ff}/N_T]$\\(col-split)};
    \node[addnode]   (AddB1) [right=1.8cm of L1Mul] {+};
    \node[dimlabel]  (Bup)   [below=1.0cm of AddB1] {$\tilde{\mathbf{b}}_{\text{up}}^{(i)}$\\$[D_{ff}/N_T]$};
    \node[auxnode]   (Act)   [right=1.8cm of AddB1] {GL};
    \node[auxnode]   (Drop1) [right=1.8cm of Act] {DO};
    \node[mulnode]   (L2Mul) [right=2.0cm of Drop1] {$\bullet$};
    \node[dimlabel]  (Wdown) [below=1.0cm of L2Mul] {$\tilde{\mathbf{W}}_{\text{down}}^{(i)}$\\$[D_{ff}/N_T, D]$\\(row-split)};
    \node[allreduce] (AR)    [right=1.8cm of L2Mul] {AR};
    \node[
      align=center,
      below=2.5cm of AR,
      text width=5.2cm
    ] (AR_info) {%
      \textbf{All Reduce Comm.}:\\[2pt]
      \begin{itemize}
        \item \textbf{Naive:} $2(N_T-1) \times [B,S,D]$
        \item \textbf{Ring:} $2\frac{N_T-1}{N_T} \times [B,S,D]$
      \end{itemize}
    };
    \node[addnode]   (AddB2) [right=1.8cm of AR] {+};
    \node[dimlabel]  (Bdown) [below=1.0cm of AddB2] {$\tilde{\mathbf{b}}_{\text{down}}$\\$[D]$};
    \node[auxnode]   (Drop2) [right=1.6cm of AddB2] {DO};
    \node            (MOut)  [right=1.6cm of Drop2] {$\mathbf{Y}$};

    % All-Reduce communication arrows
    \draw[commflow] (AR.north) -- ++(0, 1.8) node[midway, right, font=\tiny]{Node $j$};
    \draw[commflow] (AR.south) -- ++(0, -2.1) node[midway, right, font=\tiny]{Node $k$};

    \draw[flow] (MIn) -- (LN2) node[dimlabel, midway, above]{\shortstack{$\mathbf{X}$\\$[B,S,D]$}};
    \draw[flow] (LN2) -- (L1Mul) node[dimlabel, midway, above]{\shortstack{$\mathbf{H}$\\$[B,S,D]$}};
    \draw[flow2] (Wup) -- (L1Mul);
    \draw[flow] (L1Mul) -- (AddB1) node[dimlabel, midway, above]{\shortstack{$\mathbf{Z}_{\text{up},i}$\\$[B,S,D_{ff}/N_T]$}};
    \draw[flow] (Bup) -- (AddB1);
    \draw[flow] (AddB1) -- (Act) node[dimlabel, midway, above]{\shortstack{$\mathbf{A}_{\text{up},i}$\\$[B,S,D_{ff}/N_T]$}};
    \draw[flow] (Act) -- (Drop1) node[dimlabel, midway, above]{\shortstack{$\mathbf{H}_{\text{inter},i}$\\$[B,S,D_{ff}/N_T]$}};
    \draw[flow] (Drop1) -- (L2Mul) node[dimlabel, midway, above]{\shortstack{$\mathbf{U}_{\text{in},i}$\\$[B,S,D_{ff}/N_T]$}};
    \draw[flow2] (Wdown) -- (L2Mul);
    \draw[flow] (L2Mul) -- (AR) node[dimlabel, midway, above]{\shortstack{$\mathbf{Z}_{\text{down},i}$\\$[B,S,D]$}};
    \draw[flow] (AR) -- (AddB2) node[dimlabel, midway, above]{\shortstack{$[B,S,D]$}};
    \draw[flow] (Bdown) -- (AddB2);
    \draw[flow] (AddB2) -- (Drop2) node[dimlabel, midway, above]{\shortstack{$\mathbf{A}_{\text{out}}$\\$[B,S,D]$}};
    \draw[flow] (Drop2) -- (MOut) node[dimlabel, midway, above]{\shortstack{$\mathbf{Y}$\\$[B,S,D]$}};

\end{tikzpicture}%
}
  \caption{MLP forward pass with tensor parallelism. The up-projection is
  implemented as a column-parallel linear so that each device holds a
  subset of the intermediate features. The down-projection is
  row-parallel, and an All-Reduce over devices reconstructs the full
  $\mathbf{Z}_{\text{down}}$, after which dropout and the residual
  connection are applied locally.}
  \label{fig:mlp_forward_tp}
\end{figure}

\subsubsection{Backward Pass}

The backward pass for the tensor-parallel MLP reuses the same structure as
the single-node backward graph (Figure~\ref{fig:single_node_mlp_backward}),
but with sharded parameters and explicit All-Reduce operations.

Starting from $\mathrm{d}\mathbf{Y}$ on each device:

\begin{enumerate}
  \item \textbf{Residual and dropout backward}: as before, gradients are
        split between the identity path and the path through the final
        dropout, yielding $\mathrm{d}\mathbf{Z}_{\text{down}}$ on each
        device.
  \item \textbf{Down-projection backward (row-parallel)}: each device
        computes local gradients
        $\mathrm{d}\mathbf{U}^{(t)}$ and parameter gradients
        $\mathrm{d}W_{\text{down}}^{(t)}, \mathrm{d}\mathbf{b}_{\text{down}}^{(t)}$
        using its own shard $W_{\text{down}}^{(t)}$ and local activations
        $\mathbf{U}^{(t)}$. No communication is needed to form
        $\mathrm{d}\mathbf{U}^{(t)}$.
  \item \textbf{Activation backward}: the non-linearity $\phi$ is
        inverted elementwise on each device to obtain
        $\mathrm{d}\mathbf{Z}_{\text{up}}^{(t)}$.
  \item \textbf{Up-projection backward (column-parallel)}:
        for the column-parallel up-projection, each device computes local
        contributions to the gradient with respect to $W_{\text{up}}^{(t)}$
        and a partial gradient w.r.t.\ the input $\mathrm{d}\mathbf{H}^{(t)}$.
        Because $\mathbf{H}$ is shared across devices, we must sum these
        partial gradients:
        \[
          \mathrm{d}\mathbf{H}
            = \sum_{t=0}^{N_T-1} \mathrm{d}\mathbf{H}^{(t)},
        \]
        implemented as an All-Reduce over $t$.
  \item \textbf{Layer normalization backward (if present)}: if the MLP is
        preceded by a layer normalization (as in a pre-LN Transformer),
        its backward pass is applied locally using the aggregated
        $\mathrm{d}\mathbf{H}$.
\end{enumerate}

Figure~\ref{fig:mlp_backward_tp} shows these steps, highlighting the
All-Reduce on $\mathrm{d}\mathbf{H}$ (the only collective required in the
MLP backward pass).

\begin{figure}[htbp]
  % no \centering here to avoid compilation issues
  \resizebox{\linewidth}{!}{%
\begin{tikzpicture}[
    >=stealth,
    auxnode/.style={draw, rectangle, fill=white, minimum height=6mm, inner sep=2pt, font=\footnotesize, align=center},
    mulnode/.style={draw, circle, fill=white, minimum size=6mm, font=\footnotesize, align=center},
    addnode/.style={draw, circle, fill=white, minimum size=6mm, font=\footnotesize, align=center},
    sumnode/.style={draw, circle, fill=white, minimum size=6mm, font=\tiny, align=center},
    allreduce/.style={draw, rectangle, fill=red!30, minimum height=6mm, inner sep=2pt, font=\footnotesize, align=center, thick, draw=red!70},
    flow_rev/.style={<-, thick, black!85},
    flow_dw/.style={->, thick, black!85},
    flow_act/.style={double, ->, thick, black!85},
    commflow/.style={<->, thick, red!70, dashed},
    dimlabel/.style={font=\tiny, inner sep=1pt, align=center},
    gradlabel/.style={font=\tiny\bfseries, inner sep=1pt, align=center}
]
    % \node[font=\Large\bfseries] at (5, 10) {MLP Backward Pass (Node $i$)};

    \pgfmathsetmacro{\backwardoffset}{0.0}

    \node (d_MOut) at (12.6, \backwardoffset) {$\mathrm{d}\mathbf{Y}$};
    \node[auxnode] (d_Drop2) [left=1.8cm of d_MOut] {dDO};
    \draw[flow_rev] (d_Drop2) -- (d_MOut)
      node[dimlabel, midway, below]{\shortstack{$\mathrm{d}\mathbf{Y}$\\$[B,S,D]$}};

    \coordinate (split2) at ($(d_Drop2.west) + (-1.5cm, 0)$);
    \coordinate (branch_dUproj) at ($(split2) + (-1.2cm, 0)$);

    \node[sumnode] (d_SumB2) [above=0.8cm of split2] {$\sum_{B, S}$};
    \node (d_Bdown) [above=0.8cm of d_SumB2] {$\mathrm{d}\tilde{\mathbf{b}}_{\text{down}}$};
    \draw[flow_dw] (d_SumB2) -- (d_Bdown) node[dimlabel, midway, right]{$[D]$};

    \draw[flow_rev] (d_SumB2) -- (split2);

    \node[mulnode] (d_L2Mul_in) [left=2.2cm of split2] {$\bullet$};
    \draw[flow_rev] (d_L2Mul_in) -- (d_Drop2)
      node[gradlabel, midway, below]{\shortstack{$\mathrm{d}\mathbf{Z}_{\text{down}}$\\$=\mathrm{d}\mathbf{A}_{\text{out}}$\\$[B,S,D]$}};

    \node[dimlabel] (W_down_T) [align=center, below=1.0cm of d_L2Mul_in] {$\tilde{\mathbf{W}}_{\text{down}}^{(i)T}$\\$[D, D_{ff}/N_T]$\\(row-split)};
    \draw[flow_act] (W_down_T.north) -- (d_L2Mul_in);

    \coordinate (L2Mul_w_y) at ($(d_L2Mul_in) + (0, 2.5cm)$);
    \node[mulnode] (d_L2Mul_w) at (L2Mul_w_y) {$\bullet$};
    \node (d_Wdown) [align=center, above=0.9cm of d_L2Mul_w] {$\mathrm{d}\tilde{\mathbf{W}}_{\text{down}}^{(i)}$\\$[D_{ff}/N_T, D]$};
    \draw[flow_dw] (d_L2Mul_w) -- (d_Wdown);

    \draw[flow_act] (branch_dUproj.north) |- (d_L2Mul_w.east);

    \node[auxnode] (Uin_T) at ($(d_L2Mul_w.west) + (-1.5cm, 0)$) {T};
    \draw[flow_dw] (Uin_T) -- (d_L2Mul_w)
      node[dimlabel, midway, below]{\shortstack{$\mathbf{U}_{\text{in},i}^T$\\$[B, D_{ff}/N_T, S]$}};
    \node (Uin_aux) [left=1.8cm of Uin_T] {$\mathbf{U}_{\text{in},i}$};
    \draw[flow_dw] (Uin_aux) -- (Uin_T) node[dimlabel, midway, above]{\shortstack{$[B,S,$$D_{ff}/N_T]$}};

    \node[auxnode] (d_Drop1) [left=1.8cm of d_L2Mul_in] {dDO};
    \draw[flow_rev] (d_Drop1) -- (d_L2Mul_in)
      node[dimlabel, midway, below]{\shortstack{$\mathrm{d}\mathbf{U}_{\text{in},i}$\\$[B,S,$$D_{ff}/N_T]$}};

    \node[auxnode] (d_Act) [left=1.8cm of d_Drop1] {dGL};
    \draw[flow_rev] (d_Act) -- (d_Drop1)
      node[dimlabel, midway, below]{\shortstack{$\mathrm{d}\mathbf{H}_{\text{inter},i}$\\$[B,S,$$D_{ff}/N_T]$}};

    \coordinate (split1) at ($(d_Act.west) + (-1.5cm, 0)$);
    \coordinate (branch_dHpre) at ($(split1) + (-1.2cm, 0)$);

    \node[sumnode] (d_SumB1) [above=0.8cm of split1] {$\sum_{B, S}$};
    \node[dimlabel] (d_Bup) [above=0.9cm of d_SumB1] {$\mathrm{d}\tilde{\mathbf{b}}_{\text{up}}^{(i)}$\\$[D_{ff}/N_T]$};
    \draw[flow_dw] (d_SumB1) -- (d_Bup);

    \draw[flow_rev] (d_SumB1) -- (split1);

    \node[mulnode] (d_L1Mul_in) [left=2.2cm of split1] {$\bullet$};
    \draw[flow_rev] (d_L1Mul_in) -- (d_Act)
      node[gradlabel, midway, below]{\shortstack{$\mathrm{d}\mathbf{Z}_{\text{up},i}$\\$=\mathrm{d}\mathbf{A}_{\text{up},i}$\\$[B,S,D_{ff}/N_T]$}};

    \node[dimlabel] (W_up_T) [align=center, below=1.0cm of d_L1Mul_in] {$\tilde{\mathbf{W}}_{\text{up}}^{(i)T}$\\$[D_{ff}/N_T, D]$\\(col-split)};
    \draw[flow_act] (W_up_T.north) -- (d_L1Mul_in);

    \coordinate (L1Mul_w_y) at ($(d_L1Mul_in) + (0, 2.5cm)$);
    \node[mulnode] (d_L1Mul_w) at (L1Mul_w_y) {$\bullet$};
    \node (d_Wup) [align=center, above=0.9cm of d_L1Mul_w] {$\mathrm{d}\tilde{\mathbf{W}}_{\text{up}}^{(i)}$\\$[D, D_{ff}/N_T]$};
    \draw[flow_dw] (d_L1Mul_w) -- (d_Wup);

    \draw[flow_act] (branch_dHpre.north) |- (d_L1Mul_w.east);

    \node[auxnode] (Znorm_T) at ($(d_L1Mul_w.west) + (-1.5cm, 0)$) {T};
    \draw[flow_dw] (Znorm_T) -- (d_L1Mul_w)
      node[dimlabel, midway, below]{\shortstack{$\mathbf{H}^T$\\$[B, D, S]$}};
    \node (Znorm_aux) [left=1.8cm of Znorm_T] {$\mathbf{H}$};
    \draw[flow_dw] (Znorm_aux) -- (Znorm_T) node[dimlabel, midway, above]{\shortstack{$[B,S,D]$}};

    \node[allreduce] (AR) [left=1.8cm of d_L1Mul_in] {AR};
    \node[
      align=center,
      below=2.0cm of AR,
      text width=5.2cm
    ] (AR_info) {%
      \textbf{All Reduce Comm.}:\\[2pt]
      \begin{itemize}
        \item \textbf{Naive:} $2(N_T-1) \times [B,S,D]$
        \item \textbf{Ring:} $2\frac{N_T-1}{N_T} \times [B,S,D]$
      \end{itemize}
    };

    % All-Reduce communication arrows
    \draw[commflow] (AR.north) -- ++(0, 1.8) node[midway, right, font=\tiny]{Node $j$};
    \draw[commflow] (AR.south) -- ++(0, -2.1) node[midway, right, font=\tiny]{Node $k$};

    \draw[flow_rev] (AR) -- (d_L1Mul_in)
      node[dimlabel, midway, below]{\shortstack{$\mathrm{d}\mathbf{H}_i$\\$[B,S,D]$}};

    \node[auxnode] (d_LN2) [left=1.8cm of AR] {dLN};
    \draw[flow_rev] (d_LN2) -- (AR)
      node[dimlabel, midway, below]{\shortstack{$\mathrm{d}\mathbf{H}$\\$[B,S,D]$}};

    \node (d_MIn) [left=1.8cm of d_LN2] {$\mathrm{d}\mathbf{X}$};
    \draw[flow_rev] (d_MIn) -- (d_LN2)
      node[dimlabel, midway, below]{\shortstack{$\mathrm{d}\mathbf{X}$\\$[B,S,D]$}};
\end{tikzpicture}%
}
  \caption{MLP backward pass with tensor parallelism. Each device
  backpropagates through its local up- and down-projection shards. The
  only collective in the backward path is an All-Reduce that sums the
  partial input gradients $\mathrm{d}\mathbf{H}^{(t)}$ across devices to
  obtain the full $\mathrm{d}\mathbf{H}$. Parameter gradients remain
  local to each shard.}
  \label{fig:mlp_backward_tp}
\end{figure}


% ------------------------ 6.3 Communication Summary ------------------
\subsection{Communication Patterns and Costs}

Finally, we summarize where collective communication appears in the
tensor-parallel Transformer layer:

\begin{itemize}
  \item \textbf{MHA forward:}
        an All-Reduce after the row-parallel output projection to obtain
        $\mathbf{A}_{\text{lin}}$ on every device
        (Figure~\ref{fig:mha_forward_tp}).
  \item \textbf{MHA backward:}
        an All-Reduce on $\mathrm{d}\mathbf{X}_{\text{norm}}$ to sum
        gradient contributions from all Q/K/V shards
        (Figure~\ref{fig:mha_backward_tp}).
  \item \textbf{MLP forward:}
        an All-Reduce after the row-parallel down-projection to form
        $\mathbf{Z}_{\text{down}}$
        (Figure~\ref{fig:mlp_forward_tp}).
  \item \textbf{MLP backward:}
        an All-Reduce on the input gradient $\mathrm{d}\mathbf{H}$ from
        the column-parallel up-projection
        (Figure~\ref{fig:mlp_backward_tp}).
\end{itemize}

In all cases, the local computation on each device is structurally
identical to the single-node computation described in Section~\ref{sec:sn}; only the
placement of collectives and the sharding of weights and activations
differ. This makes tensor parallelism a natural extension of the
single-node Transformer, well suited for large models that do not fit into
the memory of a single accelerator.
