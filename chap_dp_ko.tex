% ==========================================================
% 7. 데이터 병렬화
% ==========================================================
\section{데이터 병렬화}
\label{sec:dp}

데이터 병렬화(data parallelism, DP)에서는 각 복제본(replica)이
모델 전체를 그대로 한 벌씩 가지고 있지만, 서로 다른 배치 조각을
처리한다. 개념적으로는 각 디바이스가 단일 노드 설정
(Section~\ref{sec:sn})과 \emph{동일한} 순전파·역전파 계산을 수행하되,
입력 미니배치만 서로 다른 셈이다. 역전파가 끝난 뒤에는
All-Reduce를 통해 기울기를 복제본들 사이에서 동기화하여,
전역 배치를 한 번에 처리하는 단일 노드 학습과
\emph{동일한} 옵티마이저 업데이트가 이루어지도록 한다.

단일 복제본의 관점에서 보면, 각 트랜스포머 블록 안의 순전파 계산 그래프는
단일 노드의 그래프(Section~\ref{sec:sn})와 완전히 같고,
텐서 병렬화(Section~\ref{sec:tp})까지 함께 사용하는 경우에도
각 복제본은 동일한 TP가 적용된 그래프를 자신의 미니배치에 대해 실행한다.
달라지는 것은 입력 배치뿐이다.
글로벌 배치
\[
  \mathbf{X} \in \mathbb{R}^{B \times S \times D}
\]
대신, 복제본 $d$는
\[
  \mathbf{X}_d \in \mathbb{R}^{B_{\text{local}} \times S \times D}
\]
라는 로컬 배치를 보게 되며, $B_{\text{local}} = B / N_D$이다.

데이터 병렬 복제본의 개수를 $N_D$라 두고,
디바이스 인덱스는 $d \in \{0,\dots,N_D-1\}$로 표기한다.

\textbf{핵심 아이디어는 다음과 같다.}
\begin{itemize}
  \item 각 디바이스는 모델 파라미터(가중치와 옵티마이저 상태)를
        \emph{온전히} 한 벌씩 가진다.
  \item 전역 배치 크기 $B$는 $N_D$개의 로컬 배치로 나뉘며,
        각 로컬 배치의 크기는 $B_{\text{local}} = B / N_D$이다.
  \item 각 디바이스의 순전파는 Section~\ref{sec:sn}의 단일 노드 계산과 동일하지만,
        자신의 로컬 배치만 사용한다.
  \item 역전파는 각 디바이스에서 로컬 기울기
        $\nabla W^{(d)}$를 계산한다.
  \item 모든 데이터 병렬 복제본에 걸쳐 All-Reduce를 수행해
        기울기를 평균(또는 합산)함으로써,
        배치 크기 $B$인 단일 노드 실행과 동등한 업데이트를 만든다.
\end{itemize}

텐서 병렬화(Section~\ref{sec:tp})와 달리,
데이터 병렬화는 가중치 행렬이나 활성값을 쪼개지 않는다.
대신 모델 전체는 복제하고, \emph{데이터만} 분할한다.

% ------------------------ 7.1 Overall DP Flow -------------------------
\subsection{데이터 병렬 학습 전체 흐름}

Figure~\ref{fig:dp_overall_flow}는 트랜스포머 레이어에
데이터 병렬화를 적용했을 때의 상위 수준 구조를 보여준다.
단일 노드 개요 그림(Figure~\ref{fig:single_node_overall})과 비교해 보면,
이제 $N_D$개의 서로 동일한 블록 복제본이 있고,
각 복제본이 서로 다른 입력 조각 $\mathbf{X}_d$를 처리하여
로컬 예측과 로컬 손실을 만든다는 점만 다르다.

하나의 학습 스텝에서 수행되는 순서는 다음과 같다.

\begin{enumerate}
  \item \textbf{배치 분할 (batch sharding).}
        전역 배치 크기 $B$를 $N_D$개의 로컬 배치로 나눈다.
        각 로컬 배치의 크기는 $B_{\text{local}}$이다:
        \[
          \{\mathbf{X}_0,\dots,\mathbf{X}_{N_D-1}\},
          \qquad
          \mathbf{X}_d \in \mathbb{R}^{B_{\text{local}} \times S \times D_{\text{in}}}.
        \]
        각 디바이스 $d$는 이에 대응하는 타깃 토큰
        $\mathbf{Y}_d$도 함께 받는다.
  \item \textbf{로컬 순전파.}
        각 디바이스는 자신의 로컬 입력 $\mathbf{X}_d$에 대해
        전체 트랜스포머 스택(입력 임베딩, MHA, MLP, 출력 프로젝션)을
        Section~\ref{sec:sn}에서와 동일하게 적용한다.
        텐서 병렬화(Section~\ref{sec:tp})도 함께 사용하는 경우,
        각 복제본은 $\mathbf{X}_d$에 대해 동일한 TP-적용 그래프를 실행한다.
        어떤 경우든 순전파 구조 자체는 변하지 않고,
        달라지는 것은 배치 차원과 기울기 동기화의 존재 여부뿐이다.
        각 디바이스는 로컬 로그릿, 확률, 손실 $\mathcal{L}_d$를 만든다.
  \item \textbf{로컬 역전파.}
        각 디바이스에서 손실 $\mathcal{L}_d$에 대한 역전파를 수행하여,
        디바이스 $d$에 있는 모든 파라미터에 대한 로컬 기울기
        $\nabla W^{(d)}$를 얻는다.
  \item \textbf{기울기 동기화 (All-Reduce).}
        각 파라미터 텐서 $W$에 대해 데이터 병렬 그룹 전체에 대해
        All-Reduce를 수행한다:
        \[
          \nabla W
            = \frac{1}{N_D}
              \sum_{d=0}^{N_D-1} \nabla W^{(d)}.
        \]
        이 단계 이후에는 모든 복제본이 동일한 평균 기울기
        $\nabla W$를 갖게 된다.
  \item \textbf{옵티마이저 업데이트.}
        각 디바이스는 자신의 로컬 파라미터 복사본에
        동일한 옵티마이저(SGD, Adam 등)를 적용하여 업데이트한다.
        기울기가 동기화되어 있기 때문에,
        모든 복제본의 파라미터는 다시 일치하게 된다.
\end{enumerate}

Figure~\ref{fig:dp_overall_flow}에서는 이러한 과정을
단일 트랜스포머 레이어 관점에서 요약한다.
각 디바이스는 단일 노드 그래프의 \emph{완전한} 복사본을 가지고 있으며,
기울기 All-Reduce 단계만 추가된다.

\begin{figure}[htbp]
  \centering
  \resizebox{\linewidth}{!}{%
\begin{tikzpicture}[
    node distance=2.5cm,
    >=stealth,
    block/.style={rectangle, draw=black, fill=white, text width=5em, text centered, rounded corners, minimum height=8em, font=\bfseries},
    forward/.style={-{Stealth[length=2mm]}, thick, black},
    backward/.style={-{Stealth[length=2mm]}, thick, black, densely dashed},
    dpcomm/.style={<->, thick, red!70, dashed},
    io/.style={text centered, font=\bfseries}
]
    % Title
    \node[font=\Large\bfseries] at (7, 12) {Data Parallel Transformer Flow};

    % ========== Node i (Upper) ==========
    \node (input_i) [io] at (0, 10) {$\mathbf{X}_i$};
    \node (encoding_i) [block, right of=input_i, yshift=-3em] {Input\\Encoding$_i$};
    \node (mha_i) [block, right of=encoding_i] {MHA$_i$};
    \node (mlp_i) [block, right of=mha_i] {FFN$_i$};
    \node (output_i) [block, right of=mlp_i] {Output\\Projection$_i$};
    \node (pred_i) [io, right of=output_i, yshift=3em] {$\mathbf{Y}_i$};
    \node (loss_i) [align=center, io, right of=output_i] {\small LOSS:\\$\mathcal{L}_i$};
    \node (gradient_i) [io, right of=output_i, yshift=-3em] {$\mathbf{dY}_i$};

    % Forward arrows - Node i
    \draw [forward] (input_i) -- ([yshift=3em]encoding_i.west);
    \draw [forward] ([yshift=3em]encoding_i.east) -- ([yshift=3em]mha_i.west);
    \draw [forward] ([yshift=3em]mha_i.east) -- ([yshift=3em]mlp_i.west);
    \draw [forward] ([yshift=3em]mlp_i.east) -- ([yshift=3em]output_i.west);
    \draw [forward] ([yshift=3em]output_i.east) -- (pred_i);
    \draw [forward] (pred_i) -- (loss_i);
    \draw [backward] (loss_i) -- (gradient_i);

    % Backward arrows - Node i
    \draw [backward] (gradient_i) -- ([yshift=-3em]output_i.east);
    \draw [backward] ([yshift=-3em]output_i.west) -- ([yshift=-3em]mlp_i.east);
    \draw [backward] ([yshift=-3em]mlp_i.west) -- ([yshift=-3em]mha_i.east);
    \draw [backward] ([yshift=-3em]mha_i.west) -- ([yshift=-3em]encoding_i.east);

    % Brace for layer repetition - Node i
    \draw[decorate, decoration={brace, amplitude=10pt}]
        ([yshift=1.0em]mha_i.north west) -- ([yshift=1.0em]mlp_i.north east)
        node[midway, above=12pt, font=\normalsize] {$N$ layers};

    % ========== Node j (Lower) ==========
    \node (input_j) [io] at (0, 6) {$\mathbf{X}_j$};
    \node (encoding_j) [block, right of=input_j, yshift=-3em] {Input\\Encoding$_j$};
    \node (mha_j) [block, right of=encoding_j] {MHA$_j$};
    \node (mlp_j) [block, right of=mha_j] {FFN$_j$};
    \node (output_j) [block, right of=mlp_j] {Output\\Projection$_j$};
    \node (pred_j) [io, right of=output_j, yshift=3em] {$\mathbf{Y}_j$};
    \node (loss_j) [align=center, io, right of=output_j] {\small LOSS:\\$\mathcal{L}_j$};
    \node (gradient_j) [io, right of=output_j, yshift=-3em] {$\mathbf{dY}_j$};

    % Forward arrows - Node j
    \draw [forward] (input_j) -- ([yshift=3em]encoding_j.west);
    \draw [forward] ([yshift=3em]encoding_j.east) -- ([yshift=3em]mha_j.west);
    \draw [forward] ([yshift=3em]mha_j.east) -- ([yshift=3em]mlp_j.west);
    \draw [forward] ([yshift=3em]mlp_j.east) -- ([yshift=3em]output_j.west);
    \draw [forward] ([yshift=3em]output_j.east) -- (pred_j);
    \draw [forward] (pred_j) -- (loss_j);
    \draw [backward] (loss_j) -- (gradient_j);

    % Backward arrows - Node j
    \draw [backward] (gradient_j) -- ([yshift=-3em]output_j.east);
    \draw [backward] ([yshift=-3em]output_j.west) -- ([yshift=-3em]mlp_j.east);
    \draw [backward] ([yshift=-3em]mlp_j.west) -- ([yshift=-3em]mha_j.east);
    \draw [backward] ([yshift=-3em]mha_j.west) -- ([yshift=-3em]encoding_j.east);

    % ========== DP Communications ==========
    \draw [dpcomm] (mha_i.south) -- (mha_j.north) node[midway, right, font=\tiny, align=left] {All-Reduce\\$(\nabla W_{Q,K,V,O})$};
    \draw [dpcomm] (mlp_i.south) -- (mlp_j.north) node[midway, right, font=\tiny, align=left] {All-Reduce\\$(\nabla W_{\text{up},\text{down}})$};

    % Labels (Legend)
    \coordinate (legend) at ([xshift=11.5cm, yshift=8.5em]input_i);

    % Forward (작은 화살표 + 작은 글자)
    \draw[
        forward,
        -{Stealth[length=1.2mm,width=1.4mm]}, % 화살표 더 작게
        line width=0.3pt                       % 선 더 얇게
    ] (legend) -- ++(0.8,0)
      node[right, font=\scriptsize] {Forward};

    % Backward
    \draw[
        backward,
        -{Stealth[length=1.2mm,width=1.4mm]},
        line width=0.3pt
    ] ([yshift=-0.4cm]legend) -- ++(0.8,0)
      node[right, font=\scriptsize] {Backward};

    % DP Comm
    \draw[
        dpcomm,
        line width=0.3pt
    ] ([yshift=-0.8cm]legend) -- ++(0.8,0)
      node[right, font=\scriptsize] {DP Comm};
\end{tikzpicture}%
}
  \caption{데이터 병렬화가 적용된 트랜스포머 레이어 전체 구조.
  각 디바이스는 모델 전체를 한 벌씩 가지고 있고,
  서로 다른 입력 배치 조각($\mathbf{X}_i$, $\mathbf{X}_j$, \dots)을 처리한다.
  순전파와 역전파는 단일 노드와 동일하게 로컬에서 수행되며,
  각 블록(MHA, MLP, 출력 프로젝션)의 기울기는 All-Reduce를 통해
  복제본 간에 동기화된다.}
  \label{fig:dp_overall_flow}
\end{figure}

% ------------------------ 7.2 Relation to Single-Node -----------------
\subsection{단일 노드 계산과의 관계}

데이터 병렬화는 Section~\ref{sec:sn}의 단일 노드 계산 위에
\emph{래퍼(wrapper)}를 씌운 것으로 보는 것이 이해에 도움이 된다.

\begin{itemize}
  \item \textbf{복제본별 동일한 계산 그래프.}
        어떤 레이어(MHA, MLP, 출력 프로젝션)를 보더라도,
        한 디바이스에서의 순전파·역전파 그래프는
        단일 노드 그림들
        (Figures~\ref{fig:single_node_mha_forward},
         \ref{fig:single_node_mha_backward},
         \ref{fig:single_node_mlp_forward},
         \ref{fig:single_node_mlp_backward} 등)과 동일하다.
        텐서 병렬화가 활성화된 경우에도,
        Section~\ref{sec:tp}에서 설명한 TP 버전 그래프를 그대로 사용하되,
        입력 배치만 $\mathbf{X}_d$로 바뀐다.
  \item \textbf{다른 미니배치.}
        순전파 관점에서의 유일한 차이는 각 복제본이
        전역 배치의 서로 다른 조각을 본다는 점이다.
        데이터 병렬화에서는 순전파 활성값에 대해
        디바이스 간 통신이 필요 없다.
  \item \textbf{기울기 집계만 추가.}
        역전파가 끝난 뒤에만,
        파라미터 기울기를 All-Reduce를 통해 합산/평균한다.
        그 외의 계산 그래프 구조는 단일 노드와 동일하다.
  \item \textbf{큰 배치와의 동등성.}
        기울기를 복제본들 사이에서 평균하면,
        결과 업데이트는 배치 크기
        $B = N_D \cdot B_{\text{local}}$인 단일 노드 실행과
        수학적으로 동등해진다
        (드롭아웃 노이즈 등 미세한 차이는 무시).
\end{itemize}

반대로, 텐서 병렬화(Section~\ref{sec:tp})는
각 레이어 내부의 그래프 자체를 수정한다.
가중치 행렬을 샤딩하고, 순전파·역전파 중간에 집합 통신을 삽입한다.
데이터 병렬화는 레이어 내부의 그래프를 그대로 두고,
\emph{기울기 수준에서만} 통신을 추가하는 방식이라고 볼 수 있다.

% ------------------------ 7.3 MHA Backward under DP -------------------
\subsection{데이터 병렬 환경에서의 MHA 역전파}

MHA 블록에 대한 데이터 병렬 역전파를 보다 구체적으로 살펴보자.
Section~\ref{sec:sn}의 단일 노드 MHA 역전파에서는,
기울기가 출력 프로젝션, 어텐션 연산, Q/K/V 프로젝션,
레이어 정규화를 거꾸로 따라가며
$\nabla W_Q, \nabla W_K, \nabla W_V, \nabla W_O$ 등을 계산한다.

데이터 병렬 환경에서도 \emph{각 복제본 내부}의 계산 그래프는
그림 구조와 텐서 모양까지 단일 노드 MHA 역전파
(Figure~\ref{fig:single_node_mha_backward})와 동일하다.
차이는 다음과 같다.

\begin{itemize}
  \item 각 디바이스 $d$는 자신의 미니배치 $\mathbf{X}_d$에 대해
        로컬 기울기
        $\nabla W_Q^{(d)}, \nabla W_K^{(d)}, \nabla W_V^{(d)}, \nabla W_O^{(d)}$
        를 계산한다.
  \item 그런 다음, 모든 데이터 병렬 복제본에 대해
        All-Reduce를 수행하여 전역 기울기를 얻는다:
        \[
          \nabla W_Q
            = \frac{1}{N_D}
              \sum_{d=0}^{N_D-1} \nabla W_Q^{(d)}, \quad
          \nabla W_K
            = \frac{1}{N_D}
              \sum_{d=0}^{N_D-1} \nabla W_K^{(d)},
        \]
        \[
          \nabla W_V
            = \frac{1}{N_D}
              \sum_{d=0}^{N_D-1} \nabla W_V^{(d)}, \quad
          \nabla W_O
            = \frac{1}{N_D}
              \sum_{d=0}^{N_D-1} \nabla W_O^{(d)}.
        \]
\end{itemize}

이 동기화 이후에는 각 디바이스가
동일한 평균 기울기
$\nabla W_Q, \nabla W_K, \nabla W_V, \nabla W_O$를 가지게 된다.

Figure~\ref{fig:mha_backward_dp}는 이 과정을 그림으로 나타낸다.
MHA 블록 내부의 노드와 텐서 모양은
Figure~\ref{fig:single_node_mha_backward}과 동일하지만,
추가된 박스와 붉은 점선 화살표가
데이터 병렬 복제본 사이에서 기울기를 All-Reduce하는 위치를 표시한다.

\begin{landscape}
\begin{figure}[p]
  % no \centering here to avoid compilation issues
  \resizebox{\linewidth}{!}{%
\begin{tikzpicture}[
  every node/.style={transform shape},
  >=stealth,
  auxnode/.style={draw, rectangle, fill=white, minimum height=6mm, inner sep=2pt, font=\footnotesize, align=center},
  mulnode/.style={draw, circle, fill=white, minimum size=6mm, font=\footnotesize, align=center},
  addnode/.style={draw, circle, fill=white, minimum size=6mm, font=\footnotesize, align=center},
  sumnode/.style={draw, circle, fill=white, minimum size=6mm, font=\tiny, align=center},
  arnode/.style={draw, rectangle, fill=red!30, minimum height=6mm, inner sep=2pt, font=\footnotesize, align=center, thick, draw=red!70},
  flow/.style={->, thick, black!85},
  flow2/.style={->, double, thick, black!85},
  dimlabel/.style={font=\scriptsize, inner sep=1pt, align=center},
  gradflow/.style={->, thick, black!85},
  gradweight/.style={->, thick, black!85},
  dpgradweight/.style={->, thick, black!85},
  dpcomm/.style={<->, thick, red!70, dashed}
]

\begin{scope}[xscale=1.35, yscale=1.2]

\def\yoffset{-1.0}
\def\dVXyoffset{-6.5}

\coordinate (Grad_Aout_B) at (17.5, \yoffset);
\coordinate (dDrop_center) at (15.9, \yoffset);
\coordinate (ProjGradSplit) at (14.3, \yoffset);
\coordinate (dOProj_center) at (12.7, \yoffset);
\coordinate (C_center) at (11.1, \yoffset);
\coordinate (R_center) at (9.0, \yoffset);
\coordinate (dPV_AS_calc_center) at (7.5, \yoffset);
\coordinate (dSoft_center) at (5.3, \yoffset);
\coordinate (dSM_calc_center) at (3.3, \yoffset);
\coordinate (dV_calc_center) at (8.2, \dVXyoffset+\yoffset);
\coordinate (R_V_bwd_center) at (5.9, \dVXyoffset+\yoffset);
\coordinate (dVX_calc_center) at (3.9, \dVXyoffset+\yoffset);
\coordinate (dQK_calc_Q_center) at (1.7, \yoffset);
\coordinate (dQK_calc_K_center) at (1.7, -3.3+\yoffset);
\coordinate (K_BWD_input_center) at (1.7, 1.0+\yoffset);
\coordinate (T_Q_bwd_center) at (1.7, -4.3+\yoffset);

\node[font=\Large\bfseries] at (8, 4.6+\yoffset) {Multi-Head Attention Backward Pass (Data Parallel)};

\node (Grad_Aout_B) at (18.5, \yoffset) {$\mathbf{dA}_{\text{out}}$\\$[B,S,D]$};
\node[auxnode] (DO) at (dDrop_center) {DO};
\node[mulnode] (dOProj) at (dOProj_center) {$\bullet$};

\node[auxnode] (T_WO) [below=0.9cm of dOProj] {T};
\node[dimlabel] (WO_BWD) [below=0.45cm of T_WO] {$\widetilde{\mathbf{W}}_{O}$\\$[D,D]$};

\node[mulnode] (dWO_calc) at ($(ProjGradSplit)+(0, 1.6)$) {$\bullet$};

% Add AR node before dWO
\node[arnode] (AR_WO) [left=0.8cm of dWO_calc] {AR};
\draw[dpcomm] ([yshift=-0.5cm]AR_WO.south) -- ([yshift=-0.05cm]AR_WO.south);
\draw[dpcomm] ([yshift=0.05cm]AR_WO.north) -- ([yshift=0.5cm]AR_WO.north);
\node[font=\tiny, red!70, align=center] at ([yshift=-0.65cm]AR_WO.south) {DP\\nodes};
\node[font=\tiny, red!70, align=center] at ([yshift=0.65cm]AR_WO.north) {DP\\nodes};

\node[align=center, left=0.8cm of AR_WO]
  (dWO_GRAD) {$\mathbf{d}\widetilde{\mathbf{W}}_{O}$\\$[D,D]$};
\draw[dpgradweight] (dWO_calc) -- (AR_WO);
\draw[gradflow] (AR_WO) -- (dWO_GRAD);

\node[auxnode] (T_AO_in) [right=1.4cm of dWO_calc] {T};
\node[dimlabel] (AO_in_local_label) [right=1.6cm of T_AO_in] {$\mathbf{AO}_{\text{cat}}$\\$[B,S,D]$};

\node[auxnode] (C) at (C_center) {dC};
\node[auxnode] (R) at (R_center) {R};

\node[mulnode] (dPV_AS_calc) at (dPV_AS_calc_center) {$\bullet$};
\node[auxnode] (dSoft) at (dSoft_center) {dS};
\node[auxnode] (dSM_calc) at (dSM_calc_center) {dSM};

\node[mulnode] (dQK_calc_Q) at (dQK_calc_Q_center) {$\bullet$};
\node[mulnode] (dQX_proj_calc) [left=5.8cm of dQK_calc_Q] {$\bullet$};

\node[mulnode] (dQK_calc_K) at (dQK_calc_K_center) {$\bullet$};
\node[mulnode] (dKX_proj_calc) [left=5.8cm of dQK_calc_K] {$\bullet$};

\node[dimlabel] (K_BWD_input) at (K_BWD_input_center) {$\mathbf{K}$\\$[B,N_H,S,D_h]$};
\node[auxnode] (T_Q_bwd) at (T_Q_bwd_center) {T};
\node[dimlabel] (Q_BWD_input) [below=0.35cm of T_Q_bwd] {$\mathbf{Q}$\\$[B,N_H,S,D_h]$};

\node[dimlabel] (V_FWD) [above=1.7cm of dPV_AS_calc] {$\mathbf{V}$\\$[B,N_H,S,D_h]$};
\node[auxnode] (T_V_bwd) [below=0.25cm of V_FWD] {T};

\node[mulnode] (dV_calc) at (dV_calc_center) {$\bullet$};
\node[auxnode] (T_AS_bwd) [right=1.5cm of dV_calc] {T};
\node[dimlabel] (AS_BWD_for_V) [right=0.5cm of T_AS_bwd] {$\mathbf{AS}$\\$[B,N_H,S,S]$};

\node[auxnode] (R_V_bwd) at (R_V_bwd_center) {R};
\node[mulnode] (dVX_calc) at (dVX_calc_center) {$\bullet$};

\node[auxnode] (T_WV) [above=0.35cm of dVX_calc] {T};
\node[dimlabel] (WV_BWD) [above=0.3cm of T_WV] {$\widetilde{\mathbf{W}}_{V}$\\$[D,D]$};

\node[sumnode] (Sum_dBO) [below=1.5cm of ProjGradSplit] {$\sum_{B, S}$};

% Add AR node before dBO
\node[arnode] (AR_BO) [below=0.6cm of Sum_dBO] {AR};
\draw[dpcomm] ([xshift=-0.4cm]AR_BO.west) -- ([xshift=-0.05cm]AR_BO.west);
\draw[dpcomm] ([xshift=0.05cm]AR_BO.east) -- ([xshift=0.4cm]AR_BO.east);
\node[font=\tiny, red!70, align=center] at ([xshift=-0.55cm]AR_BO.west) {DP\\nodes};
\node[font=\tiny, red!70, align=center] at ([xshift=0.55cm]AR_BO.east) {DP\\nodes};

\node[align=center] (dBO) [below=0.6cm of AR_BO] {$\mathbf{d}\widetilde{\mathbf{b}}_{O}$\\$[D]$};
\draw[dpgradweight] (Sum_dBO) -- (AR_BO);
\draw[gradflow] (AR_BO) -- (dBO);

\draw[gradflow] (Grad_Aout_B) -- (DO)
  node[dimlabel, midway, above]{$\mathbf{dA}_{\text{out}}$\\$[B,S,D]$};

\draw[gradflow] (DO) -- (dOProj)
  node[dimlabel, pos=0.25, above]{$\mathbf{dAO}_{\text{bias}}$\\$=\mathbf{dAO}_{\text{lin}}$\\$[B,S,D]$};

\draw[gradflow] (ProjGradSplit) -- (dWO_calc.south);
\draw[gradflow] (ProjGradSplit) -- ([yshift=-0.75cm]ProjGradSplit) -| (Sum_dBO.north);

\draw[gradflow] (dOProj) -- (C)
  node[dimlabel, midway, above]{$\mathbf{dAO}_{\text{cat}}$\\$[B,S,D]$};
\draw[gradflow] (C) -- (R)
  node[dimlabel, midway, above]{$[B,S,N_H,D_h]$};

\coordinate (R_split_point) at ($(dPV_AS_calc)!0.5!(R)$);
\draw[gradflow] (R.west) -- (dPV_AS_calc.east);
\draw[flow2] (R_split_point) -- (dV_calc.north)
  node[dimlabel, midway, right]{$\mathbf{dAO}_{\text{heads}}$\\$[B,N_H,S,D_h]$};

\draw[gradflow] (V_FWD.south) -- (T_V_bwd.north);
\draw[flow2] (T_V_bwd.south) -- (dPV_AS_calc.north)
  node[dimlabel, midway, right]{$[B,N_H,D_h,S]$};
\draw[gradflow] (dPV_AS_calc.west) -- (dSoft.east)
  node[dimlabel, midway, above]{$\mathbf{dAS}$\\$[B,N_H,S,S]$};

\node (AS_BWD_dS) [dimlabel, above=0.5cm of dSoft] {$\mathbf{AS}$\\$[B,N_H,S,S]$};
\draw[gradflow] (AS_BWD_dS.south) -- (dSoft.north);
\draw[gradflow] (dSoft.west) -- (dSM_calc.east)
  node[dimlabel, midway, above]{$[B,N_H,S,S]$};

\coordinate (dA_Split_X) at ($(dSM_calc_center)!0.5!(dQK_calc_Q_center)$);
\coordinate (dA_Split) at (dA_Split_X |- dQK_calc_Q.east);
\draw[gradflow] (dSM_calc.west) -- (dQK_calc_Q.east);
\draw[flow2] (dA_Split) -- (dA_Split |- dQK_calc_K.east) -- (dQK_calc_K.east)
  node[dimlabel, pos=-1.5, above, yshift=15]{$\mathbf{dA}$\\$[B,N_H,S,S]$};

\draw[flow2] (K_BWD_input.south) -- (dQK_calc_Q.north);
\draw[gradweight] (dQK_calc_Q) -- (dQX_proj_calc)
  node[dimlabel, midway, below]{$\mathbf{dQ}$\\$[B,N_H,S,D_h]$};

\node[auxnode] (T_WQ_bwd) [above=0.45cm of dQX_proj_calc] {T};
\node[dimlabel] (WQ_bwd) [above=0.45cm of T_WQ_bwd] {$\widetilde{\mathbf{W}}_{Q}$\\$[D,D]$};
\draw[flow] (WQ_bwd) -- (T_WQ_bwd);
\draw[flow2] (T_WQ_bwd.south) -- (dQX_proj_calc.north)
  node[dimlabel, midway, right]{$[D,D]$};

\draw[flow] (Q_BWD_input.north) -- (T_Q_bwd.south);
\draw[flow] (T_Q_bwd.north) -- (dQK_calc_K.south)
  node[dimlabel, pos=0.55, right]{$[B,N_H,D_h,S]$};

\node[auxnode] (T_dK) at ($(dQK_calc_K)!0.35!(dKX_proj_calc)$) {T};
\node[auxnode] (R_dK_mid) at ($(T_dK)!0.5!(dKX_proj_calc)$) {R};

\draw[gradweight] (dQK_calc_K) -- (T_dK)
  node[dimlabel, midway, above]{$\mathbf{dK^T}$\\$[B,N_H,D_h,S]$};
\draw[gradweight] (T_dK) -- (R_dK_mid)
  node[dimlabel, midway, above]{$[B,S,N_H,D_h]$};
\draw[gradweight] (R_dK_mid) -- (dKX_proj_calc)
  node[dimlabel, midway, above]{$\mathbf{dK}$\\$[B,S,D]$};

\node[auxnode] (T_WK_bwd) [above=0.55cm of dKX_proj_calc] {T};
\node[dimlabel] (WK_bwd) [above=0.45cm of T_WK_bwd] {$\widetilde{\mathbf{W}}_{K}$\\$[D,D]$};
\draw[gradflow] (WK_bwd) -- (T_WK_bwd);
\draw[flow2] (T_WK_bwd.south) -- (dKX_proj_calc.north);
\draw[gradflow] (AS_BWD_for_V.west) -- (T_AS_bwd.east);
\draw[gradflow] (T_AS_bwd.west) -- (dV_calc.east)
  node[dimlabel, midway, above]{$[B,N_H,S,S]$};
\draw[gradflow] (dV_calc.west) -- (R_V_bwd.east)
  node[dimlabel, midway, above]{$\mathbf{dV}$\\$[B,N_H,S,D_h]$};
\draw[gradflow] (R_V_bwd) -- (dVX_calc.east)
  node[dimlabel, midway, above]{$[B,S,D]$};

\draw[gradflow] (WV_BWD) -- (T_WV);
\draw[flow2] (T_WV) -- (dVX_calc.north)
  node[dimlabel, midway, right]{$[D,D]$};

\node[addnode] (Sum_dXnorm) [left=1.8cm of dKX_proj_calc] {$+$};

\draw[gradweight] (dQX_proj_calc.west) -| node[dimlabel, pos=0.7, left]{$\mathbf{dX}_{\text{norm},\mathbf{Q}}$\\$[B,S,D]$} (Sum_dXnorm.north);
\draw[gradweight] (dKX_proj_calc.west) -- node[dimlabel, midway, above]{$\mathbf{dX}_{\text{norm},\mathbf{K}}$\\$[B,S,D]$} (Sum_dXnorm.east);
\draw[gradweight] (dVX_calc.west) -| node[dimlabel, pos=0.9, left]{$\mathbf{dX}_{\text{norm},\mathbf{V}}$\\$[B,S,D]$} (Sum_dXnorm.south);

\coordinate (dV_branch) at ($(R_V_bwd.west)!0.52!(dVX_calc.east)$);
\node[mulnode] (dWV_mul) at ($(dV_branch)+(0,-1.6cm)$) {$\bullet$};
\draw[flow2] (dV_branch) -- (dWV_mul.north);

\node[auxnode] (T_Xnorm) [right=1.1cm of dWV_mul] {T};
\node[dimlabel] (Xnorm_local) [right=0.8cm of T_Xnorm] {$\mathbf{X}_{\text{norm}}$\\$[B,S,D]$};
\draw[gradflow] (Xnorm_local) -- (T_Xnorm);
\draw[gradflow] (T_Xnorm.west) -- (dWV_mul.east)
  node[dimlabel, midway, above]{$[B,D,S]$};

% Add AR node before dWV
\node[arnode] (AR_WV) [left=0.8cm of dWV_mul] {AR};
\draw[dpcomm] ([yshift=-0.5cm]AR_WV.south) -- ([yshift=-0.05cm]AR_WV.south);
\draw[dpcomm] ([yshift=0.05cm]AR_WV.north) -- ([yshift=0.5cm]AR_WV.north);
\node[font=\tiny, red!70, align=center] at ([yshift=-0.65cm]AR_WV.south) {DP\\nodes};
\node[font=\tiny, red!70, align=center] at ([yshift=0.65cm]AR_WV.north) {DP\\nodes};

\node[align=center] (dWV_out) [left=0.8cm of AR_WV] {$\mathbf{d}\widetilde{\mathbf{W}}_{V}$\\$[D,D]$};
\draw[dpgradweight] (dWV_mul.west) -- (AR_WV);
\draw[gradflow] (AR_WV) -- (dWV_out);

\coordinate (dQ_branch) at ($(dQK_calc_Q.east)!0.50!(dQX_proj_calc.west)$);
\node[mulnode] (dWQ_mul) at ($(dQ_branch)+(0,3.5cm)$) {$\bullet$};
\node[auxnode] (R_dQ_for_WQ) at ($(dWQ_mul)+(0,-1.0cm)$) {R};
\draw[gradflow]  (dQ_branch) -- (R_dQ_for_WQ.south);
\draw[flow2] (R_dQ_for_WQ.north) -- (dWQ_mul.south)
  node[dimlabel, midway, right]{$[B,S,D]$};

\node[auxnode] (T_XnormQ) [right=1.1cm of dWQ_mul] {T};
\node[dimlabel] (Xnorm_localQ) [right=0.5cm of T_XnormQ] {$\mathbf{X}_{\text{norm}}$\\$[B,S,D]$};
\draw[gradflow] (Xnorm_localQ) -- (T_XnormQ);
\draw[gradflow] (T_XnormQ.west) -- (dWQ_mul.east)
  node[dimlabel, midway, above]{$[B,D,S]$};

% Add AR node before dWQ
\node[arnode] (AR_WQ) [left=0.8cm of dWQ_mul] {AR};
\draw[dpcomm] ([yshift=-0.5cm]AR_WQ.south) -- ([yshift=-0.05cm]AR_WQ.south);
\draw[dpcomm] ([yshift=0.05cm]AR_WQ.north) -- ([yshift=0.5cm]AR_WQ.north);
\node[font=\tiny, red!70, align=center] at ([yshift=-0.65cm]AR_WQ.south) {DP\\nodes};
\node[font=\tiny, red!70, align=center] at ([yshift=0.65cm]AR_WQ.north) {DP\\nodes};

\node[align=center] (dWQ_out) [left=0.8cm of AR_WQ] {$\mathbf{d}\widetilde{\mathbf{W}}_{Q}$\\$[D,D]$};
\draw[dpgradweight] (dWQ_mul.west) -- (AR_WQ);
\draw[gradflow] (AR_WQ) -- (dWQ_out);

\coordinate (dK_branch) at ($(R_dK_mid)!0.52!(dKX_proj_calc)$);
\node[mulnode] (dWK_mul) at ($(dK_branch)+(0,-1.7cm)$) {$\bullet$};
\draw[flow2]  (dK_branch) -- (dWK_mul.north);

\node[auxnode] (T_XnormK) [right=1.2cm of dWK_mul] {T};
\node[dimlabel, right=0.5cm of T_XnormK] (Xnorm_localK) {$\mathbf{X}_{\text{norm}}$\\$[B,S,D]$};
\draw[gradflow] (Xnorm_localK) -- (T_XnormK);
\draw[gradflow] (T_XnormK.west) -- (dWK_mul.east)
  node[dimlabel, midway, above]{$[B,D,S]$};

% Add AR node before dWK
\node[arnode] (AR_WK) [left=0.4cm of dWK_mul] {AR};
\draw[dpcomm] ([yshift=-0.5cm]AR_WK.south) -- ([yshift=-0.05cm]AR_WK.south);
\draw[dpcomm] ([yshift=0.05cm]AR_WK.north) -- ([yshift=0.5cm]AR_WK.north);
\node[font=\tiny, red!70, align=center] at ([yshift=-0.65cm]AR_WK.south) {DP\\nodes};
\node[font=\tiny, red!70, align=center] at ([yshift=0.65cm]AR_WK.north) {DP\\nodes};

\node[align=center] (dWK_out) [left=0.4cm of AR_WK] {$\mathbf{d}\widetilde{\mathbf{W}}_{K}$\\$[D,D]$};
\draw[dpgradweight] (dWK_mul.west) -- (AR_WK);
\draw[gradflow] (AR_WK) -- (dWK_out);

\draw[gradflow] (WO_BWD) -- (T_WO);
\draw[flow2] (T_WO) -- (dOProj)
  node[dimlabel, midway, right]{$\widetilde{\mathbf{W}}_{O}^{T}$\\$[D,D]$};
\draw[gradflow] (AO_in_local_label) -- (T_AO_in);
\draw[flow2] (T_AO_in) -- (dWO_calc.east)
  node[dimlabel, midway, above]{$[B,D,S]$};

\node[auxnode] (dLN) [left=1.7cm of Sum_dXnorm] {dLN};
\draw[gradweight] (Sum_dXnorm.west) -- node[dimlabel, midway, above]
  {$\mathbf{dX}_{\text{norm}}$\\$[B,S,D]$} (dLN.east);

\node (dX_OUT) [align=center, left=1.0cm of dLN] {$\mathbf{dX}$\\$[B,S,D]$};
\draw[gradweight] (dLN.west) -- (dX_OUT);

\node[dimlabel] (LNCache) [below=0.9cm of dLN] {$\mathbf{X}$\\$[B,S,D]$};
\draw[gradflow] (LNCache.north) -- (dLN.south);

% Legend and explanation
    \node[align=left, font=\scriptsize, text width=8cm] at (-5, -9.5) {
      \textbf{AR (All-Reduce)} synchronizes weight gradients across data parallel nodes\\[4pt]
      \textbf{All-Reduce Communication Cost per MHA layer:}\\
      \quad • Total parameters: $4D^2 + D$ (W$_Q$, W$_K$, W$_V$, W$_O$, b$_O$)\\
      \quad • \textbf{Naive:} $2(N_{DP}-1) \times (4D^2 + D)$ per node\\
      \quad • \textbf{Ring:} $2\frac{N_{DP}-1}{N_{DP}} \times (4D^2 + D)$ per node\\[2pt]
      {\scriptsize (Gradients averaged across $N_{DP}$ data parallel nodes)}
    };

\end{scope}
\end{tikzpicture}
}
  \caption{데이터 병렬 환경에서의 멀티헤드 어텐션 역전파.
  각 복제본은 자신의 미니배치를 사용해
  $W_Q$, $W_K$, $W_V$, $W_O$에 대한 로컬 기울기를 계산한다.
  붉은 점선 화살표는 모든 데이터 병렬 복제본에 걸쳐
  이러한 로컬 기울기를 All-Reduce하여,
  옵티마이저에서 사용하는 전역 기울기를 형성하는 위치를 나타낸다.
  블록 내부의 역전파 그래프 구조는 단일 노드 경우와 동일하다.}
  \label{fig:mha_backward_dp}
\end{figure}
\end{landscape}

% ------------------------ 7.4 MLP Backward under DP -------------------
\subsection{데이터 병렬 환경에서의 MLP 역전파}

MLP 블록에 대한 상황도 거의 동일하다.
각 복제본은 단일 노드 MLP 역전파
(Figure~\ref{fig:single_node_mlp_backward})에서와 똑같이,
기울기를
$\mathrm{d}\mathbf{Y}$에서 시작해
down-projection, 활성 함수, up-projection,
(필요하다면) 레이어 정규화를 거쳐 다시 $\mathbf{H}$까지 전파하면서,
다음 파라미터들에 대한 기울기를 누적한다:
\[
  W_{\text{up}}, W_{\text{down}},
  \mathbf{b}_{\text{up}}, \mathbf{b}_{\text{down}}.
\]

데이터 병렬 환경에서는 각 복제본 $d$가
\[
  \nabla W_{\text{up}}^{(d)}, \quad
  \nabla W_{\text{down}}^{(d)}
\]
과 같은 로컬 기울기를 얻고,
bias에 대해서도 마찬가지이다.
이 기울기들은 복제본 전체에 대해 All-Reduce를 통해 동기화된다:
\[
  \nabla W_{\text{up}}
    = \frac{1}{N_D} \sum_{d=0}^{N_D-1} \nabla W_{\text{up}}^{(d)},\quad
  \nabla W_{\text{down}}
    = \frac{1}{N_D} \sum_{d=0}^{N_D-1} \nabla W_{\text{down}}^{(d)}.
\]
bias 기울기도 이와 동일한 방식으로 처리된다.

입력 $\mathbf{H}$는 모든 디바이스에 복제되어 있으므로,
$\mathrm{d}\mathbf{H}$ 자체를 위한 추가 통신은 필요 없다.
각 복제본은 동일한 역전파 함수를 서로 다른 데이터에 대해 적용할 뿐이며,
파라미터 기울기 측면에서의 “집계 효과”는
이들 기울기를 평균 내는 All-Reduce 단계에 모두 포함된다.

Figure~\ref{fig:mlp_backward_dp}는
MLP 역전파 그래프에서 이러한 All-Reduce 위치를 표시한다.

\begin{figure}[p]
  % no \centering here to avoid compilation issues
  \noindent
\resizebox{\linewidth}{!}{%
\begin{tikzpicture}[
    >=stealth,
    auxnode/.style={draw, rectangle, fill=white, minimum height=6mm, inner sep=2pt, font=\footnotesize, align=center},
    mulnode/.style={draw, circle, fill=white, minimum size=6mm, font=\footnotesize, align=center},
    addnode/.style={draw, circle, fill=white, minimum size=6mm, font=\footnotesize, align=center},
    sumnode/.style={draw, circle, fill=white, minimum size=6mm, font=\tiny, align=center},
    arnode/.style={draw, rectangle, fill=red!30, minimum height=6mm, inner sep=2pt, font=\footnotesize, align=center, thick, draw=red!70},
    flow_rev/.style={<-, thick, black!85},
    flow_dw/.style={->, thick, black!85},
    flow_act/.style={double, ->, thick, black!85},
    dimlabel/.style={font=\tiny, inner sep=1pt, align=center},
    gradlabel/.style={font=\tiny\bfseries, inner sep=1pt, align=center},
    dpgradweight/.style={->, thick, black!85},
    dpcomm/.style={<->, thick, red!70, dashed}
]
    \node[font=\Large\bfseries] at (5, 10) {MLP Backward Pass (Data Parallel)};

    \pgfmathsetmacro{\backwardoffset}{0.0}

    \node (d_MOut) at (12.6, \backwardoffset) {$\mathrm{d}\mathbf{Y}$};
    \node[auxnode] (d_Drop2) [left=1.8cm of d_MOut] {dDO};
    \draw[flow_rev] (d_Drop2) -- (d_MOut)
      node[dimlabel, midway, below]{\shortstack{$\mathrm{d}\mathbf{Y}$\\$[B,S,D]$}};

    \coordinate (split2) at ($(d_Drop2.west) + (-1.5cm, 0)$);
    \coordinate (branch_dUproj) at ($(split2) + (-1.2cm, 0)$);

    \node[sumnode] (d_SumB2) [above=0.8cm of split2] {$\sum_{B, S}$};
    \node (d_Bdown) [above=0.8cm of d_SumB2] {$\mathrm{d}\tilde{\mathbf{b}}_{\text{down}}$};
    \draw[dpgradweight] (d_SumB2) -- (d_Bdown) node[dimlabel, midway, right]{$[D]$};

    \draw[flow_rev] (d_SumB2) -- (split2);

    \node[mulnode] (d_L2Mul_in) [left=2.2cm of split2] {$\bullet$};
    \draw[flow_rev] (d_L2Mul_in) -- (d_Drop2)
      node[gradlabel, midway, below]{\shortstack{$\mathrm{d}\mathbf{Z}_{\text{down}}=\mathrm{d}\mathbf{A}_{\text{out}}$\\$[B,S,D]$}};

    \node (W_down_T) [below=1.0cm of d_L2Mul_in] {$\tilde{\mathbf{W}}_{\text{down}}^{T}$};
    \draw[flow_act] (W_down_T.north) -- (d_L2Mul_in)
      node[dimlabel, midway, right]{$[D, D_{ff}]$};

    \coordinate (L2Mul_w_y) at ($(d_L2Mul_in) + (0, 3.5cm)$);
    \node[mulnode] (d_L2Mul_w) at (L2Mul_w_y) {$\bullet$};

    % Add AR node before d_Wdown
    \node[arnode] (AR_down) [above=0.8cm of d_L2Mul_w] {AR};

    % Add communication arrows for AR_down
    \draw[dpcomm] ([xshift=-1.2cm]AR_down.west) -- ([xshift=-0.1cm]AR_down.west);
    \draw[dpcomm] ([xshift=0.1cm]AR_down.east) -- ([xshift=1.2cm]AR_down.east);
    \node[font=\tiny, red!70, align=center] at ([xshift=-1.5cm]AR_down.west) {DP\\nodes};
    \node[font=\tiny, red!70, align=center] at ([xshift=1.5cm]AR_down.east) {DP\\nodes};
    \node[font=\tiny, red!70, below=0.1cm of AR_down] {$2DD_{ff}$};

    \node (d_Wdown) [above=0.8cm of AR_down] {$\mathrm{d}\tilde{\mathbf{W}}_{\text{down}}$};
    \draw[dpgradweight] (d_L2Mul_w) -- (AR_down) node[dimlabel, midway, right]{$[D_{ff}, D]$};
    \draw[flow_dw] (AR_down) -- (d_Wdown) node[dimlabel, midway, right]{$[D_{ff}, D]$};

    \draw[flow_act] (branch_dUproj.north) |- (d_L2Mul_w.east);

    \node[auxnode] (Uin_T) at ($(d_L2Mul_w.west) + (-1.5cm, 0)$) {T};
    \draw[flow_dw] (Uin_T) -- (d_L2Mul_w)
      node[dimlabel, midway, below]{\shortstack{$\mathbf{U}_{\text{in}}^T$\\$[B, D_{ff}, S]$}};
    \node (Uin_aux) [left=1.8cm of Uin_T] {$\mathbf{U}_{\text{in}}$};
    \draw[flow_dw] (Uin_aux) -- (Uin_T) node[dimlabel, midway, above]{\shortstack{$[B,S,D_{ff}]$}};

    \node[auxnode] (d_Drop1) [left=1.8cm of d_L2Mul_in] {dDO};
    \draw[flow_rev] (d_Drop1) -- (d_L2Mul_in)
      node[dimlabel, midway, below]{\shortstack{$\mathrm{d}\mathbf{U}_{\text{in}}$\\$[B,S,D_{ff}]$}};

    \node[auxnode] (d_Act) [left=1.8cm of d_Drop1] {dGL};
    \draw[flow_rev] (d_Act) -- (d_Drop1)
      node[dimlabel, midway, below]{\shortstack{$\mathrm{d}\mathbf{H}_{\text{inter}}$\\$[B,S,D_{ff}]$}};

    \coordinate (split1) at ($(d_Act.west) + (-1.5cm, 0)$);
    \coordinate (branch_dHpre) at ($(split1) + (-1.2cm, 0)$);

    \node[sumnode] (d_SumB1) [above=0.8cm of split1] {$\sum_{B, S}$};
    \node (d_Bup) [above=0.8cm of d_SumB1] {$\mathrm{d}\tilde{\mathbf{b}}_{\text{up}}$};
    \draw[dpgradweight] (d_SumB1) -- (d_Bup) node[dimlabel, midway, right]{$[D_{ff}]$};

    \draw[flow_rev] (d_SumB1) -- (split1);

    \node[mulnode] (d_L1Mul_in) [left=2.2cm of split1] {$\bullet$};
    \draw[flow_rev] (d_L1Mul_in) -- (d_Act)
      node[gradlabel, midway, below]{\shortstack{$\mathrm{d}\mathbf{Z}_{\text{up}}=\mathrm{d}\mathbf{A}_{\text{up}}$\\$[B,S,D_{ff}]$}};

    \node (W_up_T) [below=1.0cm of d_L1Mul_in] {$\tilde{\mathbf{W}}_{\text{up}}^{T}$};
    \draw[flow_act] (W_up_T.north) -- (d_L1Mul_in)
      node[dimlabel, midway, right]{$[D_{ff}, D]$};

    \coordinate (L1Mul_w_y) at ($(d_L1Mul_in) + (0, 3.5cm)$);
    \node[mulnode] (d_L1Mul_w) at (L1Mul_w_y) {$\bullet$};

    % Add AR node before d_Wup
    \node[arnode] (AR_up) [above=0.8cm of d_L1Mul_w] {AR};

    % Add communication arrows for AR_up
    \draw[dpcomm] ([xshift=-1.2cm]AR_up.west) -- ([xshift=-0.1cm]AR_up.west);
    \draw[dpcomm] ([xshift=0.1cm]AR_up.east) -- ([xshift=1.2cm]AR_up.east);
    \node[font=\tiny, red!70, align=center] at ([xshift=-1.5cm]AR_up.west) {DP\\nodes};
    \node[font=\tiny, red!70, align=center] at ([xshift=1.5cm]AR_up.east) {DP\\nodes};
    \node[font=\tiny, red!70, below=0.1cm of AR_up] {$2DD_{ff}$};

    \node (d_Wup) [above=0.8cm of AR_up] {$\mathrm{d}\tilde{\mathbf{W}}_{\text{up}}$};
    \draw[dpgradweight] (d_L1Mul_w) -- (AR_up) node[dimlabel, midway, right]{$[D, D_{ff}]$};
    \draw[flow_dw] (AR_up) -- (d_Wup) node[dimlabel, midway, right]{$[D, D_{ff}]$};

    \draw[flow_act] (branch_dHpre.north) |- (d_L1Mul_w.east);

    \node[auxnode] (Znorm_T) at ($(d_L1Mul_w.west) + (-1.5cm, 0)$) {T};
    \draw[flow_dw] (Znorm_T) -- (d_L1Mul_w)
      node[dimlabel, midway, below]{\shortstack{$\mathbf{H}^T$\\$[B, D, S]$}};
    \node (Znorm_aux) [left=1.8cm of Znorm_T] {$\mathbf{H}$};
    \draw[flow_dw] (Znorm_aux) -- (Znorm_T) node[dimlabel, midway, above]{\shortstack{$[B,S,D]$}};

    \node[auxnode] (d_LN2) [left=1.8cm of d_L1Mul_in] {dLN};
    \draw[flow_rev] (d_LN2) -- (d_L1Mul_in)
      node[dimlabel, midway, below]{\shortstack{$\mathrm{d}\mathbf{H}$\\$[B,S,D]$}};

    \node (d_MIn) [left=1.8cm of d_LN2] {$\mathrm{d}\mathbf{X}$};
    \draw[flow_rev] (d_MIn) -- (d_LN2)
      node[dimlabel, midway, below]{\shortstack{$\mathrm{d}\mathbf{X}$\\$[B,S,D]$}};

    % Legend and explanation
    \node[align=left, font=\scriptsize, text width=8cm] at (-2, -3.5) {
      \textbf{AR (All-Reduce)} synchronizes weight gradients across data parallel nodes\\[4pt]
      \textbf{MLP All-Reduce Cost:} $\sim 2DD_{ff}$ parameters (W$_{\text{up}}$, W$_{\text{down}}$)\\
      \quad • \textbf{Naive:} $2(N_{DP}-1) \times 2DD_{ff}$ per node\\
      \quad • \textbf{Ring:} $2\frac{N_{DP}-1}{N_{DP}} \times 2DD_{ff}$ per node\\[2pt]
      {\scriptsize (Gradients averaged across $N_{DP}$ data parallel nodes)}
    };

\end{tikzpicture}%
}
  \caption{데이터 병렬 환경에서의 MLP 역전파.
  각 복제본은 자신의 미니배치를 사용해
  up-/down-projection 가중치와 bias에 대한 로컬 기울기를 계산한다.
  붉은 박스와 점선 화살표는 데이터 병렬 복제본 전체에 걸쳐
  이러한 기울기를 All-Reduce하여 전역 기울기를 형성하는 위치를 나타낸다.}
  \label{fig:mlp_backward_dp}
\end{figure}

% ------------------------ 7.5 Communication and Memory ----------------
\subsection{통신 및 메모리 고려사항}

마지막으로, 데이터 병렬화에서의 통신 비용과 메모리 요구사항을
단일 노드 모델과 비교해 보자.

\begin{itemize}
  \item \textbf{디바이스당 메모리.}
        각 디바이스는 모델 파라미터와 옵티마이저 상태의
        \emph{전체 복사본}을 저장한다.
        대신, 각 복제본이 보는 배치 크기가 $B_{\text{local}}$이므로,
        활성값(activation) 메모리는 대략 $N_D$배 감소한다.
  \item \textbf{통신 패턴.}
        데이터 병렬화가 도입하는 통신은
        파라미터 기울기를 동기화할 때의 All-Reduce뿐이다.
        순전파 경로나 활성값에 대해서는 통신이 없다.
  \item \textbf{확장성.}
        $N_D$를 늘리면 효과적인 배치 크기가 커지고,
        디바이스당 계산량과 활성 메모리는 줄어든다.
        반면, All-Reduce의 비용은 복제본 수와
        전체 파라미터 크기에 비례해 증가한다.
  \item \textbf{다른 병렬화 기법과의 결합.}
        실무에서는 데이터 병렬화가 텐서 병렬화(때로는 파이프라인 병렬화)와
        함께 사용되는 경우가 많다.
        이러한 설정에서는 각 데이터 병렬 그룹이
        여러 텐서 병렬 shard로 구성되며,
        기울기 동기화(All-Reduce)는 데이터 병렬 그룹 간에 수행되고,
        텐서 병렬 집합 통신은 각 그룹 내부에 국한된다.
\end{itemize}

Section~\ref{sec:sn}에서의 계산 그래프 관점에서 보면,
데이터 병렬화는 가장 “침습성이 낮은(least invasive)” 병렬화 방식이다.
레이어별 순전파·역전파 구조는 그대로 유지하면서,
그 위에 \emph{기울기 All-Reduce}만을 얹어놓는다고 볼 수 있다.
